\documentclass[landscape,12pts]{seminar}
% pour fixer le format A4 
\usepackage{sem-a4}
\usepackage[french]{babel}
% pour supprimer les contours du transparent
\slideframe{}
% pour pouvoir mettre dans les ent\^etes ce qui m'int\'eresse
\usepackage{slidesec}
\pagestyle{myheadings}
%------------------------------------------------------------------------------
\input{../../../Modele/CoursStyle}
%------------------------------------------------------------------------------
\begin{document}
%------------------------------------------------------------------------------
\begin{slide}
  \listofslides
\end{slide}
%------------------------------------------------------------------------------
\begin{slide}
 \slideheading{Introduction}%
 
 Emacs est un \'editeur de texte puissant au point d'\^etre un
 environnement complet (les mauvaises langues soutiennent que c'est un
 \'editeur qui se prend pour un OS). Il disponible gratuitement sous
 UNIX et a \'et\'e d\'evelopp\'e par R.M.\ Stallman du MIT.

\end{slide}
%------------------------------------------------------------------------------
\begin{slide}
  \slideheading{Cadre et tampon}%
  L'outil graphique qu'emacs met \`a votre disposition \`a son d\'emarrage
  est un cadre (frame). Vous pouvez avoir autant de cadre que vous le
  d\'esirez.
  
  Un cadre est constitu\'e de fen\^etres (window). Vous pouvez
  fractionner votre cadre en autant de fen\^etre que vous le
  souhait\'e (horizontalement CTRL-X 2 et verticalement CTRL-X 3).
  
  Chaque fen\^etre est associ\'ee \`a un tampon (buffer) qui contient
  le texte que vous manipulez. C'est ce tampon qui est sauv\'e dans un
  fichier sur demande.
  
\end{slide}
%------------------------------------------------------------------------------
\end{document}
%------------------------------------------------------------------------------
\begin{slide}
 \slideheading{}%
\end{slide}
%------------------------------------------------------------------------------

