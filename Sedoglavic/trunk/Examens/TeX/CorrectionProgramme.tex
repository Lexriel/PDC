\section{Correction de programme}
Cet exercice est uniquement un exercice de lecture et de
compr\'ehension de code. On ne vous demande pas de modifier
consid\'erablement ce dernier.  \par Consid\'erons le programme de la
figure~\ref{fig:Affichage}.
\begin{enumerate}
\item Ce programme ne compile pas. Pourquoi~?
\item \`A quoi sert le test
\begin{verbatim}
 if (strlen(chn) < FOO) {
   fprintf(stderr, "A quoi sert ce test~?\n");
   return;
 }
\end{verbatim}
\item \`A quoi sert le test
\begin{verbatim}
   if(!(b[i] = (char *) malloc(sizeof(char) * (sizeof(chn)+1))) ){
      fprintf(stderr, "A quoi sert ce test~?\n");
      return;
   }
\end{verbatim}
\item Une fois ce probl\`eme corrig\'e, on constate que ce programme
  s'interrompt apr\`es l'affichage suivant~:
\begin{verbatim}
espoir % a.out
Le C c'est f
e C c'est fu
 C c'est fun
C c'est fun.
 c'est fun.L
c'est fun.Le
'est fun.Le 
est fun.Le C
st fun.Le C 
t fun.Le C c
espoir % 
\end{verbatim}
  alors que l'on s'attendait \`a afficher plusieurs fois la ligne
  compl\`ete (avec l'effet de d\'ecalage).
  \par
  Explicitez le probl\`eme et proposez une solution.
\item \'Enoncez une critique concernant l'allocation m\'emoire de la
  fonction~\verb+Affichage+ (surtout si elle est utilis\'ee dans des
  programmes plus grand).
\end{enumerate}
\begin{figure}[htbp]
  \centering
\begin{verbatim}
#include <stdio.h>
#include <stdlib.h>
#include <string.h>
#define FOO 15 ;

void Affichage(char * chn){
 int i = 0 ;
 char * b[FOO], *dest_ptr, *src_ptr;
 
 if (strlen(chn) < FOO) {
   fprintf(stderr, "A quoi sert ce test~?\n");
   return;
 }

 for (; i<FOO ; i++) {
   if(!(b[i] = (char *) malloc(sizeof(char) * (sizeof(chn)+1))) ){
      fprintf(stderr, "A quoi sert ce test~?\n");
      return;
   }
   for (dest_ptr = b[i], src_ptr = chn+i; (*src_ptr) != 0; )
      *dest_ptr++ = *src_ptr++;
   for (src_ptr = chn; src_ptr != chn+i; )
      *dest_ptr++ = *src_ptr++;
   *dest_ptr = 0;
 }

 for (i=0;i<FOO;i++) 
   printf("%s\n",b[i]);
}

int main(void){
   Affichage("Le C c'est fun.");
   return 0 ;
}
\end{verbatim}
  \caption{Un code d'affichage.}
  \label{fig:Affichage}
\end{figure}
