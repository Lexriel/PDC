\section{Makefile}
On dispose d'un r\'epertoire dont le contenu est~:
\begin{verbatim}
% ls 
Makefile foo.c foo.o main.c main.o 
module.c module.h module.o prog
\end{verbatim}
Le fichier \verb+Makefile+ contient le code suivant~:
\begin{verbatim}
% cat Makefile 
CC=gcc 
CFLAGS = -Wall -ansi -pedantic
OBJS = main.o module.o foo.o

.PHONY: doit clean

prog: $(OBJS)
     $(CC) -o prog $(OBJS) -lm

main.o: main.c module.h
     $(CC) -c $(CFLAGS) main.c

clean:
    -rm *.o

doit:
     make prog 
     rm -f prog 
     make clean
\end{verbatim}
\paragraph{Questions.}
\begin{enumerate}
\item En supposant qu'il n'y a pas d'erreur de compilation, d'\'edition de
liens ou d'acc\`es aux fichiers, donner l'ensemble des commandes que le
shell ex\'ecute si l'utilisateur tape l'instruction~:
\begin{verbatim}
make doit
\end{verbatim}
\item Que se passe-t-il si on recommence cette instruction~? Justifier.
\item Modifiez ce Makefile pour corriger le probl\`eme (\verb+foo.c+ est le seul source n'incluant pas l'ent\^ete \verb?module.h?).
\end{enumerate}

\ifcorrection
\paragraph{Corrections.}
\begin{enumerate}
\item la compilation de tous les objets, la suppression de
  l'ex\'ecutable produit et de tous les objets.
\item Comme tous les objets ont \'et\'e d\'etruits et que \verb+foo.o+
  et \verb+module.o+ ne sont pas reconstruit, la compilation de
  l'ex\'ecutable \verb+prog+ ne peut qu'\'echouer.
\item il faut imposer la production des objets  \verb+foo.o+
  et \verb+module.o+ sur le m\^eme mod\`ele que celle de \verb|main.o|.
\end{enumerate}
\fi
