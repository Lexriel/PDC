

Une matrice de dimension~$3$ est basiquement un tableau \`a trois
coordonn\'ees dont chaque cellule est un entier machine.

Comme on d\'esire manipuler des matrices de tailles diff\'erentes, on
se propose de cr\'eer un type structure d'identificateur
\verb+matrix3D_t+ dont les champs sont~:
\begin{itemize}
\item trois identificateurs~\verb+x+, \verb+y+ et~\verb+z+ de type
  entiers machines sp\'ecifiant le nombre de cellules dans chaque coordonn\'ees~;
\item un pointeur sur le tableau de cellules.
\end{itemize}
\paragraph{Questions.}
\begin{enumerate}
\item Donner la d\'efinition du type \verb+matrix3D_t+.
\item Donner la d\'efinition de la fonction \verb+CreerMatrice+ qui
  prend en param\`etre trois entiers donnant la taille de la matrice,
  qui alloue l'espace n\'ecessaire \`a un objet de type
  \verb+matrix3D_t+ et qui retourne un pointeur sur cet objet.
\item Donner la d\'efinition de la fonction \verb+DetruireMatrice+ qui
  prend en param\`etre un pointeur de type \verb+matrix3D_t+ et
  d\'esalloue l'espace m\'emoire associ\'e.
\item Donner la d\'efinitian de la fonction \verb+AddMatrice+ qui
  prend en param\`etre deux pointeurs de type \verb+matrix3D_t+ et qui
  retourne un tel pointeur pointant sur une matrice obtenue par la
  somme des~$2$ pass\'ees en param\`etre.
\end{enumerate}
