\section{Correspondance entre v\oe{}ux et stock}
\label{sec:Correspondance}
Un vendeur de voitures d'occasions d\'esire utiliser un programme en~C
permettant de  faire correspondre  les  voitures  en stock  avec   les
v\oe{}ux de ses clients. La boutique de notre vendeur ne peut contenir
plus de~$500$ voitures.
\par
\`A chacune  de  ses  voitures,  on associe  les informations suivantes~:
\begin{enumerate}
\item le num\'ero d'inventaire cod\'e par un entier~;
\item l'ann\'ee du mod\`ele cod\'e par un entier~;
\item le nom du mod\`ele stock\'e dans une cha\^\i{}ne de caract\`eres
  n'exc\`edent pas~$50$ caract\`eres~;
\item la couleur qui est cod\'ee par un entier associ\'e au type
\begin{verbatim}
typedef enum {rose, fushia, saumon, jaune } couleur ;
\end{verbatim}
\item le coefficient de consommation au kilom\`etre (km/l)~;
\item le style  qui est  cod\'e par  un entier  associ\'e au  type
\begin{verbatim}
typedef enum {roadster, suppositoire_a_camion, quatrequatre } style ;
\end{verbatim}
\item le prix qui est cod\'e par un nombre flottant.
\end{enumerate}
Chaque fois qu'un client se   pr\'esente, le vendeur s'informe de  son
nom et de ses  pr\'ef\'erences  concernant ~$4$ crit\`eres et  cherche
dans la  liste  des voitures si l'une    d'entre elle correspond.   On
suppose que cette correspondance existe si, et seulement si,
\begin{itemize}
\item le code couleur existant et celui d\'esir\'e sont les m\^emes~;
\item le coefficient  de consommation de  la voiture est sup\'erieur ou
  \'egal \`a celui que sp\'ecifie le client~;
\item le code style existant et celui d\'esir\'e sont les m\^emes~;
\item le prix de la voiture ne d\'epasse pas le prix du client de plus
  de~$15$ pour cent.
\end{itemize}

\paragraph{Exemple d'entr\'ees~:}
\mbox{}\par
\begin{tabular}{rllrrrr}
  7210002 & 1985 & Sab 9000 Turbo    &      rose & 27 & roadster& 18000.00 \\
  2215 & 1983 & Plymouth Gran Fury &   rose & 25 & roadster & 3500.00 \\
  200110 & 1976 & Volvo 265GL        &    jaune & 27 & quatrequatre & 1200.00
\end{tabular}
\paragraph{V\oe{}ux des clients~:}
\mbox{}\par
\begin{tabular}{lrrrr}
Alan Turing &                rose& 26 & roadster& 17000.00 \\
Jean-Paul Sartre &          jaune& 25 & quatrequatre &  300.00 \\
Carl Gauss     &           rose & 20&  roadster& 20000.00
\end{tabular}
\paragraph{Correspondance existante~:}
\mbox{}\par
\begin{tabular}{ll}
Alan Turing & 7210001 '85 Saab 9000 Turbo\\
Jean-Paul Sartre & Pas de correspondance \\
Carl Gauss & 2215 '83 Plymouth Gran Fury \\
  &        7210001 '85 Saab 9000 Turbo
\end{tabular}
\paragraph{Questions.}
\begin{enumerate}
\item Donnez la  d\'efinition d'un type  \texttt{struct} qui permet de
  coder l'ensemble   des  informations associ\'ees  \`a  une voiture.  
  Donnez  la  d\'efinition d'un   autre type   \texttt{donnee\_client}
  permettant  de coder  l'ensemble  des informations  fournies par  le
  client.   Pour finir, donnez   la  d\'efinition d'un tableau  global
  \texttt{stock} permettant  de  stocker  l'ensemble  des informations
  concernant le  stock de voitures  (vous  n'avez pas \`a  entrer  les
  informations concernant les exemples de voitures dans ce tableau).
%\item Donnez le code d'une fonction~C qui ne renvoie rien et qui prend
%  en argument l'adresse d'une variable de type \texttt{donnee\_client}.
%  Cette fonction permet au vendeur de saisir les v\oe{}ux du client.
\item On    suppose que le   tableau  global \texttt{stock}  a \'et\'e
  construit. Donnez le code d'une fonction~C qui prend en argument une
  variable du  type \texttt{donnee\_client} et  qui  affiche le nom du
  client ainsi que toutes  les voitures correspondant \`a sa recherche
  (si cette recherche n'aboutit pas, on affichera le message \og Pas de
  correspondance \fg).
\end{enumerate}
