\section{Compilation s\'epar\'ee}
Consid\'erons le fichier d'ent\^ete~\verb+sub.h+~:
\begin{verbatim}
void f(int);
\end{verbatim}
le fichier source~\verb+sub.c+~:
\begin{verbatim}
#include "sub.h"
#include <stdio.h>
static int a;
extern int b;
void f(int c){  
   static int d = 0;
   a = 0;
   b = 0;
   printf("%d %d %d %d\n", a, b, c, d);
   a += 10;
   b += 10;
   c += 10;
   d += 10;
   printf("%d %d %d %d\n", a, b, c, d);
}
\end{verbatim}
et le fichier source~\verb+main.c+~:
\begin{verbatim}
#include "sub.h"
#include <stdio.h>
int a, b;
int main(void){
   int c;
   a = 1;
   b = 2;
   c = 3;
   {
      int c;
      a = 5;
      b = 6;
      c = 7;
      printf("%d %d %d\n", a, b, c);
      f(c);
      printf("%d %d %d\n", a, b, c);
   }
   printf("%d %d %d\n", a, b, c);
   f(c);
   printf("%d %d %d\n", a, b, c);
   return 0;
}
\end{verbatim}
\paragraph{Questions}
\begin{enumerate}
\item Donnez un makefile permettant d'obtenir un ex\'ecutable \verb+exo.exe+ 
\`a partir de ces fichiers (sans utiliser de r\`egles implicites).
\item Donnez l'affichage produit par cet ex\'ecutable.
\end{enumerate}
\ifcorrection
\begin{enumerate}
\item Pour faire simple
\begin{verbatim}
exo.exe: main.o sub.o
        gcc -o exo.exe main.o sub.o
main.o: main.c sub.h
        gcc -c main.c
sub.o: sub.c sub.h
        gcc -c sub.c
\end{verbatim}
mais bien sur on peut avoir beaucoup de variantes avec
\begin{itemize}
\item des variables (predefinies ou non)~;
\item des r\`egles implicites, etc.
\end{itemize}
\item
\begin{verbatim}
 5 6 7
 0 0 7 0
 10 10 17 10
 5 10 7
 5 10 3
 0 0 3 10
 10 10 13 20
 5 10 3
\end{verbatim}
\end{enumerate}
\fi
