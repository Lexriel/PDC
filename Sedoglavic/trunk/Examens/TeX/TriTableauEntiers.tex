\section{Exercices sur les pointeurs}
L'exercice consiste   \`a  lire des entiers  et   les stocker  dans un
tableau, s\'eparer  les nombres  pairs  et les  impairs en les mettant
dans deux tableaux diff\'erents, puis trier chacun des deux tableaux.
\par
Les tableaux ont une taille arbitraire de~$100$. La lecture des entiers
s'effectue dans  la fonction \texttt{main()}  et se termine lors de la
saisie de l'entier~$0$, cet entier est stock\'e dans  le tableau et il
est consid\'er\'e comme une sentinelle de fin de tableau.
\par
La s\'eparation  entre  nombres pairs et  impairs   est faite par  une
fonction qui re\c{c}oit en arguments les  adresses des trois tableaux. 
Cette fonction remplit les cases non occup\'ees  des tableaux pairs et
impairs par des z\'eros.
\par
Le  tri  de  chaque   tableau   est    r\'ealis\'e  par  une    fonction
\texttt{tri(int *)} qui  utilise un algorithme de  tri de votre choix. 
\par
Pour finir,  la fonction \texttt{main()} doit  afficher le contenu des
deux tableaux tri\'es.
