
\section{Exercice sur l'allocation dynamique}
On se propose de donner les structures de donn\'ees et les fonctions
permettant de repr\'esenter une partie de morpion. Ce jeu oppose~$2$
joueurs autour d'une grille de~$9$ cases~:
\begin{center}
  \begin{tabular}{|c|c|c|}
    \hline 
    X & & O   \\  \hline 
    O & X & \\    \hline
    O &  & O   \\  \hline
    \end{tabular}
\end{center}
On tire au hasard le joueur qui commence. Puis successivement, chaque
joueur marque une case de son signe et l'objectif est d'aligner~$3$
signes identiques.

\par\medskip\noindent\textbf{Travail \`a rendre.}  \'Ecrivez un
programme permettant de repr\'esenter une partie de morpion en
utilisant les d\'eclarations de la figure~\ref{fig:declaration}
page~\pageref{fig:declaration} et en vous conformant strictement aux
sp\'ecifications des questions.
\begin{figure}[htbp]
  \centering
\begin{verbatim}
#define NULL 0
typedef enum{X,O,fini} joueur_t ; 

typedef struct position_s *ptrposition_t;

partie_t * partie ;

typedef struct position_s *ptrposition_t;
\end{verbatim}
  \caption{D\'eclarations (exercice de la repr\'esentation du jeu de morpion)}
  \label{fig:declaration}
\end{figure}
Ce programme utilisera une liste cha\^\i{}n\'ee de type
\verb+partie_t+ pour stocker la suite ordonn\'ee de toutes les
positions qui ont \'et\'e jou\'ees. Vous devez donc
\begin{enumerate}
\item Donnez la d\'eclaration de \verb+position_s+ repr\'esentant une
  position. Une position permet d'avoir l'\'etat de la grille \`a un
  seul moment du jeu.
\item Donnez une d\'efinition de la fonction
\begin{verbatim}
ptrposition_t creerNouvellePosition( ptrposition_t b, joueur_t p, 
                                     int ligne, int colonne);
\end{verbatim}
  qui \`a partir d'une position ant\'erieure et d'un coups d'un joueur
  construit la position ainsi obtenue. Si le coup n'est pas valide,
  cette fonction retourne \verb+NULL+.
\item Donnez une d\'eclarations du type \verb+partie_t+ permettant de
  construire une liste cha\^\i{}n\'ee repr\'esentant une partie.
\item la fonction %
  \verb+joueur_t aQuiLeTour(partie_t *encours);+%
  qui d\'etermine \`a partir d'une partie, le prochain joueur \`a
  jouer. Cette fonction retourne \verb+fini+ \`a la fin du jeu et
  lorsqu'elle est appell\'ee avec une position dans laquelle aucun
  joueur n'a jou\'e.
%\item Donnez la d\'efinition de la fonction
%  \verb+insererPositionDansPartie+ qui ajoute une cellule contenant
%  une position qui lui est pass\'ee en param\`etre \`a une liste
%  cha\^\i{}n\'ee elle aussi pass\'ee en param\`etre.
\item Donnez la d\'efinition de la fonction \verb+detruirePartie+
  permettant de lib\'erer l'espace m\'emoire associ\'e \`a une partie.
\end{enumerate}

\par\medskip\noindent\textbf{Attention.}
Ce programme n'est pas sens\'e ni d\'ecider des coups ni les afficher 
mais seulement les enregistrer.
