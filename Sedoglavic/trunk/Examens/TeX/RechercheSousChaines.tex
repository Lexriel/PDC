\section{Recherche de sous-cha\^\i{}nes}
\label{sec:SousChaine}
\'Ecrivez une  fonction   qui prend  en entr\'ee une   cha\^\i{}ne  de
caract\`eres (cod\'ee  par un pointeur)  et qui  retourne le nombre de
lettres diff\'erentes pr\'esentes dans la cha\^i{}ne.
\par
\'Ecrivez une fonction qui prend en entr\'ee~:
\begin{itemize}
\item   un entier  positif~$n$   d\'esignant   une  longueur de   sous-cha\^\i{}ne~;
\item une cha\^\i{}ne de caract\`eres (cod\'ee par un pointeur)~;
\end{itemize}
Cette fonction retourne  le nombre de  sous-cha\^\i{}nes distinctes de
longueur~$n$ pr\'esentent dans la cha\^\i{}ne fournie en entr\'ee.
\par
Par  exemple si~${n=3}$  et la cha\^\i{}ne  est  ``daababac'' alors la
fonction  devra  retourner~$5$. En  effet,  les sous-cha\^\i{}nes sont
``daa'', ``aab'', ``aba'', ``bab'' et ``bac''.
