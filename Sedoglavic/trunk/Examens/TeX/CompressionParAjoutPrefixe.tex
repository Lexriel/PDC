\section{Compression de texte par ajout de pr\'efixe}
On consid\`ere dans la suite des cha\^\i{}nes de caract\`eres
\textsc{ascii} exclusivement compos\'ees de lettres (elles ne
comportent aucun chiffre).
\par
Une m\'ethode  primaire  de  compression  consiste \`a  rep\'erer  les
motifs   r\'ep\'et\'es dans  la cha\^\i{}ne   \`a  compresser puis \`a
construire  une   nouvelle  cha\^\i{}ne  constitu\'ee  du    nombre de
r\'ep\'etition suivit du motif. Par exemple, la cha\^\i{}ne~:
\begin{verbatim}
AAAAAAAAAABABABCCD
\end{verbatim}
est compress\'ee en
\begin{verbatim}
10A2BA1B2C1D
\end{verbatim}
\paragraph{Questions.}
\begin{enumerate}
\item  Construire une fonction qui  prend  en argument un pointeur sur
  une  cha\^\i{}ne de caract\`eres    compress\'ee et qui   renvoit un
  pointeur  sur   une  nouvelle    cha\^\i{}ne  de caract\`eres    non
  compress\'ee.
\item Construire une fonction qui lit  une cha\^\i{}ne de caract\`eres
  au  clavier et  qui  affiche sa    compression  suivant le  principe
  ci-dessus. Cette fonction   devra  retourner  un pointeur   sur   la
  cha\^\i{}ne de caract\`eres comprim\'ee.
\end{enumerate}
