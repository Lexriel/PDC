Donnez la d\'efinition des  fonctions 
\begin{verbatim}
int maxi_tableau(int *tab,int taille) ;
int min_tableau(float *tab,int taille) ;
\end{verbatim}
qui retournent l'indice de la cellule contenant le maximum d'un
tableau d'entiers et le minimum d'un tableau de r\'eels.
Ces fonctions retournent~$-1$ en cas de probl\`eme (sur la taille du tableau notemment).
\ifcorrection
\begin{correction}
\begin{verbatim}
int 
maxi_tableau
(int tab[], int taille)
{
        int res ;
        int i ;
        
        /* si le tableau est vide, on ne sait pas 
        quoi faire */
        if(taille<0)
           return -1 ;

        for(res=i=0;i<taille;i++)
        if(tab[res]<tab[i])
                res = i ;

        return res ;
}
\end{verbatim}
La seconde partie se traite de la m\^eme fa\c{c}ons. Elle n'est l\`a que pour
donner l'occasion \`a des \'etudiants se trompant (en ne conservant que l'extrema par exemple) de se corriger.
\end{correction}
\fi
