Que font exactement les d\'eclarations/d\'efinitions des variables
\verb+string+ et \verb+otherstring+ dans le code suivant~:
\begin{verbatim}
#include<stdio.h>
char *string = "La vie est belle\n" ;
char otherstring[] = "La vie est belle\n";
int
main
(void)
{
  if(string==otherstring)
     return 1 ;
  return 0 ;
}
\end{verbatim}
Que retourne ce programme~?
\ifcorrection
\begin{correction}
  Une \'etiquette est cr\'e\'ee dans le segment de donn\'ees et
  associ\'ee \`a la cha\^\i{}ne de caract\`eres
  \verb+''la vie est belle''+.  L'adresse de cette \'etiquette est
  copi\'ee dans la variable/pointeur \verb+string+. Une copie de la
  cha\^\i{}ne est fa\^\i{}te dans le tableau \verb+otherstring+ qui
  est \`a son tour r\'eserv\'e dans le segment de donn\'ee.
 \par
 Cette fonction retourne
 \begin{verbatim}
 % ./a.out ; echo $?
 0
 % 
 \end{verbatim}
\end{correction}
\fi
