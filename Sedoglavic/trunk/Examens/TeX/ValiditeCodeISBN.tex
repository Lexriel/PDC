\section{Validit\'e d'un code ISBN}
\label{sec:CodeISBN}
La plupart des livres sont  publi\'es avec un code les identifiant~:
il s'agit  du code~\textsc{isbn} pour  International Standard  Book Number. Ce
code est compos\'e d'entiers  compris entre~$0$ et~$9$. On utilise  la
lettre~$X$ pour repr\'esenter l'entier~$10$. De  plus, des tirets sont
introduit dans le code afin d'en faciliter la lecture sans pour autant
avoir d'autre signification.
\par
Seul les~$9$ premiers chiffres   d'un code~\textsc{isbn} sont  utilis\'es  pour
identifier le  livre.  Le~$10$i\`eme  caract\`ere sert \`a contr\^oler
la validit\'e du  code (comme la clef  d'un~\textsc{rib} ou  les deux derniers
chiffres de votre num\'ero de s\'ecurit\'e sociale).
\par
L'algorithme  pour tester la validit\'e du  code\textsc{isbn}  est simple.  On
calcule \`a  partir de ce  dernier deux  sommes~$s_{1}$ et~$s_{2}$. Le
code~\textsc{isbn} est correct si  la  valeur finale de~$s_{2}$ est  divisible
par~$11$.
\par
On   expose l'algorithme  au   travers de   l'exemple  du    code~\textsc{isbn}
0-13-162959-X.  Consid\'erons tout d'abord le calcul de~$s_{1}$.
\par
\begin{tabular}{lcccccccccc}
chiffres du code~\textsc{isbn} &  0 &  1 &  3 &  1 &  6 &  2 &  9 &  5 &  9 &  10(X)\\
$s_{1}$ & 0 &   1 &  4 &  5 & 11 & 13 & 22 & 27 & 36 &  46
\end{tabular}
\par
Le calcul de~$s_{2}$ est fait en sommant les sommes partielles de~$s_{1}$
\par
\begin{tabular}{lcccccccccc}
chiffres du code~\textsc{isbn} &  0 &  1 &  3 &  1 &  6 &  2 &  9 &  5 &  9 &  10(X)\\
$s_{1}$ & 0 &   1 &  4 &  5 & 11 & 13 & 22 & 27 & 36 &  46 \\
$s_{2}$ & 0 &  1 &  5 & 10 & 21 & 34&  56&  83&  119 & 165 
\end{tabular}
\par
Pour  finir,  on constate que~$165$  est  le produit de~$15$ par~$11$. 
Notre code~\textsc{isbn} est donc valide.
\paragraph{Question.}
Construisez une fonction~C d'identificateur \verb+IsISBNValid+ qui
prend en argument un tableau de caract\`eres repr\'esentant le code~\textsc{isbn} (pouvant contenir des tirets)
et retourne~$0$ si, et seulement si, ce code est correct.
%\begin{verbatim}
%#include <stdio.h>
%#include <ctype.h>

%int IsISBNValid {
%   int s, s2, i, fail;
%   char str[100], *p;

%   while (1 == scanf("%s", str)) {
%      p = str;
%      i = s = s2 = fail = 0;
%      while (*p && !fail) {
%         if (isdigit(*p)) {
%            s += *p - '0';
%            s2 += s;
%            i++;
%         } else if (*p == 'X') {
%            s += 10;
%            s2 += s;
%            i++;
%         } else if (*p != '-') {
%            fail = 1;
%         }
%         p++;
%      }
%      if (i != 10 || s2%11) fail = 1;
%      printf("%s is %scorrect.\n", str, fail ? "in" : "");
%      scanf("%*[^\n]");
%   }
%   return 0;
%}
%\end{verbatim}
