\section{Construction d'un carr\'e magique par la m\'ethode du damier cr\'enel\'e}
Un carr\'e magique d'ordre~$n$ est un tableau rempli de~$n^{2}$ nombres tel que les sommes
des nombres  d'une   m\^eme colonne ---  resp.\  ligne  ---  sont  les
m\^emes. Par exemple, le carr\'e suivant est magique d'ordre~$5$~:
\par
\begin{center}
  \begin{tabular}{ccccc}
    11 & 24 & 7  & 20 & 3 \\
    4  & 12 & 25 & 8  & 16 \\
    17 & 5  & 13 & 21 & 9\\
    10 & 18 & 1  & 14 & 22\\
    23 & 6  & 19 & 2  & 15
  \end{tabular}
\end{center}
Pour construire un carr\'e magique d'ordre  impair (${n=5}$ dans notre
exemple), il faut commencer par
num\'eroter s\'equentiellement les  diagonales (1 sur 2) d'un
carr\'e englobant le carr\'e de d\'epart comme suit~:
\par
\begin{center}
  \begin{tabular}{cc|ccccc|cc}
      &  &  &  & 1&  &  &  & \\ 
      &  &  & 6&  & 2&  &  & \\ 
      &  &11&  & 7&  & 3&  &   \\\hline
      &16&  &12&  & 8&  & 4& \\
    21&  &17&  &13&  & 9&  & 5\\
      &22&  &18&  &14&  &10&\\
      &  &23&  &19&  &15&  & \\\hline
      &  &  &24&  &20&  &  & \\ 
      &  &  &  &25&  &  &  & 
  \end{tabular}
\end{center}
La taille de ce carr\'e englobant est fix\'ee par celle de la diagonale qui
admet~$n$ \'el\'ement ($5$ dans notre exemple).
\par\medskip
Il reste \`a compl\'eter le carr\'e en ins\'erant --- judicieusement ---
les nombres rest\'es en dehors du carr\'e englobant dans le carr\'e final 
comme suit~:
\par
\begin{center}
  \begin{tabular}{ccccc}
       & 24 &    & 20 &   \\
    4  &    & 25 &    & 16 \\
       & 5  &    & 21 & \\
    10 &    & 1  &    & 22\\
       & 6  &    & 2  &  
  \end{tabular}
\end{center}
Pour \^etre plus pr\'ecis sur le qualificatif judicieusement,
remarquons que~:
\begin{itemize}
\item 
\end{itemize}

En combinant ces deux \'etapes, on obtient notre carr\'e.

\paragraph{Question.}   
Donnez  le  code d'une  fonction  en~C  qui  prend  comme argument  un
entier~$n$ et qui   affiche un carr\'e  magique  de taille~$n$ si  cet
entier est impair et un message  d'erreur sinon. 

