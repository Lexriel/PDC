\section{Exercice sur les classes d'allocations}
Consid\'erons le fichier source \verb+aux.c+ suivant~:
\begin{verbatim}
#include <stdio.h>
int x;
static int y;
void affectation(void) { x = 2; y = 3; }
void choix(int a) {
    int x = 1;
    static int y;
    if (a == 0) 
        y = 0;
    else y += x;
    printf("choix: %d %d\n", x, y);
}

void proc1(void) { printf("proc1: %d %d\n", x, y); }
\end{verbatim}
et le fichier source \verb+main.c+
\begin{verbatim}
#include <stdio.h>
extern int x;
static int y;

void proc2(void) 
{ printf("proc2: %d %d\n", x, y); }

int main(void) {
    choix(0);
    affectation();
    choix(1);
    proc1();
    proc2();
    return 0 ;
}
\end{verbatim}
\textbf{Questions~:}
\begin{enumerate}
\item Que manque-t'il au fichier source \verb?main.c? en vue de la
  production
d'un code objet~?
\item Donnez un makefile permettant d'obtenir un fichier
ex\'ecutable \`a partir des fichiers sources \verb+main.c+ et \verb+aux.c+.
\item Donnez la sortie produite par l'ex\'ecution du fichier ainsi obtenu.
\end{enumerate}
\ifcorrection
\begin{correction}
  \begin{enumerate}
  \item il manque les d\'eclarations des fonctions d\'efinies dans
    \verb+aux.c+
\begin{verbatim}
extern void affectation(void) ;
extern void choix(int) ;
extern void proc1(void) ;
\end{verbatim}
  \item
\begin{verbatim}
executable: aux.o main.o
       gcc -o executable aux.o main.o
aux.o: aux.c
       gcc -c aux.c
main.o: main.c
       gcc -c main.c
\end{verbatim}
  \item 
\begin{verbatim}
choix: 1 0
choix: 1 1
proc1: 2 3
proc2: 2 0
\end{verbatim}
  \end{enumerate}
\end{correction}
\fi
