\section{Initialisation d'une zone m{\'e}moire}
Donnez la d\'efinition de la fonction \verb+memset+ de prototype~:
\begin{verbatim}
void *memset (void *buffer, int c, size_t n);
\end{verbatim}
qui initialise les~$n$ premiers octets de \verb+buffer+ avec le caract\`ere de code \textsc{ascii}~\verb+c+ et 
qui retourne la valeur de \verb+buffer+.
\par\medskip
Par exemple le code suivant~:
\begin{verbatim}
#include <stdio.h>
#include <string.h>

int 
main
(void) 
{
   char buf[] = "C'est chouette";

   printf("Avant 'memset': %s\n", buf);
   memset(buf, '*', strlen(buf));
   printf("Buf apres 'memset': %s\n", buf);
   return 0;
}
\end{verbatim}
produit l'ex\'ecution~:
\begin{verbatim}
% ./a.out
Avant 'memset': C'est chouette
Buf apres 'memset': **************
\end{verbatim}
\begin{correction}
#include<stdio.h>

/**
* \brief implantation de la fonction memset
* \parameter void *s 
*/

void *
memset
(void *s, int c, int n)
{
	int i ;
	for(i=0; i<n; i++)
	  ((char *)s)[i] = c ;

	return s ;
}

int
main
(void)
{
	char tab[] ="C'est chouette" ;
	printf("%s\n",(char *) memset((void *) tab, '*', 14));
	return 0 ;
}
\end{correction}
