\section{Exercice sur le passage de param\`etres}
Dans   cet  exercice,   vous allez   \'ecrire  trois programmes   en~C
r\'ealisant la m\^eme op\'eration --- une addition de deux entiers par
un appel  de fonction --- qui  diff\`erent par la m\'ethode de passage
de param\`etres.
\begin{enumerate}
\item D\'efinissez   trois variables globales~\texttt{i},~\texttt{j} et
  \texttt{k} de type entier.  \'Ecrire une fonction \texttt{globadd()}
  qui   fait  l'addition de~\texttt{i}   et~\texttt{j}  et   stocke le
  r\'esultat dans~\texttt{k}.   \'Ecrire la   fonction \texttt{int main()}
  qui r\'ealise la saisie des variables~\texttt{i} et~\texttt{j}, fait
  appel \`a la  fonction  d'addition et retourne le  r\'esultat contenu
  dans~\texttt{k}.
\item   M\^eme exercice que  le  pr\'ec\'edent   mais en utilisant  le
  passage de param\`etres  et   le retour   de fonction.   Les   trois
  variables~\texttt{i},~\texttt{j} et~\texttt{k}  de  type entier sont
  d\'eclar\'ees   localement dans   la  fonction~\texttt{main()}.   La
  fonction d'addition \verb+add+ est une   fonction  retournant un entier.    Ele
  accepte deux  param\`etres entiers (\texttt{p1}  et~\texttt{p2})  et
  retourne    la somme de     ces   deux param\`etres.  La    fonction
  \texttt{main()}     permet    la   saisie     des    deux  variables
  locales~\texttt{i}  et~\texttt{j} et  utilise la fonction d'addition
  en r\'ecup\'erant le r\'esultat de  cette fonction dans la  variable
  locale~\texttt{k}.   Elle  retourne le contenu  de~\texttt{k}.
\item  M\^eme  exercice  que le  pr\'ec\'edent   mais en utilisant  le
  passage de param\`etres  et  un pointeur pour modifier  une variable
  dans         la      fonction      appelante.      Les         trois
  variables~\texttt{i},~\texttt{j} et~\texttt{k}   de type entier sont
  declar\'ees   localement dans    la   fonction \texttt{main()}.   La
  fonction d'addition~\texttt{ptadd()}    est   une fonction  qui   ne
  retourne  rien.   Elle  accepte   trois  param\`etres~: deux entiers
  (\texttt{p1} et~\texttt{p2}) et un param\`etre de type pointeur vers
  un entier qui sert  \`a affecter avec  la somme de ces deux premiers
  param\`etres,   la variable   dont    la fonction   appelante  passe
  l'adresse.  La  fonction \texttt{main()}  saisit  les deux variables
  locales~\texttt{i} et~\texttt{j}  et  appelle  la fonction d'addition
  passant l'adresse de la variable  locale~\texttt{k}. Elle retourne le
  contenu de~\texttt{k} apr\`es l'appel de fonction.
\end{enumerate}
Vous pouvez utiliser la fonction \texttt{scanf} de la librairie standard pour r\'ealiser la 
saisie des entiers.
\ifcorrection
\begin{correction}
  \begin{enumerate}
  \item
\begin{verbatim}
int i,j,k ;
void
globaladd
(void)
{
  k=i+j ;
}
\end{verbatim}
 \item 
\begin{verbatim}
#include <stdio.h>

int
add
(int p1,int p2)
{
  return p1+p2 ;
}

int
main
(void)
{
  int i,j,k ;
  scanf("%d",&i) ;
  scanf("%d",&j) ;
  k=add(i,j) ;  
  return k ;
}
\end{verbatim}

 \item 
\begin{verbatim}
#include <stdio.h>

void
ptadd
(int p1,int p2, int *res)
{
  *res = p1+p2 ;
  return  ;
}

int
main
(void)
{
  int i,j,k ;
  scanf("%d",&i) ;
  scanf("%d",&j) ;
  add(i,j, &k) ;  
  return k ;
}
\end{verbatim}
  \end{enumerate}
\end{correction}
\fi