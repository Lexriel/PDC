\section{Carr\'e magique}
Un carr\'e magique d'ordre~$n$ est compos\'e des~$n^{2}$ premiers entiers (distincts) \'ecrits sous la forme d'une matrice carr\'ee et tels que la sommes des nombres soient \'egales sur chaque lignes, colonnes et sur les~$2$ diagonales principales.
Par exemple, on a~:
\[
\begin{array}{cccccccc}
&&  2 & 7 & 6 &\rightarrow & 15 \\
& & 9 & 5 & 1 &\rightarrow & 15 \\
& & 4 & 3 & 8 &\rightarrow & 15 \\
& \swarrow& \downarrow &  \downarrow &  \downarrow & \searrow \\
 15 & & 15 & 15 & 15 &  &15 
\end{array}
\]
Donnez la d\'efinition de la fonction de prototype~:
\begin{verbatim}
int EstMagique(unsigned int ** tab, unsigned int n) ;
\end{verbatim}
qui retourne~$1$ si le carr\'e magique \verb+tab+ d'ordre~\verb+n+ est
magique et qui retourne~$0$ sinon.
\ifcorrection
\begin{correction}
  \verbatiminput{C/CarreMagique.c}
\end{correction}
\fi


