\section{Num\'eration unaire}
\label{sec:NumUn}
Un  nombre  unaire~$n$ compris   entre~$0$  et~$9$ est  cod\'e par  la
succession de~${n+1}$ symboles~$1$ suivis par un~$0$.  Par exemple, on
a~:
$$
(0)_{10}= 10,\ (1)_{10}= 110,\ (2)_{10}=1110,\ (9)_{10} = 11111111110
$$
avec~$(x)_{10}$ la repr\'esentation d\'ecimale du nombre~$x$.
\paragraph{Questions.}
\begin{enumerate}
\item  Construisez   un programme   qui   saisit une  cha\^\i{}ne   de
  caract\`eres  et  qui   affiche   la  traduction unaire   de   cette
  cha\^\i{}ne  si elle repr\'esente   un entier positif et un  message
  d'erreur dans le cas contraire.
\item  Construisez  un programme    qui   saisit une   cha\^\i{}ne  de
  caract\`eres  et   qui affiche  la   traduction  d\'ecimale de cette
  cha\^\i{}ne  si elle repr\'esente une  suite de nombre  unaire et un
  message d'erreur dans le  cas contraire. Par exemple \`a r\'eception
  de~$111011110$, le programme devra afficher~$23$.
\end{enumerate}
