 Donner la valeur des variables ra, rb, rc apr\`es l'ex\'ecution du programme~:
\begin{verbatim}
#include <stdio.h>
int p1 (int a)
{
          a = a * 2;
          return a + 5;
}
int p2 (int *b)
{
          *b = *b * 2;
          return *b + 5;
}
int p3 (int *c)
{
          return p1 (p2 (c));
}
int main(void)
{
  int a = 2, b = 3, c = 4;
  int ra, rb, rc;
  ra = p1 (a);
  rb = p2 (&b);
  rc = p3 (&c);
  printf ("%d, %d, %d\n", ra, rb, rc);
  return 0;
}
\end{verbatim}
\ifcorrection
\paragraph{Correction.}
Les variables sont ra=9, rb=11 et rc=31.
\fi
