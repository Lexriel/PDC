Expliquez la diff\'erence entre les deux d\'eclarations suivantes~:
\begin{verbatim}
  int *a()   ;
  int (*a)() ;
\end{verbatim}
\ifcorrection%
\begin{correction}
  La d\'eclaration~:
\begin{verbatim}
 int * a () ;
\end{verbatim}
  est celle d'une fonction ne prenant pas de param\`etre et retournant
  un pointeur sur un entier.
  \par
  La d\'eclaration~:
\begin{verbatim}
 int (*a) () ;
\end{verbatim}
  est celle d'un pointeur de fonction pointant sur des fonctions ne
  prenant pas de param\`etre et retournant un entier.
\end{correction}
\fi%