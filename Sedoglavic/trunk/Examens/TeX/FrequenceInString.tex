\section{Fr\'equence des lettres dans un texte}
  \'Ecrire  une fonction  \texttt{int *compteLettres()} qui  compte le
  nombre de   \texttt{a},  de  \texttt{b}, \ldots, de    \texttt{z} et
  d'espaces d'un fichier et renvoit  le r\'esultat sous la forme  d'un
  pointeur  (pensez \`a   indiquer les diff\'erentes   allocations \`a
  effectuer).
  \paragraph{Remarques~:}
  \begin{enumerate}
  \item le tableau contient~$27$ entiers associ\'es (dans l'ordre) aux
    lettres \texttt{a}, \ldots, \texttt{z} et espace~;
  \item on compte aussi bien les majuscules que les minuscules~;
  \item  la fonction r\'ecup\`ere  les   caract\`eres dans un  tableau
    \texttt{char texte[MAX]} que l'on suppose d\'efini globalement.
  \end{enumerate}
  Dans  un deuxi\`eme  temps et   en utilisant  un tableau \`a  double
  entr\'ee,  d\'eterminer  les  proportions  de s\'equences    de deux
  lettres dans un fichier.
  \par
  Ainsi,  on  peut savoir    avec quel pourcentage   appara\^\i{}t  la
  s\'equence \texttt{re}, \texttt{aa} ou $_{\sqcup}$\texttt{a} dans un
  texte.
