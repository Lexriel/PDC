\`A chaque compilation, le compilateur gcc d\'efini des macros suivant
l'architecture le supportant. Par exemple, 
\begin{itemize}
\item sur un ordinateur Sparc g\'er\'e par Solaris, la macro
  \verb+sparc+ est d\'efinie~;
\item sur un Pentium g\'er\'e par Linux, la macro \verb+linux+ est
  d\'efinie~;
\item sur un PowerPC g\'er\'e par Mac~OS~X, la macro \verb+darwin+ est
  d\'efinie.
\end{itemize}
Un utilisateur souhaite disposer d'un tableau \verb+architecture+
permettant de repr\'esenter une cha\^ine de caract\`eres indiquant le
type d'architecture (sparc, linux, darwin). Si aucune de ces macros
n'est d\'efinie, la cha\^ine donnant le type de machine est \verb+unknown+.
\par\medskip
Construisez un fichier d'ent\^ete permettant de d\'efinir cette
variable \`a l'aide de  directives au compilateur.
\ifcorrection
\begin{correction}
\begin{verbatim}
char architecture[] = 
#ifdef sparc
      "sparc" 
#else
#ifdef linux
      "linux" 
#else
#ifdef darwin
      "darwin" 
#else 
     "unknown" 
#endif /* darwin */
#endif /* sparc */ 
#endif /* linux */
;
\end{verbatim}
\end{correction}
\fi