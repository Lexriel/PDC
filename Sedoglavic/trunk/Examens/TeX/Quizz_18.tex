Consid\'erons les fonctions~:
\begin{verbatim}
int foo(int a) { | int bar(int *b) { | int foobar(int *c) {     | int barfoo(int d) {   
    a=a*2 ;      |     *b=(*b)*3 ;   |     return foo(bar(c)) ; |    int t ;            
    return a+3 ; |     return *b+7 ; | }                        |    t = foo(d);        
}                | }                 |                          |    return bar(&t) ;   
                 |                   |                          |  }                     
\end{verbatim}
   Donnez la valeur des diff\'erentes variables d\'efinies ci-dessous
   apr\`es ex\'ecution du code suivant~:
\begin{verbatim}
int a,b,c,d,ra,rb,rc,rd ;
a = 10; b = 20; c = 50; d = 70;
ra = foo(a); rb = bar(&b); rc = foobar(&c); rd = barfoo(d);
\end{verbatim}
\ifcorrection
\begin{correction}
   a = 10, b = 60, c = 150, d = 70, ra = 23, rb = 67, rc = 317, rd = 436 
\end{correction}
\fi