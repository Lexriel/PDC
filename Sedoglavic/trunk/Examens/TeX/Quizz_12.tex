 Donnez en justifiant votre r\'eponse l'affichage produit par le code suivant~:
\begin{verbatim}
#include<stdio.h>
  union compteenbanque {
   double solde ;
   int numero ;  } ;
  struct etudiant {
    char **nom;
    union compteenbanque potdevin;
    enum e {male, femelle, indetermine} genre;
    unsigned short int numiden;
    struct etudiant *acopiersur;  };                                                  
  typedef struct etudiant *PDC[100];

  int main(void){
    printf("%d\n", sizeof(struct etudiant));
    printf("%d\n", sizeof(PDC));
    return 0 ;
  }
\end{verbatim}
Vous pouvez utiliser \verb+sizeof(type)+ avec \verb+type+ un type
scalaire au lieu des chiffres renvoy\'es par cette expression.
L'exercice consiste \`a expliciter la constitution en m\'emoire d'un objet construit sur le mod\`ele \verb+struct etudiant+.
\ifcorrection
\par
Sans tenir compte de l'alignement, on a~:
\begin{verbatim}
sizeof(struct etudiant) =
   sizeof(char **) + max(sizeof(double),sizeof(int)) + sizeof(int)+
   sizeof(unsigned short) + sizeof(struct etudiant *)
\end{verbatim}
\fi
