  Dans le jeux de Nim, deux joueurs disposent de~$m$ tas d'allumettes
  comportent respectivement~${\{n_{1},\ldots,n_{m}\}}$ allumettes. A
  tour de r\^ole, chaque joueur peut enlever autant d'allumettes qu'il
  veut de l'un des tas. Le joueur qui retire la derni\`ere allumette
  perd la partie.
  \paragraph{Addition de Nim.}
  \`A chaque \'etat du jeu, on peut associer un entier obtenu par le
  processus suivant~:
  \begin{itemize}
  \item les nombres d\'ecimaux~$m_{i}$ sont convertis en mots
    binaires~;
  \item ces mots binaires sont dispos\'es comme pour une addition
    habituelle~;
  \item les colonnes sont additionner sans tenir compte des retenus.
  \end{itemize}
  Par exemple si~${m=3}$ et~${n_{1}=3,\ n_{2}=12,\ n_{3}=7,}$ on a~:
  \[
  \begin{array}{cccccc}
    & 0 & 0 & 1 & 1 & (=3)\\
    + & 1 & 1 & 0 & 0 & (=12)\\
    + & 0 & 1 & 1 & 1 & (=7)\\
    \hline
    & 1 & 0 & 0 & 0
  \end{array}
  \]
  \paragraph{Strat\'egie gagante au jeu de Nim.}
    Le r\'esultat suivant fournit un crit\`ere permettant de gagner au
    jeu de Nim~:
    \begin{theoreme}[Bouton, Charles Leonard (1869--1922)]
      Le joueur sur le point de jouer a une strat\'egie gagnante si,
      et seulement si, l'addition de Nim de la position courante est
      non nul.      
    \end{theoreme}
    En effet, la position perdante est vide (sans allumette) et donc
    d'addition de Nim nulle. 
    \par
    \`A partir d'une position d'addition de
    Nim non nulle, une strat\'egie gagante est donc de mettre
    syst\'ematiquement son adversaire dans une position d'addition de
    Nim nulle.
    \par
    Ce dernier ne peut en sortir --- si c'est possible ---
    qu'en vous repla\c{c}ant dans une position d'addition de Nim non
    nulle.
  \paragraph{Question.}
  Donnez un programme~C permettant de~:
  \begin{itemize}
  \item saisir sur la ligne de commande le nombre d'allumettes
    constituant chaque tas~;
  \item d'arr\^eter le jeu si la position est perdante~;
  \item de proposer \`a l'utilisateur de choisir un tas et le nombre
    d'allumettes \`a en retirer~;
  \item d'afficher le nombre d'allumette dans chaque tas~;
  \end{itemize}

  \paragraph{Indication~:} en~C, l'op\'erateur~\^{} correspond au xor
  entre~$2$ entiers. Vous pouvez vous servir de les fonctions~C~:
  \begin{center}
    \begin{tabular}{ll}
      fonctions & fichiers d'ent\^ete \\
      malloc & stdlib.h \\
      printf & stdio.h \\
      scanf & stdio.h
    \end{tabular}
  \end{center}
\ifcorrection%
\begin{correction}
  Les \'etats de notre graphe consistent en tous les~$2$ tas possibles
  ayant au maximum~$3$ alumettes et obtenus en retirant~$1$ ou~$2$
  allumettes. Les arcs sont le retrait des allumettes.
  \par
  Pour que le premier joueur gagne, il \^etre sur d'atteindre soit
  l'\'etat~$(0,2)$ soit l'\'etat~$(0,1)$.  On peut v\'erifier qu'en
  jouant~$(1,3)$ au premier coup, quelque soit la r\'eponse de
  l'adversaire, on peut atteindre ces~$2$ \'etats.
\end{correction}
\fi%
