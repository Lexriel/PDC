Consid\'erons le code suivant~:
\begin{verbatim}
#include <string.h>
void foo(void);
int
main
(int argv, char **argc)
{
 int i ;
 for(i=0;i<strlen(argc[0]);i++) foo() ;
 return 0 ;
}
\end{verbatim}
\paragraph{Questions.}
\begin{enumerate}
\item Donner la d\'efinition de la fonction \verb+strlen+ qui prend en
  argument un pointeur sur une cha\^\i{}ne de caract\`eres et retourne
  sa taille.
\item Avec le code de la fonction principale \verb+main+ ci-dessus,
  combien d'acc\`es m\'emoire \`a la cha\^\i{}ne de caract\`eres
  stock\'ee en \verb+argc[0]+ sont effectu\'es.
\item Modifier ce code pour qu'il ne n\'ecessite qu'un nombre
  d'acc\`es lin\'eaire en la taille de la cha\^\i{}ne.
\end{enumerate}
\ifcorrection%
\begin{correction}
  Cet exercice n'a pour objet que de montrer qu'il ne faut pas utiliser un code du genre~:
\begin{verbatim}
 for(i=0;i<strlen(argc[0]);i++) 
\end{verbatim}
  que j'ai malheureusement massivement rencontr\'e dans les copies.
  \begin{enumerate}
  \item 
\begin{verbatim}
int
strlen
(const char *str)
{
  int i ;
  for(i=0; *(str++) ; i++) ;
  return i ;
}
\end{verbatim}

  \item Si~$n$ repr\'esente la longueur de la cha\^\i{}ne, il
    faut~$n^{2}$ acc\`es.
  \item 
\begin{verbatim}
#include <string.h>
int
main
(int argv, char **argc)
{
 int i ;
 for(i=0;arg[0][i];i++) ;
 return 0 ;
}
\end{verbatim}

  \end{enumerate}
\end{correction}
\fi%
