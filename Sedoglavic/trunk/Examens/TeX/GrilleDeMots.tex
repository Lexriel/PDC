\section{Grille de mots}
Un groupe de restauration rapide propose un jeu de lettres \`a ces
clients pour les faires patienter.  Les diff\'erents noms des pizzas
disponibles~:
\begin{verbatim}
MARGARITA ALEMA     BARBECUE TROPICAL SUPREMA 
LOUISIANA CHEESEHAM EUROPA   HAVAIANA CAMPONESA
\end{verbatim}
sont pr\'esents dans le tableau~:
\begin{verbatim}
QWSPILAATIRAGRAMYKEI
AGTRCLQAXLPOIJLFVBUQ
TQTKAZXVMRWALEMAPKCW
LIEACNKAZXKPOTPIZCEO
FGKLSTCBTROPICALBLBC
JEWHJEEWSMLPOEKORORA
LUPQWRNJOAAGJKMUSJAE
KRQEIOLOAOQPRTVILCBZ
QOPUCAJSPPOUTMTSLPSF
LPOUYTRFGMMLKIUISXSW
WAHCPOIYTGAKLMNAHBVA
EIAKHPLBGSMCLOGNGJML
LDTIKENVCSWQAZUAOEAL
HOPLPGEJKMNUTIIORMNC
LOIUFTGSQACAXMOPBEIO
QOASDHOPEPNBUYUYOBXB
IONIAELOJHSWASMOUTRK
HPOIYTJPLNAQWDRIBITG
LPOINUYMRTEMPTMLMNBO
PAFCOPLHAVAIANALBPFS
\end{verbatim}
\'Ecrivez un programme~C qui  \'etant donn\'ee  un fichier contenant  le
tableau ci-dessus affiche pour chaque noms de pizza un triplet~:
\begin{itemize}
\item ordonn\'ee de la premi\`ere lettre~;
\item abscisse de la premi\`ere lettre~;
\item orientation du mot (sachant qu'il y a~$8$ orientation possible~:
  \par
  \begin{center}
    \begin{tabular}{llllllll}
      N & Nord & NE & Nord Est & E & Est & SE & Sud Est \\
      S & Sud & SO & Sud Ouest & O & Ouest & NO & Nord Ouest
    \end{tabular}
  \end{center}
\end{itemize}
Par exemple, l'affichage correspondant au nom~\texttt{CAMPONESA}
est~$(11,11,NO)$ (attention en~C, les indices des tableaux commencent
\`a~$0$).
