Parmi les lignes de codes suivantes~:
\begin{verbatim}
/* 1 */  enum VENT {BOREE, NOTOS, EUROS, ZEPHYR};
/* 2 */  enum Vent {BOREE, NOTOS, EUROS, ZEPHYR} VENT;
/* 3 */  typedef VENT enum {BOREE, NOTOS, EUROS, ZEPHYR};
/* 4 */  typedef enum VENT {BOREE, NOTOS, EUROS, ZEPHYR};
/* 5 */  typedef enum {BOREE, NOTOS, EUROS, ZEPHYR} VENT;
\end{verbatim}
quelle est la bonne d\'eclaration permettant la d\'efinition suivante~:
\begin{verbatim}
VENT d;
\end{verbatim}
\ifcorrection%
\begin{correction}
C'est bien sur~:
\begin{verbatim}
typedef enum {BOREE, NOTOS, EUROS, ZEPHYR} VENT;
\end{verbatim}
\end{correction}
\fi