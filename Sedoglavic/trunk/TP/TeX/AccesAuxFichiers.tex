Un fichier  peut \^etre consid\'er\'e  comme l'abstraction d'une suite
d'octets stock\'es sur le disque.
\section{Ouverture et fermeture d'un flot}
\label{sec:OuvertureEtFermetureFlot}
\subsection{Les fonctions fopen() et fclose()~; Gestion des erreurs}
\label{sec:fopenfcloseperror}
La  fonction \texttt{fopen()} ouvre un  fichier et lui associe le flot
de donn\'ee du type \texttt{FILE} qu'elle retourne. Sa syntaxe est~:
\begin{verbatim}
fopen("nom_de_ficher","mode") ;
\end{verbatim}\index{fopen}
Si une  erreur survient  lors de  l'ex\'ecution  de cette fonction, la
valeur   retourn\'ee est   le   pointeur  \texttt{NULL} dont   on peut
retrouver la d\'efinition \`a  partir du fichier \texttt{stdio.h}.  Le
mode correspond  \`a une  cha\^\i{}ne de caract\`eres  envoy\'ee comme
argument \`a  la fonction \texttt{fopen}  et   qui sp\'ecifie le  mode
d'ouverture du fichier (lecture, \'ecriture, etc.)~; on peut consulter
le manuel de  \texttt{fopen} pour le  d\'etail des  modes (\texttt{man
  fopen}).
\par
Il est  vivement recommand\'e de  tester la  valeur renvoy\'ee par  la
fonction \texttt{fopen} afin de d\'etecter d'\'eventuelles erreurs.
\par
De  plus, on  dispose sous   certains   syst\`eme d'exploitation  d'un
m\'ecanisme   permettant  d'obtenir de l'information   sur les erreurs
potentielles. Dans notre cas, on peut faire comme suit~:
\begin{verbatim}
#include <stdio.h>
#include <errno.h>

int main(void){

   FILE *fd = fopen("fichierquinexistepas","r") ;
   if (fd==NULL){ 
     perror("L'erreur suivante est survenue") ;
     return -1 ;
   }

   fclose(fd) ; /* fclose ferme le flot */
   return 0 ;
}
\end{verbatim}\index{perror}
La fonction \texttt{fclose()} permet  de fermer le flot associ\'e  \`a
un fichier par la  fonction \texttt{fopen()}. Cette fonction  retourne
l'entier~$0$ si, et seulement   si,  la fermeture  s'est  d\'eroul\'ee
normalement.\index{fclose}
\subsection{La structure FILE}
\label{sec:StructureFile}
Pour  information,   nous   explicitons la   structure   \texttt{FILE}
utilis\'ee pour g\'erer les flots~:
\begin{verbatim}
 _IO_FILE {
  int _flags;           /* High-order word is _IO_MAGIC; rest is flags. */
#define _IO_file_flags _flags

  /* The following pointers correspond to the C++ streambuf protocol. */
  /* Note:  Tk uses the _IO_read_ptr and _IO_read_end fields directly. */
  char* _IO_read_ptr;   /* Current read pointer */
  char* _IO_read_end;   /* End of get area. */
  char* _IO_read_base;  /* Start of putback+get area. */
  char* _IO_write_base; /* Start of put area. */
  char* _IO_write_ptr;  /* Current put pointer. */
  char* _IO_write_end;  /* End of put area. */
  char* _IO_buf_base;   /* Start of reserve area. */
  char* _IO_buf_end;    /* End of reserve area. */
  /* The following fields are used to support backing up and undo. */
  char *_IO_save_base; /* Pointer to start of non-current get area. */
  char *_IO_backup_base;  /* Pointer to first valid character of backup area */
  char *_IO_save_end; /* Pointer to end of non-current get area. */

  struct _IO_marker *_markers;

  struct _IO_FILE *_chain;

  int _fileno;
  int _blksize;
  _IO_off_t _old_offset; /* This used to be _offset but it's too small.  */

#define __HAVE_COLUMN /* temporary */
  /* 1+column number of pbase(); 0 is unknown. */
  unsigned short _cur_column;
  signed char _vtable_offset;
  char _shortbuf[1];

  /*  char* _save_gptr;  char* _save_egptr; */

  _IO_lock_t *_lock;
#ifdef _IO_USE_OLD_IO_FILE
};
\end{verbatim}
\section{\'Ecriture et lecture format\'ee d'un flot}
\label{sec:EcritureLectureFlot}
Les fonctions d'\'ecriture et de lecture format\'ee de et dans un flot
sont  similaires  aux   fonctions   \texttt{scanf}  et
\texttt{printf}.
\subsection{\'Ecriture~: fprintf()}
\label{sec:Ecriture}\index{fprintf}
La  fonction \texttt{fprintf} permet d'\'ecrire  des donn\'ees dans un
fichier. Sa syntaxe est~:
\begin{verbatim}
int fprintf(FILE *stream, const char *format, arg1, arg2, ...);
\end{verbatim}
o\`u \texttt{FILE *stream} est   le flot de donn\'ees  retourn\'e  par
\texttt{fopen}.  Les sp\'ecifications  de  format sont identiques  \`a
celle de la fonction \texttt{printf}.
\begin{exercice}[Utilisation de fprintf]
  Construire un  programme qui ouvre  un nouveau fichier en \'ecriture
  et  qui le   remplit     avec les  cha\^\i{}nes   de    caract\`eres
  repr\'esentant les nombres premiers inf\'erieurs \`a~$1000$.
  \ifcorrection
  \begin{correction}
    \input{Verbatim/fichierNP.c.verb}
  \end{correction}
  \fi
  Remarquez que le fichier  de stockage contient bien des cha\^\i{}nes
  de caract\`eres et pas des octets codant des entiers.
\end{exercice}
\subsection{Lecture~: fscanf()}
\label{sec:Lecture}\index{fscanf}
La  fonction \texttt{fscanf}  permet  de  lire des  donn\'ees  dans un
fichier.  Son utilisation et sa syntaxe  sont identiques \`a celles de
\texttt{scanf}~:
\begin{verbatim}
int fscanf(FILE *stream, const char *format, arg1, arg2,...);
\end{verbatim}
o\`u   \texttt{FILE *stream} est le   flot de donn\'ees retourn\'e par
\texttt{fopen}.  Les sp\'ecifications de format  sont  les m\^emes que
celles de la fonction \texttt{scanf}.
\begin{exercice}[Lecture d'un fichier]
  Apr\`es avoir sauvegard\'e des nombres premiers  dans un fichier, on
  se  propose de  relire se  fichier et  d'afficher les  nombres qu'il
  contient. Remarquez que le probl\`eme de fin de fichier se pose lors
  de la lecture. Pour savoir quand le  fichier est termin\'e, utilisez
  la fonction \texttt{feof}.
  \ifcorrection
  \begin{correction}
    \input{Verbatim/lectureFichierNP.c.verb}
  \end{correction}
  \fi
\end{exercice}
\section{Compl\'ements}
\label{sec:Complements}
L'\'etude de cette section peut \^etre omise en premi\`ere lecture.
\subsection{Les flots standards}
\label{sec:FlotsStandards}
Trois  flots standard peuvent \^etre utilis\'es   en~C sans qu'il soit
n\'ecessaire de les ouvrir ou de les fermer~:
\begin{itemize}
\item  \texttt{stdin}~:  l'entr\'ee par d\'efaut  (g\'en\'eralement le
  clavier)~;
\item  \texttt{stdout}~:  la  sortie   par d\'efaut  (g\'en\'eralement
  l'\'ecran)~;
\item \texttt{stderr}~:   la    sortie  des   erreurs   par   d\'efaut
  (g\'en\'eralement l'\'ecran).
\end{itemize}
\begin{exercice}[Notre utilitaire ncat et les redirections]
  Modifiez l'utilitaire \texttt{ncat} afin qu'il  prenne en compte les
  m\'ecanisme d'indirection et de redirection comme par exemple~:
\begin{verbatim}
ncat < fichier_entree > fichier_sortie
\end{verbatim}
\end{exercice}
\subsection{Manipulation des fichiers binaires}
\label{sec:FichiersBinaires}
Ce point ne sera pas abord\'e.