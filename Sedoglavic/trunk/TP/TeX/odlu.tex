%------------------------------------------------------------------------------
\section{Compilation s\'epar\'ee}
\label{sec:CompilationSeparee}

\paragraph{Principes \'el\'ementaires de programmation modulaire}
Parmi les r\`egles commun\'ement admises de codage, on peut citer~:
\begin{itemize}
\item  l'abstraction   des  constantes  litt\'erales   (utilisation du
  \verb+#define+ pour utiliser des constantes)~;
\item la  fragmentation  de code  afin,  par exemple, d'am\'eliorer sa
  maintenance~;
\item la factorisation   du code permettant  d'\'ecrire des  fonctions
  utilisables par plusieurs programmes diff\'erents.
\end{itemize}
Supposons que  l'on  souhaite produire  un  ex\'ecutable \`a partir de
deux fichier. On consid\`ere le premier fichier \texttt{foo.c}

\input{Verbatim/compilSeparee.c.verb}

en conjonction avec le fichier \texttt{bar.c}

\input{Verbatim/plusgrand.c.verb}

et l'on d\'esire obtenir     un ex\'ecutable. Il faut  tout    d'abord
construire deux fichiers objets~:
\begin{verbatim}
gcc -c foo.c
gcc -c bar.c
\end{verbatim}
et ensuite compiler un ex\'ecutable~:
\begin{verbatim}
gcc foo.o bar.o
\end{verbatim}
\paragraph{Variable partag\'ee}
Lorsqu'il   est  in\'evitable d'utiliser  une   variable   commune \`a
plusieurs  fichiers source, on doit  indiquer  au compilateur que deux
variables  portant le m\^eme nom  mais d\'eclar\'ees dans des fichiers
diff\'erents correspondent  au m\^eme objet.  Pour   ce faire, on peut
d\'efinir la variable \texttt{var} dans le fichier \texttt{foo.c}~:
\begin{verbatim}
type var ;
\end{verbatim}
et y faire r\'ef\'erence dans le fichier \texttt{bar.c} comme suit~:
\begin{verbatim}
extern type var ;
\end{verbatim}
%------------------------------------------------------------------------------
\section{Make}
\label{sec:Make}
%------------------------------------------------------------------------------
\section{De quelques outils du \texttt{shell}}
\label{sec:OutilsDuShell}
\subsection{Le manuel du programmeur Linux}
\label{sec:man}
\verb?man cmd? affiche le manuel de la commande d\'esign\'ee par
\texttt{cmd}, i.e.\ les rubriques suivantes (qui ne sont pas
forc\'ement pr\'esentes):
\begin{center}
\begin{tabular}{ll}
NAME & \\ 
SYNOPSIS : & liste des arguments, options \\
           & (ceux qui sont sp\'ecifi\'es entre [] sont facultatifs ; ne pas taper ces []) \\
DESCRIPTION : & explications sur ces arguments, options \\
EXAMPLE & \\
WARNINGS & \\
FILES : & fichiers utilis\'es \\
SEE ALSO : & r\'ef\'erences crois\'ees \\
DIAGNOSTICS : & codes de retour \\
BUGS : & comportements anormaux
\end{tabular}
\end{center}
 Ces manuels sont regroup\'es en diff\'erentes sections (voici
celles de Linux)~:
\begin{enumerate}
\item Commandes utilisateurs (\texttt{tcsh}, \texttt{cp}, \texttt{ls},
  \texttt{mkdir} UNIX, \texttt{printf} UNIX).
\item Appels syst\`emes (\texttt{fork}, \texttt{open}, \texttt{mkdir}, etc).
\item  Biblioth
  \`eques (C, fortran, pascal \ldots : \texttt{atoi}, \texttt{malloc}, \texttt{printf} de~C).
\item Fichiers sp\'eciaux et p\'eriph\'eriques (\texttt{console},
  \texttt{socket}, etc).
\item Formats de fichiers (\texttt{aliases}, \texttt{term}, etc).
\item Jeux 
\item  Autres (\texttt{ascii}, \texttt{man}, etc).
\item Commandes administration (\texttt{ping}, \texttt{halt}, etc).
\end{enumerate}
Chaque manuel correspond \`a un fichier rang\'e dans un r\'epertoire~:
par exemple, le manuel de \texttt{tcsh} est stock\'e --- sous format
comprim\'ee --- dans \texttt{/usr/man/man1/tcsh.1.bz2}

On peut pr\'eciser la section de la commande dont on cherche le manuel
: \verb?man -S 3 printf?.


Pour avoir la liste des commandes se rapportant \`a la cha\^\i{}ne de
carat\`eres \texttt{chaine} : \verb?man -k chaine?.

\subsection{Cr\'eation de fichier d'archive avec \texttt{tar}}
\label{sec:tar}
%------------------------------------------------------------------------------
\section{R\'ecapitulatif des commandes gdb}
\label{sec:gdb}
\subsection*{Lancement et arr\^et de gdb}
\index{Gnu debugger}
\par
\begin{tabular}{lcl}
  \shell{gdb fichier} &:& lancement de l'environnement gdb \\
  \commande{quit} &:& sortie de l'environnement gdb
\end{tabular}
\par
\paragraph{Remarque.}
Le raccourci clavier \texttt{CTRL-C} ne provoque pas la terminaison de
\texttt{gdb} mais interrompt la commande courante.
\subsection*{Commandes g\'en\'erales}
\par
\begin{tabular}{lcl}
  \commande{run} &:& lancement d'un programme dans l'environnement gdb \\
  \commande{kill} &:& arr\^et d\'efinitif d'un programme
\end{tabular}
\par
\subsection*{Manipulation des points d'arr\^et}
\par
\begin{tabular}{lcl}
  \commande{break FCT} &:& placer un point d'arr\^et au d\'ebut de la fonction FCT \\
  \commande{break *ADDR} &:& placer un point d'arr\^et \`a l'adresse ADDR \\
  \commande{break NUML} &:& placer un point d'arr\^et \`a la ligne NUML \\[\bigskipamount]
  \commande{disable NUM} &:& inactive le point d'arr\^et NUM \\
  \commande{enable NUM} &:& r\'eactive le point d'arr\^et NUM \\
  \commande{delete NUM} &:& supprime le point d'arr\^et NUM \\
  \commande{delete} &:& supprime tous les point d'arr\^ets \\
\end{tabular}
\par
\subsection*{Ex\'ecution d'un programme pas \`a pas}
\par
\begin{tabular}{lcl}
  \commande{step} &:& ex\'ecute une instruction \'el\'ementaire \\
  \commande{step NUM} &:& ex\'ecute NUM instructions \'el\'ementaires \\[\bigskipamount]
  \commande{next} &:& ex\'ecute une instruction (y compris les fonctions appel\'ees) \\
  \commande{next NUM} &:& ex\'ecute NUM instructions (y compris les fonctions appel\'ees) 
  \\[\bigskipamount]
  \commande{until LOC} &:& ex\'ecute les instructions jusqu`\`a ce que LOC soit atteint 
  \\[\bigskipamount]
  \commande{continue} &:& reprend l'ex\'ecution \\
  \commande{continue NUM} &:& reprend l'ex\'ecution en ignorant les points d'arr\^et NUM fois
  \\[\bigskipamount]
  \commande{finish} &:& ex\'ecute jusqu`\`a ce que la fin de la fonction en cours
  \\[\bigskipamount]
  \commande{where} &:& affiche la position actuelle
\end{tabular}
\par
\subsection*{Affichage du code et des donn\'ees}
\par
\begin{tabular}{lcl}
  \commande{list} &:& affiche le code source par paquet de~$10$ lignes\\
  \commande{list NUML} &:& affiche le code source \`a partir de NUML\\
  \commande{disas} &:& affiche le code autour de la position courante \\
  \commande{disas ADDR} &:& affiche le code autour l'adresse ADDR \\
  \commande{disas ADDR1 ADDR2} &:& affiche le code entre les adresses ADDR1 et ADDR2
  \\[\bigskipamount]
  \commande{print \$REG} &:& affiche le contenu du registre REG\\
  \commande{print /x \$REG} &:& affiche le contenu du registre REG en hexad\'ecimal\\
  \commande{print /t \$REG} &:& affiche le contenu du registre REG en binaire \\
  \commande{print /c \$REG} &:& affiche le contenu du registre REG sous forme de caract\`ere \\
  \commande{print /a \$REG} &:& affiche le contenu du registre REG sous forme d'adresse\\
  \\[\bigskipamount]
  \commande{printf "DESC",OBJ} &:& affichage \`a la~C
  \\[\bigskipamount]
  \commande{x /NFU ADDR} &:& affichage du contenu de la m\'emoire \`a l'adresse ADDR\\
   && N est le nombre d'unit\'e \`a afficher \\
   && F est le format d'affichage \\
   && U indique le groupement~: \\ 
   && b pour~$1$ octet, \\ && h pour~$2$ octets et\\&& w pour~$4$ octets
\end{tabular}
\par
\subsection*{Commandes d'aide}
\par
\begin{tabular}{lcl}
  \commande{help} &:& affiche l'aide \\
  \commande{info program} &:& affiche des informations sur le programme\\
  \commande{info functions} &:& affiche la liste des fonctions d\'efinies \\
  \commande{info variables} &:& affiche les variables et les symbols pr\'ed\'efinies \\
  \commande{info registers} &:& affiche les informations sur les registres\\
  \commande{info breakpoints} &:& affiche les informations sur les points d'arr\^ets
\end{tabular}
\par







