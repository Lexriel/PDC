\begin{exercice}[Saisie d'entier avec getchar] 
Construire un programme dont la fonction principale permette de saisir
un entier stock\'e dans l'entr\'ee standard (sous forme d'une
cha\^\i{}ne de caract\`eres termin\'ee par un retour chariot) et qui
le retourne.
\ifcorrection
\begin{correction}
\input{Verbatim/conversionASCIIEntier.c.verb}
\end{correction}
\fi 
Stocker votre cha\^\i{}ne de caract\`eres en utilisant une commande
interne du shell.  Puis, en utilisant une variable pr\'e-d\'efinie du
shell, v\'erifiez que votre programme marche correctement (attention,
la variable \$? est cod\'ee non sign\'ee sur un octet ).
\end{exercice}

\begin{exercice}[Affichage d'entier avec putchar] 
Construire un programme dont la fonction principale permette d'afficher
un entier machine stock\'e dans une variable.
\ifcorrection
\begin{correction}
\input{Verbatim/conversionEntierASCII.c.verb}
\end{correction}
\fi 
\end{exercice}

\paragraph{Remarque.}
Vous utiliserez syst\'ematiquement ces codes chaque fois que vous
aurez \`a saisir ou \`a afficher des entiers.

