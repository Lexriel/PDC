\begin{exercice}[Suite r\'ecurrente num\'eriquement instable]
  \label{sec:SuiteRecurrenteInstable}
  Nous allons \'etudier la  convergence de la suite de r\'eels~$u_{n}$
  d\'efinie  par la  relation    de  r\'ecurrence et les    conditions
  initiales suivantes~:
                                %HEVEA \par
  $$
  u_{0} = 1.0, \quad  u_{1} = 0.9, \quad
  u_{n+2} = 2.0 u_{n+1} - 0.99 u_{n}.
  $$
  Construire un programme qui affiche les~$1000$ premiers termes de cette
  suite.
  \par
  Pouvez vous en d\'eduire la limite de la suite~?
  \par
  Si vous d\'esirez  voir   tous   les  termes calcul\'es   par   votre
  programme, vous  pouvez utiliser la  commande  shell \texttt{\$~a.out |
    less}.
  \ifcorrection
  \begin{correction}
    \input{Verbatim/suite.c.verb}
  \end{correction}
  \fi
%HEVEA  Vous trouver plus d'informations ici~\ref{ResolutionExacte}.
%HEVEA \begin{cutflow}{ResolutionExacte}
  \label{ResolutionExacte}
  \paragraph{R\'esolution exacte}
  On consid\`ere la suite de  r\'eels d\'efinie pour tout  entiers~$n$
  par~:
                                %HEVEA \par
  \begin{equation}
    \label{eq:FormeExacte}
    v_{n} = \frac{9^{n}}{10^{n}}.
  \end{equation}
                                %HEVEA \par
  On remarque que~${u_{0}=v_{0}}$ et que~${u_{1}=v_{1}}$.
  \par
  De   plus,   si    on    fait     l'hypoth\`ese  de     r\'ecurrence
  que~${u_{n}=v_{n}}$ et~${u_{n+1}=v_{n+1}}$.  Alors, on en d\'eduit~:
  $$
  u_{n+2}   =  2      u_{n+1}    -  \frac{99}{100}    u_{n}    =
  2\frac{9^{n+1}}{10^{n+1}} - \frac{99}{100}\frac{9^{n}}{10^{n}}.
  $$
  C'est \`a dire que~:
  $$
  u_{n+2}  =  \frac{9^{n}}{10^{n}}(2\frac{9}{10} - \frac{99}{100})
  =\frac{9^{n}}{10^{n}}\cdot\frac{81}{100} = \frac{9^{n+2}}{10^{n+2}}.
  $$
                                %HEVEA \par
  Nous venons de  d\'emontrer par r\'ecurrence  que les suites~$u_{n}$
  et~$v_{n}$ sont \'egales.
  \par
  Dans ce  cas, la limite de  la suite~$u_{n}$  lorsque~$n$ tends vers
  l'infini est~$0$.
  \par
  Nous n'aborderons pas plus en d\'etails ce genre de ph\'enom\`enes.
  \par
  Les r\'esultats num\'eriques  que l'on peut  obtenir sont inexactes. 
  Les  erreurs  lorsqu'elles s'accumulent peuvent   ainsi laisser  \`a
  penser qu'une  suite  tends vers l'infini   alors que sa limite  est
  nulle   (imaginez le r\'esultat d'une   telle erreur  lors du calcul
  d'une trajectoire de fus\'ee ou de calculs financier).
                                %HEVEA  \end{cutflow}
\end{exercice}
