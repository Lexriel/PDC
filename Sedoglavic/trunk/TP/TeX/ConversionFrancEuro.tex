\begin{exercice}[Conversion Francs -- Euro]
  Un euro est \'equivalent \`a~$6,55957$ francs.
  \par
  Construire  un  programme qui  permet la  conversion  d'une somme de
  francs en euros.
  \par
  \textbf{Remarque.}  Le taux   de  conversion n'est pas  destin\'e  a
  \^etre  modifi\'e. Ainsi, on peut  le d\'efinir comme une constante. 
  En~C,  cette possibilit\'e  peut reposer  sur  le pr\'eprocesseur et
  correspond    \`a  l'instruction  \textbf{\#{}define}   \texttt{nom}
  \texttt{valeur}.  On ne pr\'ecise pas de type car le pr\'eprocesseur
  n'effectue que des transformations textuelles sur le fichier source.
  \par
  Modifier   votre programme afin de   demander \`a l'utilisateur s'il
  d\'esire effectuer une autre conversion  et le lui permettre le  cas
  \'ech\'eant.
  \ifcorrection
  \begin{correction}
    \input{Verbatim/fr2euromain.c.verb}
  \end{correction}
  \fi
\end{exercice}
