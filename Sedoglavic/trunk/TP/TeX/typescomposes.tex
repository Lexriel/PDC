%------------------------------------------------------------------------------
\section{\'Enum\'erations}
\label{sec:Enumerations}
%------------------------------------------------------------------------------
\section{Structures}
\label{sec:Structures}
Une variable  de type  structure est  un ensemble  fini de variable de
types diff\'erents~; ces \'el\'ements sont les champs de la structure.
\par\index{struct}
La d\'eclaration  d'un \textit{type structure}  dont  l'identificateur
est \texttt{modele} suit la syntaxe suivante~:
\begin{verbatim}
typedef struct modele {
        type1 champs1 ;
             ...
        typeN champsN ;
} modele ;
\end{verbatim}
Pour d\'eclarer une  variable \texttt{var} du type \texttt{modele}, on
utilise la syntaxe~: \texttt{struct modele var;}
\par
Si le type \texttt{modele} n'a pas \'et\'e d\'eclar\'e au pr\'ealable,
on utilise la syntaxe~:
\begin{verbatim}
 struct modele {
        type1 champs1 ;
             ...
        typeN champsN ;
} var ;
\end{verbatim}
On   acc\`ede  aux diff\'erents   champs  d'une structure  gr\^ace \`a
l'op\'erateur  \texttt{.}  appel\'e membre  de structure.  Le~$i$\`eme
\'el\'ements de   la structure   \texttt{var}  est accessible    par
l'instruction~: \texttt{var.champsi}.
\par
Les r\`egles d'initialisation d'une structure lors de sa d\'eclaration
sont les m\^emes que pour les tableaux~:
\begin{verbatim}
struct modele var = { int1, ... , initN } ;
\end{verbatim}
Contrairement  aux  tableaux,   on    peut   appliquer   l'op\'erateur
d'affectation aux structure~:
\begin{verbatim}
struct modele var2 = var ;
\end{verbatim}
\begin{exercice}[Implantation de fractions]
Les rationnels peuvent \^etre consid\'er\'es comme un couple d'entiers~:
la fraction~$3/4$ est repr\'esent\'e par le couple d'entiers~$(3,4)$. 
\par
Nous allons  implanter l'arithm\'etique  des fractions rationnelles et
pour ce faire, il nous faut d\'efinir un type \texttt{Rationnel}. Nous
pourrions  utiliser un  tableau  de  deux   entiers mais  nous  allons
plut\^ot utiliser une structure.
\par
Pour implanter notre arithm\'etique, il nous faut coder~:
\begin{itemize}
\item une fonction d'affichage d'un rationnel~: la fonction
  \texttt{PrintRationnel} prend en argument une variable de type
  \texttt{Rationnel} et affiche un rationnel~;
\item l'addition de deux rationnels~: la fonction \texttt{addRationnel} prend
  en argument deux variables de types \texttt{Rationnel} et retourne
  une variable de type \texttt{Rationnel} codant la somme des deux
  arguments~;
\item le produit de deux rationnels~: la fonction \texttt{mulRationnel} prend
  en argument deux variables de types \texttt{Rationnel} et retourne
  une variable de type \texttt{Rationnel} codant le produit des deux
  arguments~;
\item le quotient de deux rationnels~: la fonction \texttt{quoRationnel} prend
  en argument deux variables de types \texttt{Rationnel} et retourne
  une variable de type \texttt{Rationnel} codant le quotient du
  premier argument par le second~;
\item la r\'eduction sous forme irr\'eductible d'un rationnel~: la
  fonction \texttt{normalRationnel} prend en argument une variable de type
  \texttt{Rationnel} et retourne une variable de type
  \texttt{Rationnel} codant la forme irr\'eductible de l'argument (la
  forme irr\'eductible de la fraction~$2/4$ est~$1/2$).
\end{itemize}
\ifcorrection
\begin{correction}
\input{Verbatim/rationnel.c.verb}
\end{correction}
\fi
\end{exercice}
%------------------------------------------------------------------------------
\begin{exercice}[Implantation des nombres complexes]
  Les nombres complexes peuvent  \^etre consid\'er\'es comme un couple
  de r\'eels~:  le  nombre complexe~$3+4I$  est  repr\'esent\'e par le
  couple~$(3,4)$.
\par
Nous allons implanter l'arithm\'etique des complexes et pour ce faire,
il nous faut   d\'efinir un  type \texttt{Complexe}.  Nous   pourrions
utiliser un tableau de deux entiers mais nous allons plut\^ot utiliser
une structure.
\par
Pour implanter notre arithm\'etique, il nous faut coder~:
\begin{itemize}
\item  une   fonction  d'affichage    d'un  complexe~:  la    fonction
  \texttt{PrintComplexe} prend    en  argument une   variable de  type
  \texttt{Complexe} et affiche un complexe~;
\item l'addition de deux complexes~: la fonction \texttt{addComplexe} prend en
  argument deux variables de  types \texttt{Complexe} et  retourne une
  variable   de  type   \texttt{Complexe} codant la    somme  des deux
  arguments~;
\item le produit de deux complexes~: la fonction \texttt{mulComplexe} prend en
  argument  deux variables de  types \texttt{Complexe} et retourne une
  variable de   type  \texttt{Complexe}  codant le  produit   des deux
  arguments~;
\item le quotient de deux complexe~: la fonction \texttt{quoComplexe} prend en
  argument deux variables de types  \texttt{Complexe} et retourne  une
  variable de  type  \texttt{Complexe} codant le   quotient du premier
  argument par le second.
\end{itemize}
\ifcorrection
\begin{correction}
\input{Verbatim/complexe.c.verb}
\end{correction}
\fi
\end{exercice}
%------------------------------------------------------------------------------
\section{Union}
\label{sec:Union}
Une union est comparable \`a  une structure si  ce n'est que plusieurs
variables sont stock\'ees au  m\^eme endroit en m\'emoire~: cette zone
peut    donc    \^etre    interpr\'et\'ee  de    plusieurs  fa\c{c}ons
diff\'erentes.
\begin{exercice}[Une arithm\'etique \emph{mixte}]
  En  utilisant votre  implantation  des rationnels  et des complexes,
  d\'efinissez  un   type \texttt{Nombre} qui   pourra  \^etre soit un
  entier, soit un Rationnel, soit un flottant, soit un Complexe.
\paragraph{Indications~:} vous pouvez vous baser sur le type Nombre suivant~:
\begin{verbatim}
  enum TypeNombre {entier, flottant, rationnel, complexe} ;
  typedef struct Nombre {
     enum TypeNombre tn ;
     union valeur {
       int nbentier ;
       Rationnel nbrationnel  ;
       float nbflottant ;
       Complexe nbcomplexe ;
     } valeur;
  } Nombre ;

  typedef struct CoupleNombre {
       Nombre a ;
       Nombre b ;
  } CoupleNombre ;
\end{verbatim}
  Comme pr\'ec\'edent pour implanter notre arithm\'etique, il nous faut coder~:
\begin{itemize}
\item  une   fonction  d'affichage    d'un nombre~:  la    fonction
  \texttt{PrintNombre} prend    en  argument une   variable de  type
  \texttt{Nombre} et affiche ce nombre en utilisant les fonctions d\'ej\`a cod\'ees~;
\item une fonction de  conversion~: la fonction \texttt{ConvertNombre}
  prend en   argument  une variable de type   \texttt{CoupleNombre} et
  retourne une   variable  de  type \texttt{CoupleNombre}    dont  les
  composantes sont de m\^eme type~;
\item l'addition de deux complexes~: la fonction \texttt{addNombre} prend en
  argument deux variables de  type \texttt{Nombre} et  retourne une
  variable   de  type   \texttt{Nombre} codant la    somme  des deux
  arguments~;
\item le  produit  de deux complexes~:  la fonction \texttt{mulNombre}
  prend en argument  deux variables  de type \texttt{Nombre}   et
  retourne une variable de type  \texttt{Nombre} codant le produit des
  deux arguments~;
\item le  quotient  de deux complexe~:  la fonction \texttt{quoNombre}
  prend en argument deux variables de type \texttt{Nombre} et retourne
  une variable de type \texttt{Nombre}  codant le quotient du  premier
  argument par le second.
\end{itemize}
Pour fixer les id\'ees, on cherche \`a \'elaborer des fonctions permettant les manipulations d\'ecrites dans la fonction principale suivante~:
\begin{verbatim}
int main(void){

  Nombre var1, var2, var3 ;

  var1.tn = rationnel ;
  var1.valeur.nbrationnel.numerateur = 1 ;
  var1.valeur.nbrationnel.denominateur = 4 ;

  var2.tn = entier ;
  var2.valeur.nbentier= 4 ;

  var3.tn = complexe ;
  var3.valeur.nbcomplexe.partie_reelle = 2.54 ;
  var3.valeur.nbcomplexe.partie_imaginaire = 4.45 ;

  PrintNombre(var1) ;
  printf(" et ");
  PrintNombre(var2) ;
  printf(" et ");
  PrintNombre(var3) ;
  printf(".\n");


  PrintNombre(QuoNombre(MulNombre(var1,var2),var3)) ;
  printf(".\n");

  return 1 ;
}
\end{verbatim}
\ifcorrection
\begin{correction}
\input{Verbatim/nombre.c.verb}
\end{correction}
\fi 
\paragraph{Remarque~:} ce genre de  probl\`eme est plus facilement
g\'erable     dans un    langage   orient\'e  objet.    
\end{exercice}
%------------------------------------------------------------------------------

