Dans le jeux de Fan Tan, deux joueurs disposent de~$2$ d'allumettes. A
tour de r\^ole, chaque joueur peut enlever (selon la r\`egle choisie)
un certain nombre d'allumettes de l'un des tas. Le joueur qui retire
la derni\`ere allumette perd la partie.
\paragraph{Question.}
Construisez une interface permettant de faire jouer~$2$ joueurs.
Ainsi, votre programme doit~:
\begin{itemize}
\item Demander le cardinal du premier tas d'allumettes~;
\item Demander le cardinal du second tas d'allumettes~;
\item D\'eclarer une variabe~$i$ et la positionner \`a~$0$~;
\item Tant qu'aucun des tas n'est vides~:
  \begin{itemize}
  \item Demander au~${i+1}$\`eme joueur combien d'allumettes il veut
    retirer et de quel tas. R\'ep\'eter cette \'etape tant que la
    r\'eponse du joueur n'est pas correcte~;
  \item Retirer les allumettes du tas choisi~;
  \item positionner~$i$ \`a~${i+1}$ modulo~$2$
  \end{itemize}
\item Annoncer la victoire du joueur~${i+1}$.
\end{itemize}
