
\documentclass{article}
\usepackage[french]{babel}
\usepackage{hevea}
\usepackage{graphicx}
\usepackage{makeidx}

\begin{document}
%------------------------------------------------------------------------------
\section{Listes cha\^\i{}n\'ees}
\label{sec:ListesChaines}\index{Listes cha\^\i{}n\'ees}
Une  liste  cha\^\i{}n\'ee est   constitu\'ee de cellules d'un type
\texttt{struct} contenant~:
\begin{itemize}
\item un pointeur/l'adresse sur/de la cellule suivante~;
\item des champs d'informations.
\end{itemize}
On peut ainsi d\'efinir une cellule par le type suivant~:
\begin{verbatim}
typedef struct cellule{
  struct cellule *suivante ;
  type_1 el_1 ;
    .... 
  type_n el_n ;
} cellule ;
\end{verbatim}
Ceci fait,   il     nous  reste  \`a   d\'efinir    les  manipulations
\'el\'ementaires associ\'ee \`a une liste~; il nous faut des fonctions 
qui permettent~:
\begin{itemize}
\item de cr\'eer d'une cellule~;
\item d'ins\'erer une cellule dans la liste~;
\item de supprimer une cellule de la liste~;
\item de d\'etruire d'une cellule~;
\end{itemize}
Pour cr\'eer une cellule, il faut allouer de l'espace m\'emoire~:
\begin{verbatim}
cellule * CreerCellule(type_1 data_1, ... type_n data_n){

  cellule * cell = (cellule *) malloc(sizeof(cellule)) ;

  cell->suivante = NULL   ; /* la cellule nouvellement cr\'ee
                             * ne pointe sur rien             */
  cell->el_1     = data_1 ;
     ....
  cell->el_n     = data_n ;

  return cell ;
}
\end{verbatim}
Cette fonction cr\'ee (et remplit) une cellule. Remarquez que l'espace
m\'emoire allou\'e gr\^ace  \`a la fonction  \texttt{malloc} n'est pas
d\'etruit    \`a   la   fin de    la   fonction  \texttt{CreerCellule}
contrairement  \`a l'ensemble  des   variables et  arguments de  cette
fonctions.  Pour  \'eviter d'encombrer  la   m\'emoire, il nous   faut
disposer d'une fonction permettant de  d\'etruire une cellule. Pour ce
faire, on   utilise  la   fonction  \texttt{free} de   la    librairie
standrard~:
\begin{verbatim}
void DetruireCellule( cellule * cell) {
  free( (void *) cell) ;
}
\end{verbatim}
Pour ins\'erer une cellule  dans  une liste cha\^\i{}n\'ee  apr\`es la
cellule \verb+cell_courante+, il suffit d'utiliser le code~:
\begin{verbatim}
void InsererCellule( cellule * cell_a_inserer, 
                     cellule * cell_courante){
  cell_a_inserer -> suivante = cell_courante -> suivante ;
  cell_courante -> suivante = cell_a_inserer ;
}
\end{verbatim}
De m\^eme,    pour  extraire une  cellule  apr\`es  une  cellule  (ici
qualifi\'ee de m\`ere), on peut utiliser le code~:
\begin{verbatim}
cellule * ExtraireCellule (cellule * cell_mere){
  cellule * tmp ;
  tmp = cell_mere->suivante ;
  cell_mere -> suivante = tmp -> suivante ;
  return tmp ;
}
\end{verbatim}
\paragraph{Manipulations \'el\'ementaires}
  Construire un  crible   d'\'Eratosth\`ene  \`a l'aide   d'une  liste
  cha\^\i{}n\'ee  qui  au d\'epart   stockera  l'ensemble  des nombres
  inf\'erieurs \`a un entier saisi au clavier.
  \par
  Apr\`es une \'etape   de     construction, il   faudra     consid\'erer
  successivement  les \'el\'ements  de  la   liste  et supprimer   les
  multiples.
  \par
  Pour   finir, l'ensemble  des  nombres   premiers  est afficher  \`a
  l'\'ecran.
  \subparagraph{Imp\'eratif~:}  
  on  prendra  soin     de  d\'etruire les   cellules     de  la liste
  cha\^\i{}n\'ee associ\'ees \`a des nombres compos\'es.

\end{document}

