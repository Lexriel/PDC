\section{Calcul du score d'un match de tennis}
\label{sec:ScoreTennis}
On  cherche dans  ce probl\`eme \`a  calculer le  score d'un  match de
tennis  entre  deux   joueurs~$A$  et~$B$ \`a   partir   d'un  fichier
d'entr\'ees.
\par
Un    \textit{match}  de tennis est  une    suite de \textit{sets}, un
\textit{set} est un suite de \textit{jeux} et un jeu  est une suite de
\texttt{points}. Le nombre de points  (resp.\ de jeu, de set) gagn\'es
par chaque joueur   est accumul\'e depuis~$0$ dans  un  jeu (resp.\ un
set, un match).

Un point  est gagn\'e par un  seul des joueurs  suivant un ensemble de
r\`egles hors du propos de  cet exercice. Un   jeu est gagn\'e par  le
premier  joueur qui  accumule  au  moins~$4$  points et m\`ene  par au
moins~$2$ points d'\'ecarts. Un set est  gagn\'e par le premier joueur
qui  accumule au moins~$6$ jeux  et m\`ene par   au moins~$2$ jeux. Un
match est gagn\'e par le premier joueur qui accumule au moins~$2$ sets.
\par
Le score est donn\'e par le nombre de jeux que chaque joueur a gagn\'e
dans chaque set.
\paragraph{Format des entr\'ees~:}
les   entr\'ees sont contenues    dans un  fichiers.  Ce fichier   est
constitu\'e par une suite de caract\`eres  qui sont soit~$0$ soit~$1$. 
Ces caract\`eres  d\'ecrivent      le vainqueur de     chaque point.   
Le~$n$i\`eme   point est gagn\'e   par  le joueur~$A$  si le~$n$i\`eme
caract\`eres  du  fichier  est un~$0$.  Si ce   caract\`ere est un~$1$
alors, le joueur~$B$ a gagn\'e le~$n$i\`eme point.
\paragraph{Format des sorties~:}
le  r\'esultat     doit \^etre  sauvegard\'e  dans   un   fichier dont
l'utilisateur indiquera  le  nom.  Si  le  fichier d'entr\'ee comporte
suffisamment de  points pour d\'eterminer le   vainqueur, le fichier de
sortie doit indiquer le score final et  qui est le vainqueur.  Dans le
cas contraire, le fichier de sortie  doit indiquer le score du dernier
set en cours et indiquer que le match est en cours.

\paragraph{Exemple de fichier d'entr\'ees~:}
\begin{verbatim}
000111010001000100100010100010001101000000010000011101111101111111001001010
111110101101100010111111010110000010000000110000000100101101011011001000101111111
\end{verbatim}

\paragraph{Exemples de fichier de sortie~:}
\begin{verbatim}
Score du 1 set: A-6 B-0
Score du 2 set: A-3 B-6
Score du 3 set: A-6 B-2
Match termine, A est vainqueur.

Score partiel. Dernier set: A-1 B-3
Match incomplet.
\end{verbatim}
\par\bigskip
\'Ecrivez un programme en~C  qui r\'eponde \`a toutes les  contraintes
ci-dessus.
%------------------------------------------------------------------------------

\section{Planning de diners}
\label{sec:PlanningDeDiners}
Chaque soir   de      la semaine, le     professeur   Granteaut quitte
l'universit\'e pour  aller d\'ejeuner en  ville. Il est confront\'e au
probl\`eme  de choisir un restaurant.  Pour  un jour donn\'e, le choix
d\'epend  de   son  app\'etence  pour    la  cuisine  des diff\'erents
restaurant, de leurs distance de l'universit\'e  et du nombre de jours
\'ecoul\'e depuis  sa derni\`ere  visite  \`a ce  restaurant. Dans  ce
probl\`eme, on cherche \`a calculer une suite des meilleurs restaurant
suivant ces crit\`eres sur une p\'eriode de vingt jours.

\paragraph{Format des entr\'ees~:}
les entr\'ees forment un fichier constitu\'e d'un nombre pair de lignes. Chaques paires constituent des informations concernant un restaurant.
\par
La premi\`ere ligne de chaque paire contient le nom du restaurant et on suppose que ce nom ne contient pas plus de~$100$ caract\`eres.
\par
La seconde ligne de    chaque paire contient deux  entiers~$a$  et~$b$
(repr\'esent\'es par  deux  cha\^\i{}nes de  caract\`eres s\'epar\'ees
par  un espace).    L'entier~$a$ est compris   entre~$0$   et~$100$ et
repr\'esente l'app\'etence du professeur Granteaut pour ce restaurant.
Le second entier~$b$ est positif et repr\'esente  la distance entre le
restaurant et l'universit\'e.
\paragraph{Format des sorties~:}
 pour un jour donn\'e, l'attrait d'un restaurant est calcul\'e par la formule~:
$$
\max(0,a-b^{2}+\min(t,10)^{2}),
$$
o\`u~$a$ et~$b$ ont \'et\'e d\'efinis dans la description du format
des  entr\'ees  et  o\`u~$t$ repr\'esente   le  nombre de jours depuis
lequel le professeur n'a pas mang\'e dans ce restaurant (on suppose que~$t$ est l'infini si le restaurant n'a jamais \'et\'e visit\'e).
\par
Une liste des jours et des restaurants les plus int\'eressant pour ces
jours sur  une p\'eriode  de~$20$ jours  doit \^etre  stock\'ee  dans un
fichier de sortie. Si deux restaurants ont le m\^eme attrait, on doit choisir celui qui a \'et\'e visit\'e le moins r\'ecemment. Si aucun n'a \'et\'e visit\'e, on choisit par ordre alphab\'etique.
\paragraph{Exemple de fichier d'entr\'ees~:}
\begin{verbatim}
Chez Wendy
90 3
Chinois
95 3
Les delices de Deli
90 2
Wienerschnitzel
70 6
Del Taco
50 5
KFC
55 5
Carl's Jr.
20 1
La tomate rouge
75 2
Le restaurant du metro
30 2
\end{verbatim}
\paragraph{Exemples de fichier de sortie~:}
\begin{verbatim}
Jour  Restaurant

  1   Chinois
  2   Les delices de Deli
  3   Chez Wendy
  4   La tomate rouge
  5   Wienerschnitzel
  6   KFC
  7   Le restaurant du metro
  8   Chinois
  9   Les delices de Deli
 10   Chez Wendy
 11   Del Taco
 12   La tomate rouge
 13   Carl's Jr.
 14   Chinois
 15   Wienerschnitzel
 16   Les delices de Deli
 17   KFC
 18   Chez Wendy
 19   Le restaurant du metro
 20   La tomate rouge
\end{verbatim}
\'Ecrivez un programme en~C  qui r\'eponde \`a toutes les  contraintes
ci-dessus.
%------------------------------------------------------------------------------
\section{Validit\'e d'un code ISBN}
\label{sec:ISBN}
  La plupart des livres sont publi\'es avec  un code les identifiant~:
  il s'agit  du code ISBN  pour International Standard Book Number. Ce
  code est compos\'e d'entiers compris entre~$0$ et~$9$. On utilise la
  lettre~$X$  pour  repr\'esenter l'entier~$10$.   De plus, des tirets
  sont introduits dans le code afin d'en faciliter la lecture sans pour
  autant avoir d'autre signification.
  \par
  Seul les~$9$ premiers  chiffres d'un code  ISBN sont utilis\'es pour
  identifier le livre.  Le~$10$i\`eme caract\`ere sert \`a contr\^oler
  la validit\'e du code (comme  la clef d'un RIB  ou les deux derniers
  chiffres de votre num\'ero de s\'ecurit\'e sociale).
  \par
  L'algorithme pour tester la validit\'e du code ISBN  est simple.  On
  calcule \`a partir de ce  dernier deux sommes~$s_{1}$ et~$s_{2}$. Le
  code ISBN  est correct si la valeur  finale de~$s_{2}$ est divisible
  par~$11$.
  \par
  On  expose l'algorithme    au  travers de   l'exemple  du code  ISBN
  0-13-162959-X.  Consid\'erons tout d'abord le calcul de~$s_{1}$.
  \par
  \begin{tabular}{lcccccccccc}
    chiffres du code ISBN& 0 & 1 & 3 & 1 &  6 &  2 &  9 &  5 &  9 &  10(X)\\
    $s_{1}$ & 0 &   1 &  4 &  5 & 11 & 13 & 22 & 27 & 36 &  46
  \end{tabular}
  \par
  Le calcul de~$s_{2}$ est fait en sommant les sommes partielles de~$s_{1}$
  \par
  \begin{tabular}{lcccccccccc}
    chiffres du code ISBN & 0 & 1 & 3 & 1 & 6 &  2 &  9 &  5 &  9 &  10(X)\\
    $s_{1}$ & 0 &   1 &  4 &  5 & 11 & 13 & 22 & 27 & 36 &  46 \\
    $s_{2}$ & 0 &  1 &  5 & 10 & 21 & 34&  56&  83&  119 & 165 
  \end{tabular}
  \par
  Pour finir,  on constate que~$165$ est le  produit de~$15$ par~$11$. 
  Notre code ISBN est donc valide.
  \begin{exercice}[Question]
  Construisez  un programme~C qui permet  de saisir au clavier un code
  ISBN et qui teste s'il est correct.
  \ifcorrection
  \begin{correction}
        \input{Verbatim/isbn.c.verb}
  \end{correction}
  \fi
\end{exercice}
%------------------------------------------------------------------------------
\section{Utilisation du clavier d'un t\'el\'ephone en mode SMS}
\label{sec:SMS}
% (Sado Maso Servile)
  Vous  avez d\'ej\`a  remarqu\'e  que  le clavier d'un  t\'el\'ephone
  outre la touche~\#, ressemblait \`a~:
  \par
  \begin{tabular}{ccc}
    1 &  2  & 3 \\
    & ABC & DEF \\ \\
    4 & 5 & 6 \\
    GHI & JKL & MNO  \\ \\
    7 & 8 & 9 \\
    PQRS &TUV  &WXYZ
  \end{tabular}
  \par
  Cette  disposition permet \`a l'utilisateur de  taper du texte.  Par
  exemple, presser une fois la touche~$7$ correspond \`a la lettre~$P$
  alors  que  presser cette    touche~$4$    fois correspond   \`a  la
  lettre~$S$.  La touche~\# du t\'el\'ephone   sert \`a s\'eparer  les
  lettres  et peut \^etre omise lorsqu'il   est clair qu'une lettre se
  termine et l'autre commence.
  \par
  Par exemple, la s\'equence
\begin{verbatim}
777666222559996#6668866#8244466
\end{verbatim}
  correspond donc au mot
\begin{verbatim}
rockymountain
\end{verbatim}
\begin{exercice}[Question]
  Donnez un programme qui permet de traduire  une suite tap\'ee sur le
  clavier d'un t\'el\'ephone en un texte.
  \ifcorrection
  \begin{correction}
    \input{Verbatim/SMS.c.verb}
  \end{correction}
  \fi
\end{exercice}
%------------------------------------------------------------------------------
%\section{Une implantation sommaire de la fonction printf}
\label{sec:printf}
Dans cette section, on se propose d'implanter la fonction printf dans
une architecture dans laquelle le passage des param\`etres se fait par
une pile (voir les indications en fin d'exercice).

Par ailleurs, on suppose ne disposer que d'une seule fonction externe
d'affichage dont le prototype est \verb?int putchar(int);? et qui
\'ecrit dans la sortie standard un caract\`ere dont le code
\textsc{ascii} est pass\'e en param\`etre. Par exemple, pour afficher
le caract\`ere~\~\, on peut utiliser le code suivant~:
\begin{verbatim}
#include<stdio.h>

int main(void){
  char c = '~' ;
  putchar( c ) ;
  return 0 ;
}
\end{verbatim}
L'objectif est d'implanter la fonction de prototype~:
\verb?void mprintf(const char *format, ...)?
permettant d'afficher sur la sortie standard~:
\begin{itemize}
\item des caract\`eres \textsc{ascii}~;
\item des cha\^\i{}nes de caract\`eres~;
\item des entiers machines dans les bases d\'ecimale et binaire~;
\item des entiers machines de Gauss nomm\'es i.e.\ des nombres
  complexes dont les parties r\'eelles et imaginaires sont des entiers
  machines et auxquels on associe une cha\^\i{}ne de caract\`eres.
\end{itemize}
\par
Cette fonction a un param\`etre obligatoire et un nombre variable de
param\`etres. 
\par
Le param\`etre obligatoire est constitu\'e par une cha\^\i{}ne de
caract\`eres qui est compos\'ee de caract\`eres ordinaires et
de~${0, 1}$ ou plusieurs directives.  Un caract\`ere ordinaire est un
caract\`ere \textsc{ascii} \`a l'exception du caract\`ere~\%.  Une directive
commence par le caract\`ere~\% et peut \^etre de plusieurs formes~:
\begin{itemize}
\item  la directive~\%c indique que l'on souhaite afficher un caract\`ere~;
\item la directive~\%s indique que l'on souhaite afficher une
  cha\^\i{}ne de caract\`eres~;
\item la directive~\%d indique que l'on souhaite afficher un entier en
  base d\'ecimale~;
\item la directive~\%b indique que l'on souhaite afficher un entier en
  base binaire (cette base est indiqu\'ee par la lettre
  minuscule~b~: l'entier d\'ecimal~$2$ est affich\'e comme~$10b$)~;
\item la directive~\%Gd indique que l'on souhaite afficher un entier de
  Gauss en base d\'ecimale (par exemple~$2 + 2 I$)~;
\item la directive~\%Gb indique que l'on souhaite afficher un entier
  de Gauss en base binaire (par exemple~$ 10b + 10b I$)~;
\item toute autre lettre suivant un~\% est affich\'ee (ce qui permet
  d'afficher le caract\`ere \textsc{ascii} \%).
\end{itemize}

\paragraph{Questions}
\begin{enumerate}
\item Donnez la d\'efinition d'une fonction %
  \verb?void PrintInt(int i,int b)? qui \'ecrit sur la sortie
  standard l'entier~$i$ dans la base~${b\in\{2,10\}}$.
\item 
  Donnez la d\'efinition d'une fonction %
  \verb?void PrintString(char *s)? qui \'ecrit sur la sortie standard
  la cha\^\i{}ne de caract\`eres associ\'ee \`a~s.
%\item Donnez la d\'efinition de la fonction de prototype %
%\verb?int *parpt(int,int);? qui prend en entr\'ee la position du param\`etre
%dans la liste des param\`etres\footnote{dans la fonction %
%  \verb?void PrintInt(int i,int b)? i est le premier param\`etre,~b est le second.}
%et qui retourne un pointeur sur ce param\`etre.
\item Donnez la d\'efinition de la fonction \verb?mprintf?.
\end{enumerate}

\paragraph{Indications.}
Rappelons que les param\`etres d'une fonction sont pass\'es par la
pile. Cette pile est compos\'ee de cellules dont la taille en octet
correspond au type \verb?int?, elle croit vers les adresses
d\'ecroissantes et elle a la structure suivante~:
\begin{center}
  \begin{tabular}{l|c|}
    \hline 
    0000 & $\cdots$ \\ \hline
    & seconde variable locale \\ \hline
    & premi\`ere variable locale \\ \hline
    & ancien pointeur de contexte \\ \hline
    & adresse de retour \\ \hline
    & premier param\`etre \\ \hline
    & second param\`etre \\ \hline
    FFFF &$\cdots$ \\ \hline
  \end{tabular}
\end{center}
Ainsi si \verb?foo? est la premi\`ere variable locale automatique
d\'efinie dans la fonction appel\'ee et que \verb?ptr?  est un
pointeur sur cette variable, \verb?ptr+1? pointe sur l'ancien pointeur
de contexte.
\par
De plus, lors de l'appel d'une fonction, ses param\`etres sont
empil\'es du dernier au premier et on note qu'un param\`etre de type
\begin{itemize}
\item entier occupe une cellule~;
\item pointeur occupe une cellule~;
\item caract\`ere occupe une cellule (sizeof(char) octet pour le caract\`ere et
sizeof(int)-sizeof(char) octets inutilis\'es~;
\item entier de Gauss occupe~$3$ cellules, les champs sont empil\'es
  du dernier au premier.
\end{itemize}

