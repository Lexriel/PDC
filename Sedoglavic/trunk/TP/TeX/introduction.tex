Les  pr\'esentes  notes  constituent un support   de travaux pratiques
consacr\'es au langage~C et destin\'es  \`a des \'etudiants de Licence.
\par
Ces travaux pratiques occupent  un  semestre.  Les machines mises  \`a
disposition sont des  ordinateurs    personnels   sous  linux.     
\ifcorrection
\par
Afin  de facilite les  choses, une correction   de chaque exercice est
disponible. Il est fortement conseill\'e de r\'ediger du code avant de la
consulter.
\fi
\section*{Bref historique}
\label{sec:BrefHistorique}
Le langage~C a \'et\'e  \'elabor\'e  en~$1972$ par Dennis  Ritchie aux
laboratoires Bell pour  le syst\`eme d'exploitation \texttt{UNIX}.  Ce
syst\`eme d'exploitation,   le   compilateur~C  et  la    plupart  des
programmes d'application \texttt{UNIX} sont \'ecrits en~C.
\par
En~$1983$,  l'ANSI d\'ecida de normaliser ce   langage et d\'efinit la
norme ANSI~C  en~$1989$.   Elle fut reprise  int\'egralement en~$1990$
par l'ISO.
\par
Les outils li\'es au langage C (compilateur, etc.) sont gratuit sous
Linux.  Il est disponible --- parmi beaucoup d'autres choses --- dans
la distribution Mandrake par exemple.  Des versions libres sont
disponibles sous Windows.
\section*{Orientation g\'en\'erale du langage}
\label{Orientation}
Le langage~C  est un  langage  imp\'eratif~;  les instructions  qui le
composes se succ\`edent les unes apr\`es les  autres et des structures
de  contr\^ole permettent de  manipuler   ce flot.  Comme la plupart
des langages, C permet
\begin{itemize}
\item l'abstraction des  donn\'ees~: la repr\'esentation des donn\'ees
  est   s\'epar\'ee des op\'erations d\'efinies    sur celles-ci et de
  leurs implantations machine~;
\item l'abstraction des structures de  contr\^ole~: il est inutile  de
  nommer   individuellement les points  de  contr\^ole  (comme dans un
  programme assembleur).
\end{itemize}
C n'est pas un langage de \textit{haut niveau}. Ainsi, il ne pr\'evoit
pas d'instruction traitant des objets compos\'es comme des cha\^\i{}nes
de caract\`eres, des listes, etc.  Pour comparer deux cha\^\i{}nes, il
faut  utiliser une  fonction.   Le   langage~C  est ind\'ependant   de
l'architecture de toute machine. Malgr\`es tout,  c'est un langage qui
reste efficace et puissant.
\section*{Bibliographie}
\label{Bibliographie}
En ce qui concerne la bibliographie, on peut citer comme ouvrage de
r\'ef\'erence~\cite{KernighanRitchie1986} pour la compr\'ehension du
langage et~\cite{Sedgewick2001} pour un aper\c{c}u des algorithmes
fondamentaux.

%\section*{Calendrier premier semestre~2005}
%\label{sec:Calendrier}
%Cette section pr\'esente un calendrier pr\'evisionnel de progression
%pour le semestre en cours. Si vous d\'ecrochez de ce calendrier,
%n'h\'esitez pas \`a venir m'en parler.

% \newcounter{pub}

%\begin{list}{Semaine~\arabic{pub}.}%
%{\usecounter{pub}\setcounter{pub}{2}%
%\setlength{\itemsep}{\smallskipamount}%
%\setlength{\labelsep}{1.5\labelsep}}
%\item Shell
%  \begin{itemize}
%  \item[Cours-TD]~: l'interpr\'eteur de commandes interactif~;
%  \item[TP]~: th\`eme~\ref{cha:Shell} sur l'utilisation du shell.
%  \end{itemize}
%\item Premiers pas en~C
%  \begin{itemize}
%  \item[Cours-TD]~: premier programme, expressions et instructions~;
%  \item[TP]~: th\`eme~\ref{cha:PremierPas} sur les premiers pas en~C
%    (compilation, expressions, instructions).
%  \end{itemize}
%\item Exercices
%  \begin{itemize}
%   \item[Cours-TD]~: exercices~;
%   \item[TP]~: th\`eme~\ref{cha:InstructionDeControle} sur les
%     instructions de contr\^ole.
%  \end{itemize}
%\item Modularit\'es et tableau  
%  \begin{itemize}
%  \item[Cours-TD]~: notions de fonctions, de tableaux, compilation
%    s\'epar\'ee, make et gdb~;
%  \item[TP]~: th\`eme~\ref{cha:Recursivite} sur les fonctions et
%    th\`eme~\ref{cha:Tableaux} sur les tableaux.
%  \end{itemize}
%\item Interruption p\'edagogique.
%\item Types compos\'es.
%  \begin{itemize}
%  \item[Cours-TD]~: structures, unions. \'Enum\'erations et champs de
%    bits. D\'efinitions de nouveaux types. Directives au
%    pr\'eprocesseur~;
%  \item[TP]~: th\`eme~\ref{cha:TypesComposes} sur les types
%    compos\'es.
%  \end{itemize}
%\item Premier contact avec les pointeurs.
%  \begin{itemize}
%  \item[Cours-TD]~: pointeurs, notions de bases~;    
%  \item[TP]~: d\'ebut du th\`eme~\ref{cha:Pointeurs} sur les pointeurs.
%  \end{itemize}
%\item Pointeurs~: compl\'ements~;    
%  \begin{itemize}
%  \item[Cours-TD]~: pointeurs, compl\'ements~;    
%  \item[TP]~:  fin du th\`eme~\ref{cha:Pointeurs} sur les pointeurs.
%  \end{itemize}
%\item Premier projet~: manipulation d'images.
%  \begin{itemize}
%  \item[Cours-TD]~: structures auto-r\'ef\'erentes~;
%  \item[TP]~: th\`eme~\ref{cha:StructuresAutoreferencees} les
%    structures auto-r\'ef\'erentes.
%  \end{itemize}
%\item Pile d'ex\'ecution.
%  \begin{itemize}
%  \item[Cours-TD]~: pile d'ex\'ecution~;
%  \item[TP]~: th\`eme~\ref{cha:ComplementsFonctions} compl\'ement sur les fonctions.
%  \end{itemize}
%\item Contr\^ole continu et remise du premier projet.
%  \begin{itemize}
%   \item[Cours-TD]~: \'evaluation \'ecrite pour le contr\^ole continu~;
%  \item[TP]~:  \'evaluation --- programme \`a faire et \`a rendre en temps limit\'e~;
%  \end{itemize}
%\item Pr\'esentation du second projet.
%  \begin{itemize}
%   \item[Cours-TD]~: les classes d'allocations en~C~;
%  \item[TP]~: pr\'esentation du second projet et bilan du contr\^ole continu.
%  \end{itemize}
%\item Interruption p\'edagogique.
%\item Interruption p\'edagogique.
%\item Remise du second projet. 
%  \begin{itemize}
%   \item[Cours-TD]~: le d\'ebordement de pile~;
%  \item[TP]~: exercices de synth\`ese.
%  \end{itemize}
%\item Derni\`ere semaine --- r\'evisions.
%  \begin{itemize}
%   \item[Cours-TD]~: un exemple de synth\`ese. B\'etisier~;
%  \item[TP]~: exercices de synth\`ese.
%  \end{itemize}
%\item Pas de cours.
%\item Premi\`ere session d'examens.
%\item Premi\`ere session d'examens.
%\end{list}
