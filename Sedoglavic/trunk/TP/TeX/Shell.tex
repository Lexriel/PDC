Pour se familiariser avec l'interpr\'eteur de commandes, on se propose
d'utiliser l'aide en ligne afin de r\'ealiser un certains nombres de
t\^aches. Ces derni\`eres sont fortement inspir\'ees des contraintes
locales i.e.\ de la configuration logicielle offerte aux \'etudiants
\`a Lille.
\section{Manipulation du syst\`eme de fichiers }
\label{sec:ShellFileSystem}
Pour pr\'eparer notre travail, nous allons configurer notre
environnement. Pour se faire, nous utiliserons les commandes shell~:
\begin{verbatim}
cat, cd, cp, df, du, grep, less, ls, mkdir, tar.
\end{verbatim}
Ce document ne donne pas d'indication sur ces commandes mais vous
\^etes incit\'e \`a utiliser l'aide en ligne \verb?man? afin d'en
obtenir.
\begin{exercice}[Construire une arborescence]
  Cr\'eer un r\'epertoire~C qui contiendra l'ensemble des fichiers
  associ\'es au cours.  Dans ce r\'epertoire, cr\'eer les
  r\'epertoires projet et tp.
\end{exercice}
\begin{exercice}[Droits d'acc\`es]
  En utilisant la commande \verb?ls -l?, d\'eterminez les droits
  d'acc\`es des sous r\'epertoires de ma racine (\~{}sedoglav).
  \par
  Construisez vous un r\'epertoire public dans lequel les membres de
  votre groupe pourrons lire mais pas \'ecrire.
\end{exercice}
\begin{exercice}[La commande ls]
  La commande ls permet d'examiner le contenu des r\'epertoires. Il
  est utile de conna\^\i{}tre certaines de ces options.
  \par
  D\'eterminez l'effet des options suivantes de ls~:
  \begin{center}
    a \qquad d  \qquad l \qquad r \qquad t 
  \end{center}
  Remarquez que ces options sont cumulables.
  \par
  Comment visualiser la date de cr\'eation de l'ensemble des sous
  r\'epertoires de votre racine.
  
\end{exercice}
\begin{exercice}[Configurer la taille des polices par d\'efaut]
  Votre professeur a l'intention d'\'economiser le peu de forces qui
  lui restent~; par exemple, il n'a pas l'intention de se fatiguer les
  yeux \`a chercher des erreurs dans vos codes sources alors que les
  polices de caract\`eres utilis\'ees sont minuscules.
  \par
  Sans \'editeur de texte, afficher le contenu du fichier
  \verb?.Xdefaults? (s'il existe). Ce fichier permet de fixer des
  param\`etres par d\'efaut --- comme la taille des polices
  utilis\'ees --- pour certaines applications --- comme xterm, emacs,
  etc.
  \par
  Vous trouverez dans le r\'epertoire \verb?~sedoglav/public/? un
  fichier qui contient la cha\^\i{}ne de caract\`eres \verb?r24?. Ce
  dernier fixe quelques r\'eglages qui faciliterons vos rapports avec
  votre enseignant de~C pr\'ef\'er\'e.
  \par
  Copier ce fichier sur votre racine et ajouter ses r\'eglages \`a
  votre fichier \verb?.Xdefaults? (si possible sans user d'un \'editeur
  de texte). Si ce dernier n'existe pas, cr\'eer le.
  \par
  Pour v\'erifier que tout va bien, utiliser la commande %
  \verb?xrdb .Xdefaults? dans votre shell et lancer un \verb?xterm?
  en arri\`ere fond.
\end{exercice}
\begin{exercice}[Gestion de l'espace disque]
  Votre espace disque est limit\'e (et g\'en\'eralement pollu\'e par
  votre butineur). 
  \par
  D\'eterminer le r\'epertoire occupant le plus d'espace disque.
  \par
  Quels sont les droits du r\'epertoire \verb?/tmp/?~?
  \par
  Combien d'espace disque est disponible dans ce r\'epertoire (dont le
  contenu n'est malheureusement pas sauvegard\'e)~?
\end{exercice}

\begin{exercice}[La commande tar]
  Des archives sont disponibles dans le r\'epertoire
  \verb?~sedoglav/public/?. Par convention, une archive se termine par
  le suffixe tar (tgz si elle est compress\'ee).
  \begin{enumerate}
  \item Listez le contenu de l'archive non compress\'ee (commande
    \verb?tar tvf <nomarchive>?). 
  \item Extrayez de l'archive non compress\'ee celle qui est
    compress\'ee (commande %
    \verb?tar xvf <nomarchive> <fichieraextraire>?).
  \item Listez le contenu de l'archive compress\'ee (ajoutez un z \`a
    la liste des options ci-dessus).
  \item D\'ecompressez et ouvrez l'archive compress\'ee afin d'obtenir son
    contenu (commande \verb?tar xvfz <nomarchive> <nomrep>?).
  \item Utilisez la commande \verb?tar cvfz <nomarchive> <nomrep>? pour faire
    une archive compress\'ee d'un de vos r\'epertoires.
  \end{enumerate}
\end{exercice}
\begin{exercice}[Liens symboliques]
  Vous avez peut \^etre d\'ej\`a \'et\'e surpris par le fait que nos
  voisins flamands s'obstinent --- outre \`a parler une langue
  incompr\'ehensible --- \`a affubler certaines villes de noms
  impossibles. Par exemple~:
  \begin{center}
    \begin{tabular}{c|c}
Nom fran\c{c}ais & Nom flamand \\ \hline
Anvers           &     Antwerpen \\
Bruxelles        &     Brussel\\
Gand             &    Gent\\
La haye          &    Den Haag \\
Lille            &    Rijsel \\
Paris            &     Paris \\
Tournai          &     Doornik 
    \end{tabular}
  \end{center}
  Nous allons profiter de cet \'etat de fait pour nous exercer \`a
  faire des liens symboliques. Ces \emph{liens} permettent le partage
  de fichiers sans duplication de ces derniers (man ln).
  \par
  \begin{enumerate}
  \item Construisez un r\'epertoire villes contenant un fichier pour
    chaque nom de ville en fran\c{c}ais (man touch).
  \item Construisez un r\'epertoire steden contenant un lien
    symbolique pour chaque nom de ville en flamand et qui pointe vers
    le fichier correspondant dans le r\'epertoire villes.
  \end{enumerate}
\end{exercice}
\section{Expressions r\'eguli\`eres}
\label{sec:ExpressionsRegulieres}
Les expressions r\'eguli\`eres d\'ecrivent des r\`egles des
propri\'et\'es de cha\^\i{}nes de caract\`eres. Pour ce faire, on
utilise en shell les \textit{m\'etacaract\`eres} suivants~:
\begin{itemize}
\item le  point d'interrogation  \texttt{?}\ correspond \`a  n'importe
  quel     caract\`ere   (sauf    EOL).   L'expression   r\'eguli\`ere
  \texttt{b?l} repr\'esente    les   cha\^\i{}ne  \textit{bal}      et
  \textit{bol} et toutes les autres combinaisons comme \textit{bwl}~;
\item la  paire de  crochet  \texttt{[ ]}  permet de  sp\'ecifier plus
  restrictivement  un      ensemble   de  caract\`eres.   L'expression
  r\'eguli\`ere     \texttt{dupon[dt]}   ne  repr\'esente    que   les
  cha\^\i{}nes \textit{dupond}  et    \textit{dupont}.    L'expression
  r\'eguli\`ere     \texttt{dupon[d-t]}     repr\'esente   toutes  les
  cha\^\i{}nes commen\c{c}ant  par \textit{dupon}  et se terminant par
  une   lettre comprise  entre \textit{d}  et~\textit{t}. L'expression
  r\'eguli\`ere   \texttt{dupon[\^{}dt]} repr\'esente    toutes    les
  cha\^\i{}nes commen\c{c}ant par \textit{dupon} et ne se terminant ni
  par un~\textit{d} ni par un~\textit{t}~;
\item l'\'etoile  \texttt{*} d\'esigne~$0,1$ ou plusieurs caract\`eres
  quelconques.   L'expression   r\'eguli\`ere  \texttt{*} repr\'esente
  toutes les cha\^\i{}nes.
\end{itemize}
Le pr\'efixe~$\backslash$  (antislash) transforme un m\'etacaract\`ere
en caract\`ere.  \index{Shell!expression r\'eguli\`ere}
\begin{exercice}[Shell et expressions r\'eguli\`eres]
  Nous allons faire des recherches dans le r\'epertoire
  \verb?/usr/include?. Ce dernier regroupe des fichiers d'ent\^ete
  contenant des d\'efinitions qui nous serons utiles par la suite.
  \begin{enumerate}
  \item Donnez la liste des fichiers commen\c{c}ant par std (utilisez
    ls et les expressions r\'eguli\`eres).
  \item Donnez la liste des fichiers dont le nom contient au moins un
    chiffre.
  \item Donnez la liste des fichiers contenant les mots putchar et
    getchar (utilisez grep et les expressions r\'eguli\`eres.
  \item Donnez la liste des fichiers contenant le mot EOF.
  \end{enumerate}
\end{exercice}
\section{Variable et commandes internes du shell}
\label{sec:CommandesInternes}
\begin{exercice}[Variables du shell]
  Stockez dans un fichier \verb?foo.temp? l'ensemble des variables
  d\'efinies dans votre shell.
  \par
  Stockez dans un fichier \verb?bar.temp? l'ensemble des commandes
  externes accessibles dans votre shell sans avoir \`a fournir de
  chemin.
\end{exercice}

\begin{exercice}[La commande which]
  Que permet la commande which~?
  \par
  Que vous apprend l'instruction \verb?which echo?.
\end{exercice}

\begin{exercice}[Historique des commandes]
Le sell g\`ere une liste des derni\`eres commandes ex\'ecut\'ees dont la
taille est donn\'ee par la variable \verb+history+

Si, cette variable  est initialis\'ee \`a~$40$, ce  m\'ecanisme permet
d'obtenir  un historique  des~$40$ derni\`eres commandes ex\'ecut\'ees
et   de redemander   l'ex\'ecution  de  ces  commandes.  La   commande
\verb+history+ permet d'obtenir l'historique des derni\`eres commandes
avec un num\'ero d'ordre associ\'e \`a  chaque commande. Les commandes
ainsi num\'erot\'ees peuvent �tre r\'e-ex\'ecut\'ees simplement~:
\begin{itemize}
  \item \verb+!!+ permet de r\'e-ex\'ecuter la derni\`ere commande~;
  \item \verb+!n+ permet de r\'e-ex\'ecuter la commande num\'ero n~;
     \item \verb+!ch+ permet de r\'e-ex\'ecuter la commande la plus
    r\'ecente commen\c{c}ant par {\em ch}.
\end{itemize}
\begin{enumerate}
\item  Testez la commande  \verb+history+   et la  r\'e-ex\'ecution  de
  commandes selon les trois m\'ethodes pr\'ecis\'ees ci-dessus.
\item Combien de commandes votre historique peut-il m\'emoriser~?
\item Cette commande est elle interne ou externe. Justifiez votre r\'eponse.
\end{enumerate}
\end{exercice}

\begin{exercice}[Les alias]
  Un alias est une abr\'eviation pour une commande ou une s\'erie de
  commandes fr\'equemment utilis\'ees. Vous pouvez visualisez les alias
  d\'ej\`a d\'efinis (ils le sont en g\'en\'eral dans le fichier .cshrc qui est
  ex\'ecut\'e au d\'ebut de chaque shell) par la commande alias.
  \begin{enumerate}
  \item Donnez la liste des alias d\'ej\`a d\'efinis dans votre shell.
  \item D\'efinissez l'alias \verb?h? qui constituera un raccourci pour la commande history.
  \item D\'efinissez l'alias \verb?ll? qui affiche sous format long le
    contenu d'un r\'epertoire tri\'e par date de derni\`ere
    modification.
  \end{enumerate}
\end{exercice}

\section{Utilisation de commandes externes standards}
\label{sec:CommandesExternes}

\begin{exercice}[La commande find]
  La commande find permet de retrouver des fichiers. En particulier,
  il est possible de rechercher des fichiers dont le nom correspond
  \`a un motif d\'efini par une expression r\'eguli\`ere.
  \par
  En utilisant la commande find~:
  \begin{enumerate}
  \item Donnez la liste des fichiers dont le nom se termine par un
    tilde (il s'agit de copie de sauvegarde engendr\'ee par emacs).
  \item De plus, la commande find permet d'ex\'ecuter une commande.
    Donnez la commande permettant de trouver et d'effacer les fichiers
    inutiles dans votre arborescence. Pour effacer les copies faites
    par les \'editeurs, on peut utiliser %
    \verb?find *~ -exec rm {} \;?.
  \end{enumerate}
\end{exercice}
\begin{exercice}[Notion de montage~: la commande mount]
  Certains syst\`eme de fichier --- NTFS par exemple --- distinguent
  les arborescences situ\'ees physiquement sur des supports
  diff\'erents (disquettes, cdrom, disques durs, clefs USB, r\'eseaux,
  etc).
  \par\smallskip
  D'autres ne  font  pas cette distinction gr\^ace   \`a  la notion de
  \emph{montage}.   Il s'agit    d'associer  \`a un  r\'epertoire  une
  arborescence  de fichiers   stock\'e sur un support.
  \par\smallskip
  Ainsi, une disquette format\'ee sur un ordinateur utilisant un OS de
  type windows aura une arborescence de fichiers cod\'ees par le File
  System FAT. Il est possible de la \lq monter \rq sur une
  arborescence de type unix et d'acc\'eder \`a ces fichiers.
  \par
  Le m\^eme principe s'applique aux cdroms, aux FS accessibles par
  r\'eseaux, etc.
  \begin{enumerate}
  \item Consulter le fichier \verb?/etc/fstab? pour avoir la liste des
    supports montable sur votre machine.
  \item Consulter le fichier \verb?/etc/mtab? pour avoir la liste des
    supports actuellement mont\'e sur votre machine.
  \item Quel r\'epertoire est utilis\'e pour monter un cdrom sur votre
    machine~?
  \item Utilisez la commande mount pour monter une disquette, une clef
    ou un cdrom sur votre machine.
  \item Utilisez la commande umount pour d\'emonter une disquette, une clef
    ou un cdrom sur votre machine.
  \end{enumerate}
\end{exercice}
\section{Gestion des processus}
\label{sec:GestionProcessus}
L'ex\'ecution     d'un   programme   provoque    la cr\'eation    d'un
\emph{processus}.  Si le  programme est ex\'ecut\'e plusieurs fois, un
processus sera    cr\'e\'e par  ex\'ecution.    Chaque  processus  est
identifi\'e par un num\'ero, son \verb+PID+.
\par
Au cours de sa vie, un processus peut passer par diff\'erents \'etats~:
\begin{itemize}
\item \verb+R+ pour Run~: il est actif~;
\item \verb+T+ pour Traced~: il est passif et attend d'�tre remis en activit\'e~;
\item \verb+S+ pour Sleep~: il dort, et se r\'eactivera tout seul en temps utile~;
\item \verb+Z+ pour Zombie~: il est mort, mais il occupe encore de la m\'emoire.
\end{itemize}
La commande \verb+kill+ permet d'envoyer des signaux aux processus
afin d'influer sur leur existence.  La syntaxe est~: %
\verb+kill -s pid+ ou \verb+kill pid+, o\`u \verb+pid+ d\'esigne le
num\'ero d'un processus (on peut en mettre plusieurs s\'epar\'es par
des espaces), et \verb+s+ d\'esigne le signal \`a envoyer.
\par
La liste compl\`ete des signaux s'obtient en ex\'ecutant la commande
\verb+kill -l+.
\begin{exercice}[Manipulation de processus]
  \begin{enumerate}
  \item Obtenez la liste de l'ensemble des processus existant sur
    votre machine.
  \item Lancer la commande \verb?du -h / &?. Suspendez le processus
    correspondant puis relancer le pour finir par le tuer. Comment
    auriez vous du lancer cette commande afin de ne pas polluer la
    sortie standard.
  \end{enumerate}
\end{exercice}

\begin{exercice}[T\^ache de fond]
  \begin{enumerate}
  \item Que provoque la commande sleep~?
  \item Tentez de d\'eterminez \`a l'avance le comportement des instructions suivantes~:
\begin{verbatim}
sleep 5 ; echo A
echo A ; sleep 5
sleep 5 & echo A
echo A & sleep 5
(echo A ; sleep 5) &
echo A ; sleep 5 ; echo B
echo A ; sleep 5 & echo B
(echo A ; sleep 5 ) & echo B
echo A ; (sleep 5 & echo B)
echo A ; (sleep 5 ; echo B) &
sleep 5 & echo A ; ( sleep 5 ; echo B)
sleep 5 & echo A & ( sleep 5 ; echo B)
sleep 5 & echo A & ( sleep 5 & echo B) &
sleep 5 & echo A & ( sleep 5 ; echo B) &
\end{verbatim}
  \end{enumerate}
\end{exercice}

\begin{exercice}[Jouons avec les redirections]
  Lancer un shell puis d\'eterminez le num\'ero du processus associ\'e
  \`a ce shell.
  \par
  Dans le r\'epertoire proc, donnez la liste --- sous format long ---
  de l'ensemble des r\'epertoires commen\c{c}ant par un chiffre.
  \par
  Remarquez qu'un nom de r\'epertoire est identique au num\'ero du
  processus shell.
  \par
  Placer le r\'epertoire courant dans le sous-r\'epertoire fd du
  r\'epertoire portant comme nom le num\'ero d'identification
  (utilisez cd).
  \par
  Lister sous format long le contenu de ce r\'epertoire.
  \par
  Utilisez la commande \verb?echo Salut? pour afficher Salut sur la
  sortie standard.
  \par
  Utilisez la commande \verb?echo Salut? et une redirection pour
  placer la cha\^\i{}ne Salut dans un fichier contenu dans le
  r\'epertoire tmp.
  \par
  Utilisez la commande \verb?echo Salut? et une redirection pour
  placer la cha\^\i{}ne Salut dans un des fichiers pr\'eexistant dans
  le r\'epertoire fd.
  \par
  Bravo, vous venez de d\'ecouvrir que la sortie standard (ce qui est
  affich\'e \`a l'\'ecran correspond \`a un fichier comme les autres
  de l'arborescence).
\end{exercice}
\section{Environnement r\'eseaux}
\label{sec:Reseaux}
\begin{exercice}[Un pr\'etexte pour manipuler les commandes du shell]
  La commande \verb?ypcat passwd? permet d'afficher sur la sortie
  standrad des informations concernant les utilisateurs ayant un
  compte sur le r\'eseau~; ces informations sont~:
  \begin{itemize}
  \item leur login~;
  \item leur mot de passe crypt\'e~;
  \item leur num\'ero d'identification~;
  \item le num\'ero  d'identification de leur groupe~;
  \item leurs nom et pr\'enoms~;
  \item leurs racine~;
  \item le shell utilis\'e.
  \end{itemize}
  \begin{enumerate}
  \item Combien y a-t'il d'utilisateur ayant acc\`es \`a votre
    machine~? (utilisez la commande nl).
  \item Trier cette liste par ordre alphab\'etique (utilisez sort).
  \item Combien y a-t'il d'enseignant~? (utilisez grep et nl).
  \item Afficher les noms et pr\'enoms de l'ensemble des enseignants
    (utilisez en compl\'ement des commandes ci-dessus, la commande
    cut).
  \end{enumerate}
  Vos r\'eponses ne devront pas utilisez de fichiers temporaire.
\end{exercice}
\begin{exercice}[Ouverture d'une session sur une machine distante]
  La commande ssh permet d'ouvrir une session sur une machine
  distante.  Par exemple, la commande %
  \verb?ssh nom.fil.univ-lille1.fr? permet d'ouvrir une session sur la
  machine \verb?nom? du r\'eseau du M~5.
  \par
  Ouvrez une session sur la machine \`a cot\'e de vous.
\end{exercice}

\begin{exercice}[Un utilitaire pratique]
  wget un programme non-interactif de t\'el\'echargement de fichiers
  depuis le Web i.e.\ il peut travailler en arri\`ere plan sans
  intervention de l'utilisateur.
  \par
  Avec l'aide en ligne de wget, construisez une archive compress\'ee
  contenant l'ensemble des pages web constituant le fascicule de
  travaux pratiques (afin par exemple de l'utiliser hors connexions
  chez vous).
\end{exercice}
