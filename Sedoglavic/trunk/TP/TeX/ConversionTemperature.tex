\begin{exercice}[Conversion de temp\'erature]
  La temp\'erature se quantifie d'apr\`es diff\'erentes \'echelles~:
  \begin{itemize}
  \item  \textit{L'\'echelle   Celsius},  qui a   pour   rep\`eres les
    temp\'eratures~$0\tmpdeg$C   (glace fondante)     et~$100\tmpdeg$C
    (\'ebullition  de l'eau) comporte,   entre ces  deux points,~$100$
    degr\'es Celsius.
  \item \textit{L'\'echelle  Fahrenheit}   en   usage dans  les   pays
    anglo-saxons utilise le mercure comme   \'etalon. La glace   fonds
    \`a~$32\tmpdeg$F et l'eau bout \`a~$212\tmpdeg$F.
  \item  \textit{L'\'echelle    absolue} comprend       toujours   des
    temp\'eratures positives, qui sont compt\'ees en Kelvin \`a partir
    du z\'ero absolu~($0$K=$-273\tmpdeg$C).
  \end{itemize} 
  \'Etablir  les r\`egles   de  conversion entre ces  \'echelles  (par
  exemple, K=$\tmpdeg$C~$+273$).
  \par
  Construire  un  programme   qui   permet  la  conversion  entre  ces
  diff\'erentes \'echelles.  Apr\`es  avoir   permit la saisie  de  la
  temp\'erature, ce   programme  devra   tout  d'abord  demander   \`a
  l'utilisateur  de    saisir   l'\'echelle   dans     laquelle  cette
  temp\'erature est  exprim\'ee puis  s'enqu\'erir de  l'\'echelle dans
  laquelle on veut faire la conversion.
  \par
  Enfin,  ce programme  devra  permettre gr\^ace  \`a  une   boucle de
  recommencer cette op\'eration   autant de fois que  l'utilisateur le
  d\'esirera.
  \ifcorrection
  \begin{correction}
    \input{Verbatim/conversiontemp.c.verb}
  \end{correction}
  \fi
\end{exercice}