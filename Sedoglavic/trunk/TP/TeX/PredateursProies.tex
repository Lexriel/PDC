\begin{exercice}[Syst\`eme proies-pr\'edateurs]
  Nous allons mod\'eliser  l'\'evolution   de  lapins et  de   renards
  virtuels.
  \par
  On  consid\`ere une population  initiale  de  lapins  constitu\'ee
  de~$38195$ lapins et une population initiale de renards constitu\'ee
  de~$200$ renards.
  \par
  Dans notre  mod\`ele,  ces deux populations  sont virtuelles  car on
  suppose     qu'elles     sont    d\'ecrites    par     deux   suites
  r\'ecurrentes~$(u_{j})$   et~$(v_{j})$.     Ainsi,    elles  ne   se
  renouvellent qu'une fois par cycle (par an, par exemple).
  \par
  Les    terme~$u_{i}$  et~$v_{i}$  des  suites~$(u_{j})$ et~$(v_{j})$
  d\'ecrivent le nombre de lapins et de  renards la~$i$i\`eme ann\'ee. 
  Mais pour les  calculer, on doit  utiliser des nombres flottants car
  si deux couples de lapins ont en moyenne  trois petits lapins sur un
  an, le taux de reproduction de  la population est le nombre flottant
  associ\'e  \`a~$3/4$. Avec  ce taux,   si  une ann\'ee le  nombre de
  lapins est~$u_{j}$, l'ann\'ee suivante  le nombre de lapin~$u_{j+1}$
  sera l'entier le plus proche de~${3u_{j}/4}$.
  \par
  Les d\'emographes  lapins et renards ont  constat\'e que  le taux de
  reproduction des lapins est~:
  $$
  L_{j}=C(1-\frac{u_{j}}{10^{5}}  -\frac{v_{j}}{5000})
  $$
  et que celui des renards est
  $$
  R_{j}=\frac{1-\frac{v_{j}}{25000} + \min(4,\frac{u_{n}}{1000})}{4}.
  $$
  Ces taux sont des nombres flottants~;~$C$ repr\'esente le taux de
  reproduction des lapins (un entier compris entre~$1$ et~$4$ inclus).
  \par
  Ainsi, les populations \'evoluent suivant les r\'ecurrences~:
  $$
  u_{j+1} = L_{j}u_{j}\quad 
  \mathrm{et}\quad   v_{j+1}   =  R_{J}v_{j}.
  $$
  Les populations sont des nombres entiers,  il faut donc convertir
  en entier les r\'esultats flottants obtenus apr\`es multiplication.
  \par
  Construire  une proc\'edure qui permet  de saisir au clavier le taux
  de reproduction des  lapins et qui affiche  les nombres de lapins et
  de   renards pour les     deux  cents premi\`eres ann\'ees.    Cette
  proc\'edure devra   permettre   \`a l'utilisateur   de faire  autant
  d'exp\'erience qu'il le souhaite.
  \ifcorrection\begin{correction}
  \input{Verbatim/lapinsEtRenards.c.verb}
  \end{correction}
\fi
\par
\textbf{Remarques bibliographiques.} %
Ce   probl\`eme   est  tir\'e  de  travaux   pratiques  propos\'es par
\url{http://www.ircam.fr/equipes/analyse-synthese/tassart}{St\'ephan
  Tassart}. Il est inspir\'e par un  article de \textit{La Recherche},
num\'ero~$296$.
\end{exercice}
