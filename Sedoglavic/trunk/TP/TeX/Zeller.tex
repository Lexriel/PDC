\begin{exercice}[D\'etermination du jour correspondant \`a une date]
On d\'esire obtenir, \`a partir d'une date, le jour de la semaine \`a
laquelle elle correspond. Pour cela on utilise la formule de Zeller~:
$$
\left(\frac{13*mm-1}{5}+j+aa+\frac{aa}{4}+\frac{ss}{4}-2*ss \right)
~\%~7
$$
avec les notations~:
\begin{description}
  \item[${a}/{b}$] repr\'esente la division enti\`ere de a par
    b~;
  \item[$a \% b$] donne le reste de la division enti\`ere de a par
    b~;
  \item[j] est le num\'ero du jour dans le mois~;
  \item[mm] est le num\'ero du mois dans l'ann\'ee, diminu\'e de 2
    pour tous les mois sauf janvier et f\'evrier, num\'erot\'es
    respectivement 11 et 12, et qui sont consid\'er\'es comme
    appartenant \`a l'ann\'ee pr\'ec\'edente~;
  \item[aa] est le nombre compos\'e des 2 derniers chiffres de
    l'ann\'ee~;
  \item[ss] est le nombre compos\'e des chiffres de l'ann\'ee sauf
    les 2 derniers\footnote{ aa et ss peuvent \^etre modifi\'es
      par la remarque pr\'ec\'edente sur janvier et f\'evrier.  }.
\end{description}

On obtient ainsi un nombre de 0 \`a 6. Le nombre 0 correspond \`a
dimanche, le nombre 1 \`a lundi, etc.\ le nombre 6 \`a samedi.
    
La formule pr\'ec\'edente n'est  valable qu'apr\`es le 15 octobre 1582
du   fait du changement  de calendrier  le jeudi 4  octobre 1582~: le
lendemain de ce jour a \'et\'e le vendredi 15 octobre 1582.
\paragraph{Questions~:}
\begin{enumerate}
  \item  Transformer par une conditionnelle  la formule pour la rendre
    valable  pour  toute  date  (en   tenant compte du   changement de
    calendrier).  Donner le domaine  de validit\'e des donn\'ees j,m,a
    (date donn\'ee)   en respectant les  contraintes  du calendrier (y
    compris les ann\'ees  bissextiles et cons\'equences  du changement
    de calendrier).
  \item \'Ecrire un programme qui prend en entr\'ee une date au format
  jjmmssaa et qui affiche le jour correspondant.
\end{enumerate}
\end{exercice}
