%\documentclass[licence,solution]{controle}
\documentclass[licence]{controle}
\matiere{Pratique du~C}
\genre{Fiche de TD~2}
\commentaire{Langage~C}

\usepackage{epsfig}

\newcommand{\B}[1]{\overline{#1}}

%%%%%%%%%%%%%%%%%%%%%%%%%%%%%%%%%%%%%%%%%%%%%%%%%%%%%%%%%%%%%%%%%%%%%%%%%%%%%%
% graphismes
%

%\pssilent

\newlength{\longDSfig}

\def\normalDSfig{0.7}
\def\petiteDSfig{0.5}
\def\tinyDSfig{0.42}
\def\petiteXpixDSfig{0.9}
\def\smallDSfig{0.4}

\newcommand{\DSfig}[2]
{ \settowidth{\longDSfig}{\epsfig{figure=#1}}%
  \setlength{\longDSfig}{#2\longDSfig}%
  \ifthenelse{\lengthtest{\longDSfig > \linewidth}}%
    {\setlength{\longDSfig}{\linewidth}}%
    {\setlength{\longDSfig}{\longDSfig}}%
  \epsfig{figure=#1,width=\longDSfig}}

% the end


\begin{document}

%%%%%%%%%%%%%%%%%%%%%%%%%%%%%%%%%%%%%%%%%%%%%%%%%%%%%%%%%%%%%%%%%%%%%%
%%%%%%%%%%%%%%%%%%%%%%%%%%%%%%%%%%%%%%%%%%%%%%%%%%%%%%%%%%%%%%%%%%%%%%

\section{Identificateurs, types simples, op\'erateurs, priorit\'e et conversion implicite}
\begin{question}
Indiquer si les identificateurs suivants sont valides~:
\begin{center}
\begin{tabular}{lll}
foo-1 & \_foo\_bar & 3cavaliers \\
foo. & t\^ete & \_\_\_A\_ \\
\_ & a & 3 
\end{tabular}
\end{center}
\end{question}
\begin{solution}
\begin{itemize}
\item[foo-1] contient un op\'erateur et n'est donc pas valide~;
\item[\_foo\_bar] est valide~;
\item[3cavaliers] commence par un entier et n'est donc pas valide~;
\item[foo.] contient un op\'erateur et n'est donc pas valide~;
\item[t\^ete] contient un accent et n'est donc pas valide~;
\item[\_\_\_A\_] est valide~;
\item[\_ et a] sont valides~;
\item[3]  n'est pas valide.
\end{itemize}
\end{solution}
\begin{question}
En supposant~:
$$
A=20 \quad B=5 \quad C=-10 \quad D=2 \quad X=12 \quad Y=15
$$
\'evaluer les expressions valides parmi les expressions suivantes~:
$$
\begin{array}{llll}
5*X+2*3*B/4 & A == B = 5 & A += X + 2 & A != C *= -D\\
A \%= D++ & A \%= ++D & (X++) * A+C & A = X*B<C+Y*!(B<C) \\
!(X-D+C) | |D  & A \&\&B | | !0\&\&C\&\&!D &C=12-- & (1<<(2<<1))==((1<<2)<<1) 
\end{array}
$$
En utilisant le parenth\'esage, corriger les expressions invalides.
\end{question}
\begin{solution}
  Il serait p\'edagagique de rappeler les op\'erateurs.
  \begin{enumerate}
  \item[$5*X+2*3*B/4$] pas de pb, de gauche \`a droite avec priorit\'e
    de la multiplication sur l'addition~;
  \item[$A == B = 5$] pas valide. On commence par == qui retourne~0 et
    l'affectation n'accepte qu'un identificateur sur la gauche~;
  \item[$A += X + 2$] on commence par l'addition puis l'affectation additive~;
  \item[$A != C *= -D$] on commence par -D. on fait C *= -D et il reste
    \`a faire le test diff\'erent de.
  \item[$A \%= D++ $] on commence par A \%= D puis l'incr\'ementation~;
  \item[$A \%= ++D$] on commence par l'incr\'ementation puis le reste~;
  \item[$X++ * A+C$] l'incr\'ementation de~X se fait apr\`es la *~;
  \item[$A = X*B<C+Y*!(B<C)$] on commence par les op\'erateurs de
    comparaisons et on fait l'arithm\'etique (on rapelle 0 faut,
    diff\'erent de~$0$ vrai)~;
  \item[$!(X-D+C) | | D$] idem  le r\'esultat est !0 || D
  \item[$A \&\&B | | !0\&\&C\&\&!D$] 1 || 1 \&\& 1 \&\& 0 qui est \'egal \`a 0~;
  \item[$C=12--$] C est affect\'e \`a~$13$ puis on incr\'emente~;
  \item[$(1<<(2<<1))==((1<<2)<<1)$] suivre les parenth\`eses.
  \end{enumerate}
\end{solution}
\begin{question}
En supposant faites les d\'eclaration~:
\begin{verbatim}
long int A = 15 ;
char B = 'A' ;
short int C = 10 ;
\end{verbatim}
\'evaluer et donner le type des expressions valides suivantes~:
$$
\begin{array}{lll}
B+1 & B+A  & B+C \\
3*B+2*B & 2*B+(A+10)/C & 2*B+(A+10.0)/C \\
B=2*B+(A+10.0)/C & C = 666 & B *= 3.14
\end{array}
$$
\end{question}
\begin{solution}
\end{solution}
  {\tiny
    \begin{center}
      \begin{tabular}{c|l|c}
        16 &
        \verb+()+ 
        \hspace{3mm}
        \verb+[]+ 
        \hspace{3mm}
        \verb+->+
        \hspace{3mm}
        \verb+.+ 
        & G
        \\
        15 &
        \verb|++| 
        \hspace{3mm}
        \verb+--+ (postfix\'e)
        & D
        \\
        14 &
        \verb+!+ 
        \hspace{3mm}
        \verb+~+ 
        \hspace{3mm}
        \verb|++|
        \hspace{3mm}
        \verb+--+ (pr\'efix\'e) 
        \hspace{3mm}
        \verb+-+ (unaire) 
        & D
        \\
        & \verb+*+ (indirection)
        \hspace{3mm}
        \verb+&+ (adresse)
        \hspace{3mm}{\tt sizeof} 
        & D
        \\
        13 &
        \verb+*+ (multiplication)
        \hspace{3mm}
        \verb+/+ 
        \hspace{3mm}
        \verb+%+
        & G
        \\
        12 & 
        \verb|+| 
        \hspace{3mm}
        \verb+-+ 
        & G
        \\
        11 &
        \verb+<<+ 
        \hspace{3mm}
        \verb+>>+       
        & G
        \\
        10 &
        \verb+<+ 
        \hspace{3mm}
        \verb+<=+ 
        \hspace{3mm}
        \verb+>+ 
        \hspace{3mm}
        \verb+>=+ 
        & G
        \\
        9 &\verb+==+ \hspace{3mm}\verb+!=+ & G\\
        8 &\verb+&+ (et bit \`a bit) & G\\
        7 &\verb+^+ & G\\
        6 &\verb+|+ & G\\
        5 &\verb+&&+ & G\\
        4 &\verb+||+ & G\\
        3 &\verb+?:+ & D\\
        2 &\verb+=+ 
        \hspace{3mm}
        \verb|+=| 
        \hspace{3mm}
        \verb+-=+ 
        \hspace{3mm}
        \verb+*=+ 
        \hspace{3mm}
        \verb+/=+ 
        \hspace{3mm}
        \verb+%=+ 
        \hspace{3mm}
        \verb+>>=+ 
        \hspace{3mm}
        \verb+<<=+ 
        \hspace{3mm}
        \verb+&=+ 
        \hspace{3mm}
        \verb+^=+ 
        \hspace{3mm}
        \verb+|=+ 
        & D
        \\
        1 &
        \verb+,+ 
        &G
      \end{tabular}
    \end{center}
}
\newpage
\section{Instructions de contr\^ole}
    La fonction \texttt{getchar()} renvoie le code ASCII du caract\`ere lu
    sur \texttt{stdin}, la valeur \texttt{EOF} en fin de fichier ou
    ou cas d'erreur. 

\begin{question}    
    On entre sur \texttt{stdin} une valeur \texttt{unsigned int}. On
    sort sur \texttt{stdout} sa traduction en binaire.
    \begin{solution}
        \begin{itemize}
          \item On remarque que \texttt{n\%2} donne le bit de poids
            faible de \texttt{n}. 
          \item La r�cursivit� peut faciliter l'obtention des chiffres
            dans le bon ordre.
        \end{itemize}
        \begin{verbatim}
void binaire (unsigned int n)
{
  if (n!=0) {
    binaire (n>>1) ;
    printf("%d", n%2) ;
  }
}                                        
        \end{verbatim}
    \end{solution}
\end{question}
\begin{question}
  Vous  avez d\'ej\`a  remarqu\'e  que  le clavier d'un  t\'el\'ephone
  outre la touche~\#, ressemblait \`a~:
  \par
        \begin{center}
  \begin{tabular}{ccc}
    1 &  2  & 3 \\
    & ABC & DEF \\
    4 & 5 & 6 \\
    GHI & JKL & MNO  \\ 
    7 & 8 & 9 \\
    PQRS &TUV  &WXYZ
  \end{tabular}
        \end{center}
  \par
  Cette  disposition permet \`a l'utilisateur de  taper du texte.  Par
  exemple, presser une fois la touche~$7$ correspond \`a la lettre~$P$
  alors  que  presser cette    touche~$4$    fois correspond   \`a  la
  lettre~$S$.  La touche~\# du t\'el\'ephone   sert \`a s\'eparer  les
  lettres  et peut \^etre omise lorsqu'il   est clair qu'une lettre se
  termine et l'autre commence.
  \par
  Ainsi, la s\'equence \verb+777666222559996#6668866#8244466+
  correspond au mot \verb+rockymountain+.

  Donnez un programme qui permet de traduire  une suite tap\'ee sur le
  clavier d'un t\'el\'ephone en un texte.

  \begin{solution}
\begin{verbatim}
#include "stdio.h"
#include "stdlib.h"
#define MAX 100

void traduction(char code[MAX]){
  
  int i,j ;
  char trad[MAX] ;

  i = j = 0 ;

  while(code[j]!=0){

    switch(code[j]){
    case '#' : j++ ; continue ; break ;
    case '1' : trad[i] =' ' ; break ;
    case '2' : trad[i] ='a' ; break ;
    case '3' : trad[i] ='d' ; break ;
    case '4' : trad[i] ='g' ; break ;
    case '5' : trad[i] ='j' ; break ;
    case '6' : trad[i] ='m' ; break ;
    case '7' : trad[i] ='p' ; break ;
    case '8' : trad[i] ='t' ; break ;
    case '9' : trad[i] ='w' ; break ;
    default : printf("Erreur dans la chaine\n") ; exit(-1) ;
    }  

    j++ ;
    while(code[j-1]==code[j]) {
      trad[i]++ ;
      j++ ;
    } 
    i++ ;
  }
  trad[i] = 0 ;
  printf("La traduction est %s\n",trad) ;
}

int main(void){

  char code[MAX];

  printf("Saisir le code a traduire => ");
  scanf("%s",code) ;
  traduction(code) ;
  return 1 ;

} 
\end{verbatim}
  \end{solution}

\end{question}

\begin{question}    
    On veut calculer le nombre de bits sur lequel sont cod\'es les
    \texttt{unsigned int}. Le format des donn\'ees n'est pas pr\'ecis\'e
    dans les sp\'ecifications du langage C, il peut d\'ependre de la
    machine ou du compilateur.
    
    L'op\'erateur arithm\'etique \verb+x << lg+ %>>
    calcule un d\'ecalage sur \texttt{x} d'une longueur \texttt{lg} bits
    vers les bits de poids fort.  Les \texttt{lg} bits de poids forts
    sont perdus, ceux de poids faible sont mis � 0. 
    
    On remarquera qu'on peut utiliser cet op\'erateur pour une
    multiplication (ou une division avec~\verb+>>+) par 2, 4..., �
    condition de faire attention aux pertes \'eventuelles
    d'informations.

    \begin{solution}
        \begin{itemize}
          \item On remarque que 0 est cod\'e sous forme d'une suite de
            z\'eros et que 1 est cod\'e sous la forme d'une suite de z\'eros
            suivie d'un unique 1. 
        \end{itemize}
        \begin{verbatim}
int longint()
{
  unsigned int V = 1 ;
  int nb_decal = 0;
  while (V != 0) {
    V <<= 1 ;  /* ou  V = V << 1; */
    nb_decal ++ ;
  }
  return nb_decal ;
}
        \end{verbatim}
    \end{solution}
\end{question}


\begin{question}
    On lit sur l'entr\'ee standard (\texttt{stdin}) un texte, termin\'e
    par \texttt{EOF}, dont on veut v\'erifier que le parenth\'esage est
    correct. On sortira sur la sortie standard (\texttt{stdout}) un
    message pour le r\'esultat.
    
    \begin{solution}
        \begin{itemize}
          \item Le \texttt{switch}/\texttt{case}/\texttt{break}
            n'avait pas \'et\'e utilis\'e dans l'exercice pr\'ec\'edent...
          \item Introduction de \texttt{getchar()}.
          \item Introduction de \texttt{exit()}
        \end{itemize}
        Pourquoi \texttt{carac} est d\'eclar\'e \texttt{int} et non
        \texttt{char} 
        \begin{verbatim}
#include <stdio.h>
main () 
{
  int ouvertes = 0 ;
  int carac ;
  while ((carac=getchar()) != EOF)
  switch (carac) {
    case '(' : 
      ouvertes ++;
      break;
    case ')' : 
      if (ouvertes <= 0)
        printf("Erreur de fermeture\n");
        exit (1) ; 
      else
        ouvertes --;
        break;
      default : ;
  }
  if (ouvertes != 0)
    printf ("Manque %d fermetures\n", ouvertes) ;
  else
    printf ("Parenth\'esage correct\n") ;
    exit (0) ; 
 }
        \end{verbatim}
    \end{solution}
\end{question}
\begin{question}    
    On calcule la valeur d'un entier entr\'e sur \texttt{stdin} sous
    forme d'une suite de caract\`eres (termin\'ee par un caract\`ere autre
    qu'un chiffre).

    L'arithm\'etique << classique >> peut s'appliquer sur le type
    \texttt{char}. La valeur num\'erique d'un caract\`ere est le code
    \textsc{ASCII} de ce caract\`ere (65 pour \texttt{'A'}, 66
    pour \texttt{'B'}...). 
        
    \begin{solution}
        \begin{itemize}
          \item Arithm\'etique des caract\`eres
        \end{itemize}
        \begin{verbatim}
unsigned int lire_entier ()
{
  unsigned int resu = 0 ;
  int carac = getchar() ;
  while ((carac >= '0') && (carac <= '9')) {
    resu *= 10 ;
    resu += carac - '0' ;
    carac = getchar() ;
  }
  return resu ;
}
        \end{verbatim}
        Ce qui peut �tre r\'e\'ecrit (au niveau du \texttt{getchar()} : 
        \begin{verbatim}
unsigned int lire_entier ()
{
  unsigned int resu = 0 ;
  int carac ;
  while (carac = getchar (), (carac >= '0') && (carac <= '9')) {
    resu *= 10 ;
    resu += carac - '0' ;
  }
  return resu ;
}
        \end{verbatim}
    \end{solution}
\end{question}
\begin{question}
  La racine carr\'ee par    d\'efaut d'un entier positif~$n$  est   un
  entier positif~$p$ qui v\'erifie~:
  $$
  p^{2}\leq n <(p+1)^{2}.
  $$
  Construiser un programme  qui calcul la racine carr\'ee par d\'efaut d'un entier.
  \par
  Vous pouvez tester deux  approches~:  la recherche lin\'eaire de  la
  racine par d\'efaut et la recherche dichotomique.
\end{question}
  \begin{solution}      
\begin{verbatim}
#include <stdio.h>
int racine_par_defaut(int n){

  int d=n ;
  int g=0 ;
  int i ;
  
  do{
    i = (d+g)/2 ;

    if (i*i>n)
      d = i ;
    else if ((i+1)*(i+1)<n)
      g = i ;
    else return i ;

  } while (1) ;
}
\end{verbatim}
  \end{solution}

\end{document}





