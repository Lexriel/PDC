\documentclass{article}
\usepackage{fullpage}
\advance\hoffset by -3mm  % A4 is narrower.
\advance\voffset by  8mm  % A4 is taller.
%------------------------------------------------------------------------------
% Use a small font for the verbatim environment
\makeatletter  % makes '@' an ordinary character
\renewcommand{\verbatim@font}{%
  \ttfamily\footnotesize\catcode`\<=\active\catcode`\>=\active%
}
\makeatother   % makes '@' a special symbol again

\begin{document}
\begin{verbatim}
FIND(1L)                 Manuel de l'utilisateur Linux                FIND(1L)

NOM   find - Rechercher des fichiers dans une hi\'erarchie de r\'epertoires.

SYNOPSIS  find [chemin...] [expression]

DESCRIPTION
       Cette  page  de manuel documente la version GNU de find.  find parcourt
       les arborescences de r\'epertoires commen�ant en chacun des chemins  men-
       tionn\'es,  en \'evaluant les expressions fournies pour chaque fichier ren-
       contr\'e.  L'\'evaluation de l'expression se fait de gauche  \`a  droite,  en
       suivant  des  r\`egles  de  priorit\'e d\'ecrites dans la section OP\'ERATEURS,
       jusqu'\`a ce que le r\'esultat soit connu (par  exemple  la  partie  gauche
       vraie pour un op\'erateur OU ou fausse pour un op\'erateur ET).

       Le premier argument commen�ant par `-', `(', `)', `,', ou `!'  est con-
       sid\'er\'e comme le d\'ebut de l'expression, tous  les  arguments  pr\'ec\'edents
       sont  des  chemins \`a parcourir. Tous les arguments ult\'erieurs sont con-
       sid\'er\'es comme le  reste  de  l'expression  rationnelle.   de  point  de
       d\'epart.  Si  aucune expression n'est fournie, find utilise l'expression
       `-print' par d\'efaut.

EXPRESSIONS
       L'expression  est  constitu\'ee  d'ooppttiioonnss  (affectant   l'ensemble   des
       op\'erations  plut\^ot  que le traitement d'un fichier particulier, et ren-
       voyant toujours vrai), de tteessttss (renvoyant une valeur vraie ou fausse),
       et  d'aaccttiioonnss  (ayant des effets de bords et renvoyant une valeur vraie
       ou fausse), tous ces \'el\'ements \'etant s\'epar\'es par des op\'erateurs.   Quand
       un  op\'erateur  est manquant, l'op\'eration par d\'efaut -and est appliqu\'ee.
       Si l'expression  ne  contient  pas  d'action  autre  que  -prune,  find
       applique l'action -print par d\'efaut sur tous les fichiers pour lesquels
       l'expression est vraie.

   OPTIONS
       Toutes les options renvoient toujours la valeur vraie. Elles  ont  tou-
       jours  un  effet  global,  plut\^ot  que de s'appliquer uniquement \`a leur
       emplacement dans l'expression.   N\'eanmoins,  pour  am\'eliorer  la  lisi-
       bilit\'e, il est pr\'ef\'erable de les placer au d\'ebut de l'expression.

       -empty fichier  vide.  De plus ce fichier doit \^etre r\'egulier ou \^etre un
              r\'epertoire.
       -type c              Fichier du type c :
              b      fichier sp\'ecial en mode bloc (avec tampon)
              c      fichier sp\'ecial en mode caract\`ere (sans tampon)
              d      r\'epertoire
              p      tube nomm\'e (FIFO)
              f      fichier r\'egulier
              l      liens symbolique
              s      socket
       -name motif
              Fichier  dont  le  nom  de  base (sans les r\'epertoires du chemin
              d'acc\`es), correspond au motif  du  shell.   Les  m\'eta-caract\`eres
              (`*', `?', et `[]') ne sont jamais mis en correspondance avec un
              point `.'  au d\'ebut du nom.  Pour ignorer un  r\'epertoire,  ainsi
              que  tous  ses sous-r\'epertoires, utiliser l'option -prune ; vous
              trouverez un exemple dans la description de l'option -path.
   TESTS
       Les arguments num\'eriques peuvent \^etre indiqu\'es comme suit :
       -user utilisateur
              fichier appartenant  \`a  l'utilisateur  indiqu\'e  (U-ID  num\'erique
              \'eventuellement)
   ACTIONS
       -exec commande ;
              Ex\'ecute la commande ; vrai si le code de retour 0  est  renvoy\'e.
              Tous  les  arguments  suivants de find sont consid\'er\'es comme des
              arguments pour la ligne de commande, jusqu'\`a ce qu'on  rencontre
              un  `;'. 
\end{verbatim}
\begin{verbatim}
CUT(1)                   Manuel de l utilisateur Linux                  CUT(1)

NOM     cut - Supprimer une partie de chaque ligne d'un fichier.

SYNOPSIS  cut  {-b  liste_octets, --bytes=liste_octets} [-n] [--help] [--version]
       [fichier...]

       cut  {-c  liste_caract\`eres,   --characters=liste_caract\`eres}   [--help]
       [--version] [fichier...]

       cut  {-f  liste_champs,  --fields=liste_champs}  [-d  s\'eparateur]  [-s]
       [--delimiter=s\'eparateur]   [--only-delimited]   [--help]    [--version]
       [fichier...]

DESCRIPTION

       cut  affiche  une  partie  de  chaque ligne de chacun des fichiers men-
       tionn\'es, ou de l'entr\'ee standard si aucun fichier n'est indiqu\'e. Un nom
       de fichier `-' correspond \'egalement \`a l'entr\'ee standard.

       La  partie  affich\'ee  est  s\'electionn\'ee  par les options pr\'esent\'ees ci-
       dessous.

   OPTIONS
       Les liste d'octets, liste de caract\`eres, et liste de champs se  pr\'esen-
       tent  sous forme d'un ou plusieurs nombres ou intervalles (deux nombres
       s\'epar\'es par un tiret), s\'epar\'es par des virgules.  La  num\'erotation  des
       octets,  caract\`eres ou champs commence \`a 1.  Des intervalles incomplets
       peuvent \^etre indiqu\'es : `-m' signifiant `1-m'; `n-' signifiant de `n' \`a
       la fin de la ligne.

       _-_b_, _-_-_b_y_t_e_s _l_i_s_t_e___d___o_c_t_e_t_s
              Afficher  uniquement  les octets aux positions indiqu\'ees dans la
              _l_i_s_t_e___d___o_c_t_e_t_s.  Les tabulations et  les  caract\`eres  BackSpaces
              sont  trait\'es  comme  tous les autres caract\`eres, ils occupent 1
              octet.

       _-_c_, _-_-_c_h_a_r_a_c_t_e_r_s _l_i_s_t_e___d_e___c_a_r_a_c_t_\`e_r_e_s
              Afficher uniquement les caract\`eres aux positions indiqu\'ees  dans
              la  _l_i_s_t_e___d_e___c_a_r_a_c_t_\`e_r_e_s.   Pour l'instant c'est \'equivalent \`a -b,
              mais cette option diff\'erera  avec  l'internationalisation.   Les
              tabulations et les caract\`eres BackSpaces sont trait\'es comme tous
              les autres caract\`eres, ils occupent 1 caract\`ere.

       _-_f_, _-_-_f_i_e_l_d_s _l_i_s_t_e___d_e___c_h_a_m_p_s
              N'afficher que les champs indiqu\'es dans la _l_i_s_t_e___d_e___c_h_a_m_p_s.  Les
              champs sont s\'epar\'es, par d\'efaut, par une Tabulation.

       _-_d_, _-_-_d_e_l_i_m_i_t_e_r _s_\'e_p_a_r_a_t_e_u_r
              Avec  -f,  les champs sont d\'elimit\'es par le premier caract\`ere du
              _s_\'e_p_a_r_a_t_e_u_r \`a la place de la Tabulation.

       _-_n     Ne pas couper les caract\`eres multi-octets (sans  effet  pour  le
              moment).

       _-_s_, _-_-_o_n_l_y_-_d_e_l_i_m_i_t_e_d
              Avec  -f,  ne  pas afficher les lignes qui ne contiennent pas le
              caract\`ere s\'eparateur de champs.

       _-_-_h_e_l_p Afficher un message d'aide sur la sortie  standard  et  terminer
              normalement.

       _-_-_v_e_r_s_i_o_n
              Afficher un num\'ero de version sur la sortie standard et terminer
              normalement.
\end{verbatim}
\newpage
\begin{verbatim}
TR(1)                    Manuel de l utilisateur Linux                   TR(1)

NOM     tr - Transposer ou \'eliminer des caract\`eres.

SYNOPSIS  tr [-cst] [--complement] [--squeeze-repeats] [--truncate-set1] cha\^ine_1
       cha\^ine_2
       tr {-s,--squeeze-repeats} [-c] [--complement] cha\^ine_1
       tr {-d,--delete} [-c] cha\^ine_1
       tr {-d,--delete} {-s,--squeeze-repeats}  [-c]  [--complement]  cha\^ine_1
       cha\^ine_2
DESCRIPTION
       tr copie son entr\'ee standard sur sa sortie standard en effectuant l'une
       des manipulations suivantes :

              �  transposer, et \'eventuellement r\'eunir les caract\`eres dupliqu\'es
              de la cha\^ine r\'esultante
              � r\'eunir les caract\`eres dupliqu\'es
              � supprimer des caract\`eres
              � supprimer des caract\`eres, et \'eventuellement  r\'eunir  les  car-
              act\`eres dupliqu\'es de la cha\^ine r\'esultante

       Les  arguments _c_h_a_\^i_n_e_1 et (\'eventuellement) _c_h_a_\^i_n_e_2 d\'ecrivent des ensem-
       bles ordonn\'es de caract\`eres, que l'on mentionnera  plus  bas  sous  les
       noms  de  jeu1  et  jeu2.  Ces ensembles repr\'esentent les caract\`eres de
       l'entr\'ee standard sur lesquels tr travaillera.   L'option  _-_-_c_o_m_p_l_e_m_e_n_t
       (_-_c)  remplace  jeu1  par  son compl\'ement (tous les caract\`eres n'appar-
       tenant pas \`a jeu1).

   D\'EFINIR LES ENSEMBLES DE CARACT\`ERES
       Le format des arguments  _c_h_a_\^i_n_e_1  et  _c_h_a_\^i_n_e_2  ressemble  \`a  celui  des
       expressions  rationnelles.  Il  ne  s'agit  toutefois pas d'expressions
       rationnelles, mais simplement de listes de caract\`eres.  La plupart  des
       caract\`eres  sont  repr\'esentes par eux-m\^emes, n\'eanmoins les cha\^ines peu-
       vent \'egalement contenir des raccourcis plus simples d\'ecrits ci-dessous.
       Certains de ces raccourcis ne peuvent \^etre utilis\'es que dans _c_h_a_\^i_n_e_1 ou
       que dans _c_h_a_\^i_n_e_2, comme c'est mentionn\'e ci-dessous.

       Intervalles.  La notation `_m-_n' repr\'esente tous les caract\`eres  compris
       entre _m et _n, en ordre croissant.  _m doit \^etre inf\'erieur \`a _n, sinon une
       erreur se produit.  Par exemple, `0-9' est \'equivalent  \`a  `0123456789'.
 
       R\'ep\'etition de caract\`eres. La notation `[_c*_n]' dans _c_h_a_\^i_n_e_2 se d\'eveloppe
       en  _n  copies du caract\`ere _c. Ainsi, `[y*6]' est \'equivalent \`a `yyyyyy'.
       La notation `[_c*]' dans _c_h_a_\^i_n_e_2 se d\'eveloppe en autant de  copie  de  _c
       qu'il  le  faut  pour rendre le jeu2 aussi long que jeu1. Si _n commence
       par 0, il est interpr\'et\'e en octal, sinon en d\'ecimal.

   TRADUCTION
       tr effectue les traductions de caract\`eres lorsqu'on lui  fournit  \`a  la
       fois  _c_h_a_\^i_n_e_1  et  _c_h_a_\^i_n_e_2,  et si l'on n'utilise pas l'option --delete
       (_-_d).  tr transpose chaque caract\`ere d'entr\'ee appartenant au jeu1 en un
       caract\`ere  correspondant  du jeu2. Les caract\`eres non trouv\'es dans jeu1
       sont copies sans modification. Quand un  caract\`ere  appara\^it  plusieurs
       fois dans jeu1, et si les caract\`eres correspondants de jeu2 ne sont pas
       toujours les m\^emes, seule la version finale est utilis\'ee.  Par  exemple
       ces deux commandes sont \'equivalentes :
              tr aaa xyz
              tr a z

   \'ELIMINER LES R\'EP\'ETITIONS ET EFFACER DES CARACT\`ERES
       Quand  seule  l'option  --delete (_-_d) est fournie, tr supprime tous les
       caract\`eres d'entr\'ee pr\'esents dans jeu1.

       Quand seule l'option --squeeze-repeats (_-_s) est  fournie,  tr  remplace
       chaque  r\'ep\'etitions  de  caract\`eres  appartenant  au jeu1 par une seule
       occurrence de ce caract\`ere.
\end{verbatim}
\newpage
\begin{verbatim}
SORT(1)                  Manuel de l utilisateur Linux                 SORT(1)
NOM   sort - Trier les lignes d'un fichier texte.

SYNOPSIS sort [-cmus] [-t s\'eparateur] [-o fichier_de_sortie] [-T r\'epertoire_tem-
       poraire] [-bdfiMnr] [+POS1 [-POS2]] [-k POS1[,POS2]] [fichier...]
       sort {--help,--version}
DESCRIPTION
       sort trie, regroupe ou compare toutes les lignes des fichiers indiqu\'es.
       Si  aucun fichier n'est fourni, ou si le nom `-' est mentionn\'e, la lec-
       ture se fera depuis l'entr\'ee standard.

       Par d\'efaut, sort \'ecrit ses r\'esultats sur la sortie standard.

       sort peut op\'erer suivant trois modes : tri (par d\'efaut),  regroupement,
       et  v\'erification  de  l'ordre.  Les options suivantes modifient le mode
       op\'eratoire :

       _-_c     ([NDT] c = check - v\'erifier) V\'erifie  si  les  fichiers  fournis
              sont  d\'ej\`a  tri\'es  :  s'ils  ne le sont pas, afficher un message
              d'erreur, et terminer avec un code de retour valant 1.

       _-_m     ([NDT] m = merge - melanger) Regrouper les fichiers indiqu\'es  en
              les  triant.  Chaque  fichier d'entr\'ee doit d\'ej\`a \^etre tri\'e indi-
              viduellement. Il est toujours possible de trier  plut\^ot  que  de
              r\'eunir,  le  regroupement est fourni parce qu'il est plus rapide
              dans les cas o� il fonctionne.

       La comparaison de deux lignes se fait ainsi : Si un  champ  cl\'e  a  \'et\'e
       indiqu\'e,  sort compare chaque paire de champs, dans l'ordre pr\'ecis\'e sur
       la ligne de commande, jusqu'\`a ce qu'une  diff\'erence  soit  trouv\'ee,  ou
       qu'il ne reste plus de champs.

       Finalement,  si  toutes  les  cl\'es sont \'egales, en dernier ressort sort
       compare les lignes octet  par  octet  suivant  l'ordre  d\'efini  sur  la
       machine.  Cette  derni\`ere  comparaison  accepte  l'option  globale  _-_r.
       L'option _-_s (stable) inhibe cette comparaison en dernier  recours  afin
       que les lignes consid\'er\'ees comme \'egales restent \`a leurs positions rela-
       tives.  Si aucun champ cl\'e, et aucune option ne sont  fournis,  _-_s  est
       sans effet.

       Les options suivantes affectent l'ordre des  lignes  de  sortie.  Elles
       peuvent  \^etre  mentionn\'ees  globalement,  ou  appliqu\'ees \`a un champ cl\'e
       sp\'ecifique. Si aucun champ cl\'e  n'est  indiqu\'e,  les  options  globales
       s'appliquent  aux  comparaisons  des  lignes enti\`eres, sinon elles sont
       transmises aux champs cl\'es n'ayant pas d'option sp\'ecifique.

       _-_b     Ignorer les blancs en d\'ebut de ligne pendant la recherche de  la
              cl\'e de tri sur chaque ligne.
       _-_d     Trier  dans l'ordre des r\'epertoires t\'el\'ephoniques : ignorer pour
              le tri tous les caract\`eres autres que les lettres, les  chiffres
              et les blancs.
       _-_f     Consid\'erer  les  minuscules  comme  leur \'equivalent en majuscule
              pendant le tri. Ainsi `b' est tri\'e de mani\`ere \'equivalente a `B'.
        _-_i     Ignorer pour le tri les caract\`eres  en  dehors  de  l'intervalle
              ASCI octal 040-0176 (bornes comprises).
       _-_M     Une  cha\^ine  initiale,  consistant  en  un  nombre quelconque de
              blancs, suivi de trois lettres correspondant \`a  une  abr\'eviation
              de  mois  est convertie en majuscules avant d'\^etre compar\'ee dans
              l'ordre `JAN' < `FEB' < ... < `DEC.'  Les  noms  invalides  sont
              consid\'er\'es comme inf\'erieurs aux noms valides.  ([NDT] Qu'en-est-
              il vis-\`a-vis de la localisation ?)
       _-_n     Comparer suivant la valeur arithm\'etique d'une  cha\^ine  num\'erique
              initiale compos\'ee d'espaces \'eventuelles, suivies optionnellement
              du signe -, et de z\'ero  ou  plusieurs  chiffres,  \'eventuellement
              suivis d'un point d\'ecimal et de z\'ero ou plusieurs chiffres.
       _-_r     Inverser l'ordre de tri, afin que les lignes avec la plus grande
              valeur de cl\'e apparaissent en premier.
\end{verbatim}
\end{document}
