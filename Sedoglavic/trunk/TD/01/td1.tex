%\documentclass[licence,solution]{controle}
\documentclass[licence]{controle}
\matiere{Pratique du~C}
\genre{Fiche de TD 1}
\commentaire{Interpr\'eteur de commandes}

\usepackage{epsfig}

\newcommand{\B}[1]{\overline{#1}}

%%%%%%%%%%%%%%%%%%%%%%%%%%%%%%%%%%%%%%%%%%%%%%%%%%%%%%%%%%%%%%%%%%%%%%%%%%%%%%
% graphismes
%

%\pssilent

\newlength{\longDSfig}

\def\normalDSfig{0.7}
\def\petiteDSfig{0.5}
\def\tinyDSfig{0.42}
\def\petiteXpixDSfig{0.9}
\def\smallDSfig{0.4}

\newcommand{\DSfig}[2]
{ \settowidth{\longDSfig}{\epsfig{figure=#1}}%
  \setlength{\longDSfig}{#2\longDSfig}%
  \ifthenelse{\lengthtest{\longDSfig > \linewidth}}%
    {\setlength{\longDSfig}{\linewidth}}%
    {\setlength{\longDSfig}{\longDSfig}}%
  \epsfig{figure=#1,width=\longDSfig}}

% the end


\begin{document}

%%%%%%%%%%%%%%%%%%%%%%%%%%%%%%%%%%%%%%%%%%%%%%%%%%%%%%%%%%%%%%%%%%%%%%
%%%%%%%%%%%%%%%%%%%%%%%%%%%%%%%%%%%%%%%%%%%%%%%%%%%%%%%%%%%%%%%%%%%%%%

\section{Expressions r\'eguli\`eres et m\'eta-caract\`eres}
\begin{question}
Utilisez la commande \texttt{ls} pour r\'ealiser l'exercice suivant.
\begin{enumerate} \item Lister les entr\'ees du r\'epertoire \texttt{/usr/bin} dont le
  nom commence par la lettre \texttt{m}. 
\item Lister les entr\'ees du r\'epertoire \texttt{/usr/bin} dont le
  nom commence par la lettre \texttt{m} et comporte exactement 3
  caract\`eres.  
\item Lister les entr\'ees du r\'epertoire \texttt{/usr/bin} dont le
  nom commence par la lettre \texttt{m} et comporte au moins 3
  caract\`eres.  
\item Lister les entr\'ees du r\'epertoire \texttt{/usr/bin} dont le
  nom commence par la lettre \texttt{m} et comporte une extension
  (suffixe suivant un \texttt{.}) non vide. 
\item Lister les entr\'ees du r\'epertoire \texttt{/usr/bin} dont le
  nom commence par la lettre \texttt{m} ou \texttt{j}. 
\item Lister les entr\'ees du r\'epertoire \texttt{/usr/bin} dont le
  nom commence par la lettre \texttt{m} et comporte une lettre
  majuscule. 
\item Lister les noms de fichiers  comportant une \'etoile \texttt{*}
\item Lister les entr\'ees du r\'epertoire \texttt{/usr/bin} dont le nom
  commence par la lettre \texttt{m} et ne comporte pas de lettre
  majuscule.
\end{enumerate}
\end{question}
\begin{solution}

\end{solution}
Il n'est pas possible de traiter la derni\`ere question avec la
commande \texttt{ls} et les m\'eta-caract\`eres. Il faudrait donc soit
pouvoir utiliser une autre commande, soit prendre le r\'esultat de 
\texttt{ls} pour finir de le traiter avec une autre commande.
\section{Les redirections}
\begin{question}
Dans cet exercice, nous allons utiliser les redirections coupl\'ees \`a
d'autres commandes : \texttt{find}, \texttt{grep}, \texttt{cut},
\texttt{tr} et \texttt{sort}.
\begin{enumerate}
%%Interet le pipe
\item Listez les entr\'ees du r\'epertoire \texttt{/usr/bin} dont le nom
  commence par la lettre \texttt{m} et comporte au moins une lettre
  majuscule ! (Avec la commande \texttt{grep}).
%%Interet grep -v et tjs le pipe
\item En vous servant de la question pr\'ec\'edente, r\'ealisez la derni\`ere
  question de l'exercice 1.
\end{enumerate}
\end{question}

\begin{question}
La commande \texttt{ls -l} permet d'afficher l'ensemble des informations
sur les fichiers d'un r\'epertoire.
\begin{verbatim}
total 104K
-rw-r--r--  1 dumont  west  234 Sep 28 11:33 td1.aux
-rw-r--r--  1 dumont  west 6.2K Sep 28 11:33 td1.dvi
-rw-r--r--  1 dumont  west 7.1K Sep 28 11:33 td1.log
-rw-r--r--  1 dumont  west 3.7K Sep 28 12:01 td1.tex
\end{verbatim}
\begin{enumerate}
%%Interet pipe + cut
\item Listez uniquement le nom des propri\'etaires des fichiers se
  trouvant dans un r\'epertoire (on se servira de l'exemple ci-dessus).
%%Interet pipe + tr + cut + les pieges a la con avec des espaces
\item M�me question mais en listant cette fois-ci, le mois de la
  derni\`ere modification de ces fichiers. Attention il y a un pi\`ege !
%%Interet les redirections et les probl\`emes avec les redirections
%%   (effacement du fichier)
\item Sauvegardez le resultat de la derni\`ere commande dans un fichier
  et tri\'e le nom des propri\'etaires dans l'ordre alphabetique et
  sauvegardez le r\'esultat \`a nouveau dans ce m�me fichier.
\end{enumerate}
\end{question}

\begin{question}
  Nous allons illustrer la commande \texttt{find} ainsi que certains
  points du cours.
\begin{enumerate}
\item Que fait la  commande \verb?ps aux | grep $USER?
\item Que fait la  commande \verb?cat /etc/passwd | cut -d: -f1 | sort > ~/res?
%%Interet find et la substitution des meta caracteres
\item Lister tous les fichiers de votre r\'epertoire personnel et de ces
  sous r\'epertoires qui commence par la lettre \texttt{m}. Attention \`a la
  substitution des m\'eta-caract\`eres.
\item Que fait la  commande \verb?find . -name tmp -type d ! -empty?
\item Que fait la  commande \verb?find /usr/include -name ``*a*b*'' -exec ls -l?
\end{enumerate}
\end{question}
\end{document}
%%%%%%%%%%%%%%%%%%%%% PART 3
\begin{itemize}
\item Le shell ne conna\^\i{}t que les cha\^\i{}nes de caract\`eres ; 
\item Les valeurs enti\`eres sont stock\'ees comme des cha\^\i{}nes de
  caract\`eres ; 
\item Affectation : \texttt{nom-var=<chaine>} La variable \texttt{nom-var}
  re�oit la valeur de la cha\^\i{}ne de caract\`eres ; 
\item Attention, pas d'espace s\'eparateur autour du \texttt{=}
\item La valeur de la variable de nom \texttt{nom-var} est d\'esign\'ee
  par \texttt{nom-var}. L'interpr\'eteur/le shell substitue \`a chaque
  occurrence de \texttt{nom-var} la valeur (cha\^\i{}ne de caract\`eres)
  de la variable identifi\'ee par \texttt{nom-var}.
\item Exemples : 
\begin{verbatim}
% a="hello"
% b="Unix"
% c=a 
% d=b 
% echo $a $b $c $d
hello Unix a b       

% a=pwd
% b=`$a`
% echo $a $b
pwd /home/phm 
 
% a=`ls`
% echo $a
TP1 sujet.ps sujet.tex  

% `ls`
sh: TP1: command not found     

% date
Sun Sep 27 00:04:23 CEST 1998      
% `date`
sh: Sun: command not found
\end{verbatim}
\end{itemize}






