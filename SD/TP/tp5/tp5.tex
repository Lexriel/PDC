% xelatex tp1.tex

\documentclass[12pt]{article}
\usepackage{xunicode}
\usepackage{fontspec}
\usepackage{xltxtra}
\usepackage{url}
\usepackage[francais]{babel}
\usepackage{fullpage}
\usepackage{multicol}


% \usepackage{epsf}
% \usepackage{amsmath}
% \usepackage{amsfonts}
% \usepackage{amssymb}	% pour \varnothing
% \usepackage[latin1]{inputenc}
% \usepackage[T1]{fontenc}
% \usepackage{times}
% \usepackage{epic,eepic}
% \usepackage{mathrsfs}   % pour la commande \mathscr
% \usepackage{moreverb}   % pour verbatiminput
% \usepackage{ifthen}     % pour les versions enseignant/étudiant
% \usepackage{graphics}
% \usepackage{epsfig}
% \usepackage{multicol}


%% Environnements

\newtheorem{notation}{Notation}
\newtheorem{abus}{Abus de langage}
\newtheorem{corollary}{Corollaire}
\newtheorem{theorem}{Théorème}
\newtheorem{lemma}{Lemme}
\newtheorem{proposition}{Proposition}
\newtheorem{convention}{Convention}
\newtheorem{definition}{Définition}
\newcommand{\Algf}{\bfseries \sffamily }
\newcommand{\algf}{\sffamily }
\newenvironment{proof}{{\bf Preuve} }{\ensuremath{\Box}\normalsize}
\newcommand{\marginlabel}[1]{\mbox{}\marginpar{\raggedleft\hspace{0pt}#1}}

%% Algorithmes

\newcommand{\BEGIN}{{\algf begin}}
\newcommand{\END}{{\algf end}}
\newcommand{\IF}{{\algf if}}
\newcommand{\THEN}{{\algf then}}
\newcommand{\ELSE}{{\algf else}}
\newcommand{\ELIF}{{\algf elif}}
\newcommand{\FI}{{\algf fi}}
\newcommand{\WHILE}{{\algf while}}
\newcommand{\FOR}{{\algf for}}
\newcommand{\DO}{{\algf do}}
\newcommand{\OD}{{\algf end do}}
\newcommand{\RETURN}{{\algf return}}
\newcommand{\PROCEDURE}{{\algf procedure}}
\newcommand{\FUNCTION}{{\algf function}}
\newcommand{\INDENTER}{{\algf si} \=\+\kill}

%% Opérateurs

\newcommand{\initial}{\mathop{\mathsf{init}}}
\newcommand{\separant}{\mathop{\mathsf{sep}}}
\newcommand{\rem}{\mathop{\mathsf{rem}}}
\newcommand{\quo}{\mathop{\mathsf{quo}}}
\newcommand{\lcoeff}{\mathop{\mathsf{lcoeff}}}
\newcommand{\mvar}{\mathop{\mathsf{mvar}}}
\newcommand{\ld}{\mathop{\mathrm{ld}}}
\newcommand{\rank}{\mathrm{\mathsf{rank}}}
\newcommand{\prem}{\mathop{\mathsf{prem}}}
\newcommand{\remp}{\mathrel{\mathsf{partial\_rem}}}
\newcommand{\remf}{\mathrel{\mathsf{full\_rem}}}
\renewcommand{\gcd}{\mathop{\mathrm{gcd}}}
\renewcommand{\deg}{\mathop{\mathrm{deg}}}
\newcommand{\ord}{\mathop{\mathrm{ord}}}
\newcommand{\pairs}{\mathop{\mathrm{pairs}}}
\newcommand{\dd}{\mathrm{d}}
\newcommand{\ideal}[1]{(#1)}
\newcommand{\cont}{\mathop{\mathrm{contenu}}}
\newcommand{\pp}{\mathop{\mathrm{partie\_primitive}}}
\newcommand{\init}{\mathop{\mathrm{coefficient\_dominant}}}
\newcommand{\pgcd}{\mathop{\mathrm{pgcd}}}
\newcommand{\ppmc}{\mathop{\mathrm{ppcm}}}
\newcommand{\NF}{\mathop{\mathrm{NF}}}
\newcommand{\rang}{\mathop{\mathrm{rang}}}
\newcommand{\card}{\mathop{\mbox{card}}}

%% Ensembles mathématiques

\newcommand{\A}{\mathbb{A}}
\newcommand{\Q}{\mathbb{Q}}
\newcommand{\Z}{\mathbb{Z}}
\newcommand{\C}{\mathbb{C}}
\newcommand{\N}{\mathbb{N}}
\newcommand{\R}{\mathbb{R}}
\renewcommand{\emptyset}{\varnothing}

%% Réécriture

\newcommand{\egaldef}{\mathop{\underset{\mathrm{def}}{=}}}
\newcommand{\fleche}{\hbox to 1cm {\rightarrowfill}}
\newcommand{\chefle}{\hbox to 1cm {\leftarrowfill}}
\newcommand{\mafleche}{\mathrel{\fleche}}
\newcommand{\machefle}{\mathrel{\chefle}}
\def\build#1_#2^#3{\mathrel{\mathop{\kern 0pt#1}\limits_{#2}^{#3}}}

%% Saturations

\newbox\deuxpoints
\setbox\deuxpoints=\hbox{:}
\def\dv{\mathbin{\raisebox{1pt}{\copy\deuxpoints}}}
\newbox\infini
\setbox\infini=\hbox{\footnotesize $\infty$}
\def\sat#1{#1^{\copy\infini}}
\def\H#1{H_{\!{#1}}}
\def\S#1{S_{\!{#1}}}
\def\HH#1{\sat{\H{#1}}}
\def\SS#1{\sat{\S{#1}}}

%% \question -> numérote les question
%% \Question{k} -> indique en plus que la question vaut k points
%% \Section{intitulé} -> rajoute à l'intitulé la somme des points des 
%%    questions qui figurent dans la section (faire deux compilations).

\newcounter{question_counter}
\setcounter{question_counter}{0}
\newcommand{\question}{\addtocounter{question_counter}1 %
    \paragraph{\bf Question \arabic{question_counter}.}}
\newcommand{\questiondiff}{\addtocounter{question_counter}1 %
    \paragraph{\bf Question \arabic{question_counter} (plus difficile).}}

\newcounter{points_counter}
\newcounter{section_points_counter}
\newcommand{\Question}[1]{\addtocounter{question_counter}1 %
    \addtocounter{points_counter}{#1} %
    \addtocounter{section_points_counter}{#1} %
    \typeout{Question \arabic{question_counter}. Total des points \arabic{points_counter}} %
    \typeout{points_counter_\Alph{section} \arabic{section_points_counter}} %
    \paragraph{\bf Question \arabic{question_counter}}{[#1\,\ifnum #1 = 1 pt\else pts\fi]}.}

\newcommand{\Section}[1]{\setcounter{section_points_counter}0 %
    \section{{#1} (\ref{points_counter_\Alph{section}} points)}}

%% \solution{...}
%% \progsol{fichier_source}

% \newboolean{corrige}
% \setboolean{corrige}{false}
% \newcommand{\solution}[1]{%
% \ifthenelse{\boolean{corrige}}%
% {\small\begin{quote}{\sc Solution. }#1\end{quote}\normalsize}{}}
% \newcommand{\progsol}[1]{%
% \ifthenelse{\boolean{corrige}}%
% {\small\begin{quote}{\sc Solution. }\footnotesize\verbatiminput{#1}\end{quote}\normalsize}{}}

%% Fait en sorte que la font verbatim soit \footnotesize

\makeatletter %
\renewcommand{\verbatim@font}{ %
    \ttfamily\footnotesize\catcode`\<=\active\catcode`\>=\active %
}
\makeatother

\begin{document}
\begin{center}
{\bf Polytech'Lille --- GIS 3 --- Structures de Données --- Feuille de TP numéro 5}
 
\end{center}
\medskip
\hrule
 
\bigskip

Ce TP se déroule sur deux séances (quatre heures).
Les étudiants qui se sentent en difficulté devraient réaliser 
la section~\ref{sec1}, uniquement. Les autres peuvent réaliser 
la section~\ref{sec2}, après s'être assurés qu'ils savent effectivement
faire la première partie. 

\section{Implantation d'un module d'ABR}\label{sec1}

On veut réaliser un module dédié aux ABR, qui permette
de faire fonctionner le programme principal {\tt testABR.c} suivant.
Les valeurs sont des entiers. 
Vous pouvez vous inspirer de l'extrait de code disponible sur le
site du cours mais il serait préférable que vous refassiez tout vous-même.

\begin{verbatim}
#include "ABR.h"
#include <stdio.h>

int main ()
{   struct ABR* racine;
    int x;
    racine = NIL;
    scanf ("%d", &x);
    while (x != -1)
    {   racine = ajouter_ABR (x, racine);
        afficher_ABR (racine);
        scanf ("%d", &x);
    }
    printf ("la hauteur de l'ABR est %d\n", hauteur_ABR (racine));
    printf ("le nombre de noeuds de l'ABR est %d\n", 
                                        nombre_noeuds_ABR (racine));
    clear_ABR (racine);
    return 0;
}
\end{verbatim}

\question
Combien de fichiers doit-on écrire~?

\question
On souhaite compiler séparément ce qui peut l'être.
Quelles seront les commandes de compilation nécessaires~?

\question
Écrire le fichier d'entête.
Spécifier la structure de données, dans un commentaire, placé dans le fichier.

\question
Écrire le fichier source. Lors des essais, commentez, dans le programme
principal, les appels aux fonctions que vous n'avez pas encore réalisées.

\question
Écrire une fonction qui permette de visualiser un ABR avec {\tt dot} (voir
feuille de TD). Gérer le cas de l'arbre vide et celui de l'arbre réduit
à une feuille. Pour les tests, sortir l'affichage de la boucle du
programme principal.

\section{Expérimentations}

\question Modifier {\tt testABR.c} pour qu'il affiche, à chaque nouvel 
entier enregistré dans l'ABR, le nombre~$f(n)$ cumulé de comparaisons 
effectuées, où~$n$ désigne le nombre de n{\oe}uds de l'arbre. 
Afin de faire des analyses, afficher les résultats sur deux colonnes~:~$n$
et~$f(n)$.

\bigskip

On souhaite écrire un programme {\tt gentest.c} qui imprime sur
sa sortie standard les entiers de~$1$ à~$16384$, terminés par~$-1$,
suivant différents ordres. L'idée consiste ensuite à rediriger la
sortie de {\tt gentest.c} sur l'entrée de {\tt testABR.c}, 
afin de générer des ABR de différentes formes. Des exemples sont
disponibles sur le site du cours, dans l'archive {\tt exemples-arbres.tar}.

\question
Programmer {\tt gentest.c} pour qu'il fabrique un ABR filiforme.
Rediriger sortie de {\tt gentest.c} sur l'entrée de {\tt testABR.c}.
En utilisant vos connaissances et la fonction {\tt fit} de GNUPLOT,
déterminer une approximation de~$f(n)$.

\question
Programmer {\tt gentest.c} pour qu'il fabrique un ABR équilibré en
nombre de n{\oe}uds. Chercher une solution récursive.

\question
En utilisant vos connaissances et la fonction {\tt fit} de GNUPLOT,
déterminer une approximation de~$f(n)$. Attention~: {\tt gentest.c} 
doit énumérer les entiers de façon à ce que l'ABR soit en permanence
équilibré en nombre de n{\oe}uds. Si vous ne trouvez pas de solution
récursive, utilisez le code suivant~:
\begin{verbatim}
static void equilibre (int n)  /* n = le nombre d'entiers à imprimer */
{   int w, b, i, k;
    b = n;
    w = 1;
    for (k = n; k >= 1; k /= 2)
    {   for (i = 0; i < w; i++)
            printf ("%d\n", b*(2*i+1));
        b /= 2;
        w *= 2;
    }
}
\end{verbatim}

\question
Programmer {\tt gentest.c} pour qu'il imprime des entiers aléatoires
distincts deux-à-deux. Utiliser {\tt srand48} et {\tt drand48} pour
la génération de nombres aléatoires. Voici une suggestion pour
initialiser le générateur aléatoire~:
\begin{verbatim}
#include <time.h>
...
    srand48 (time (0));
\end{verbatim}

Un résultat théorique affirme que la hauteur moyenne d'un ABR
après~$n$ insertions d'éléments aléatoires, se comporte asymptotiquement
comme~$c \, \ln(n)$, où~$c$ vaut approximativement $4.311$
\cite[Theorem 5.10, page 261]{SF96}.
Ce résultat se vérifie-t-il sur vos expérimentations~?

\section{Les codages de Huffman}\label{sec2}

Le TP porte sur les codages de Huffman, qui constituent une 
technique de compression de données très utilisée.
Dans ce TP, la donnée à compresser est un texte.
Du point de vue des structures de données, ce TP nous amène à
utiliser des arbres binaires et des files de priorité.

L'idée consiste à coder chaque caractère~$c$ d'un texte 
par une suite de bits, qui dépend du nombre d'occurences de~$c$ dans le
texte. Plus le caractère est fréquent, plus la suite de bits est courte.
Prenons pour exemple, le texte~: «~exemple de codage de Huffman~», formé
de~$29$ caractères. Le codage de Huffman qui lui est associé est présenté
sous la forme d'un arbre binaire, Figure~\ref{fig-ex}.

\begin{figure}[h!t]
\centerline{\includegraphics[height=8cm]{exemple.pdf}}
\caption{\footnotesize Arbre binaire représentant un codage de Huffman. 
Chaque feuille est étiquetée par un caractère et son nombre d'occurences.
L'espace et le retour chariot sont représentés par leur code ASCII en
hexadécimal. Chaque nœud intérieur est étiqueté par le nombre d'occurences
total des feuilles du sous-arbre dont il est la racine. Les arcs vers les
fils gauches sont étiquetés~$0$~; les arcs vers les fils droits sont 
étiquetés~$1$.}
\label{fig-ex}
\end{figure}

Pour obtenir la suite de bits qui code un caractère~$c$, il suffit de 
suivre le chemin qui part de la racine vers la feuille~$c$ et d'écrire
les étiquettes des arcs suivis. Par exemple, le caractère «~{\tt e}~»,
qui apparaît~$6$ fois dans le texte, est codé par la suite de deux 
bits «~{\tt 00}~». Le caractère «~{\tt H}~» qui n'apparaît qu'une fois,
est codé par la suite de cinq bits «~{\tt 10001}~». Le texte complet
est codé par la concaténation des codages des caractères. Il
commence donc par «~{\tt 001000000}~», c'est-à-dire «~{\tt exe}~».
Au total, la chaîne se code sur~$107$ bits ($14$ octets).
C'est deux fois plus court que les~$7 \times 29 = 203$ bits utilisés
par le codage ASCII. 
Ce codage a de nombreuses propriétés, très intéressantes.
Voir \cite[chapitre 16.3]{CLRS02}.

\subsection*{Construction de l'arbre}

On se donne une file de priorité~$F$ d'arbres de Huffman.
Un arbre de Huffman~$H_1$ est plus prioritaire qu'un arbre de Huffman~$H_2$
si le nombre d'occurences qui étiquette la racine de~$H_1$ est {\em inférieur}
à celui de~$H_2$.
On lit le texte, caractère par caractère.
Pour chaque caractère~$c$, deux cas de figure se présentent~:
si~$c$ est lu pour la première fois, on crée un nouvel arbre de Huffman
(une feuille) étiquetée par~$c$ et le nombre d'occurences~$1$, qu'on
enfile dans~$F$~; si~$c$ a déjà été lu, on incrémente le nombre
d'occurences de la feuille qui lui correspond (le caractère est
forcément présent dans la file~$F$, sous la forme d'une feuille)
et on restructure la file, puisque la priorité de~$c$ a baissé.

À la fin de la lecture du texte, on a donc une file de priorité, ne
comportant que des feuilles. Pour former l'arbre~$H$, il suffit alors
d'appliquer l'algorithme de la Figure~\ref{fig-code}. 
Une trace d'exécution est donnée
Figure~\ref{fig-fx}.
\begin{figure}[h!t]
\begin{tabbing}
\WHILE\ la file $F$ contient deux arbres ou plus \DO \\
\INDENTER
    $G$ := défiler ($F$) \\
    $D$ := défiler ($F$) \\
    $N$ := un nouveau nœud avec fils gauche $G$, fils droit $D$ (opération de {\em fusion}) \\
    le nombre d'occurences qui étiquette $N$ doit être égal à la 
	somme des \\ \qquad nombres d'occurences qui étiquettent $G$ et $D$ \\
    enfiler $N$ dans $F$ \-\\
\OD \\
$H$ := défiler ($F$)
\end{tabbing}
\caption{\footnotesize Algorithme de construction de l'arbre de Huffman. Initialement, la file de priorité contient les feuilles correspondant aux caractères lus. À la fin, la file ne contient qu'un arbre~: l'arbre de Huffman.}
\label{fig-code}
\end{figure}

\begin{figure}[h!t]
\begin{tabular}{llll}
(a) & \includegraphics[width=.35\textwidth]{stepa.pdf} &
(b) & \includegraphics[width=.35\textwidth]{stepb.pdf} \\
(c) & \includegraphics[width=.35\textwidth]{stepc.pdf} &
(d) & \includegraphics[width=.35\textwidth]{stepd.pdf} \\
(e) & \includegraphics[width=.35\textwidth]{stepe.pdf} &
(f) & \includegraphics[width=.35\textwidth]{stepf.pdf} 
\end{tabular}
\caption{\footnotesize États successifs de la file de priorité, lors de l'exécution de l'algorithme de la Figure~\ref{fig-code}, sur un exemple. Les éléments de la file sont des arbres. Initialement, la file contient~$6$ feuilles correspondant aux caractères lus. À la fin, la file ne contient qu'un arbre~: l'arbre de Huffman de l'exemple.}
\label{fig-fx}
\end{figure}

\subsection*{Travail à faire}

\question
Écrire un algorithme qui construise l'arbre de Huffman d'un texte
et qui imprime le nombre de bits nécessaire au codage de Huffman de ce texte.
Pour cela, mettre au point deux structures de données~: une
pour les arbres de Huffman et une pour les files de priorité.

Le programme devrait se répartir sur cinq fichiers~: deux fichiers
d'entête et deux fichiers source correspondant aux deux structures,
ainsi qu'un programme principal.

Bien spécifier la structure d'arbre de Huffman.

Pour vérifier que l'arbre est correct, il peut être utile de fabriquer
un fichier ``.dot'' en s'inspirant de la Figure~\ref{fig-dot}.

\question
Votre programme fini, comparer le taux de compression que vous obtiendriez
avec celui de l'utilitaire {\tt gzip}. Que constatez-vous~? Expliquez
rapidement pourquoi en effectuant une recherche sur internet.

\question
On souhaite maintenant mettre au point une structure de données permettant
d'imprimer les suites de bits, correspondant au codage d'un texte, sur 
la sortie standard. On ne peut imprimer ces séquences de bits que par
paquets de huit, sous la forme d'un caractère. Quelle structure de
données vous paraît la plus appropriée~? Spécifiez-la.

\question
Déterminer le codage d'un caractère dans un arbre de Huffman n'est pas
complètement immédiat. Quelle solution proposez-vous~?

\makeatletter %
\renewcommand{\verbatim@font}{ %
    \ttfamily\tiny\catcode`\<=\active\catcode`\>=\active %
}
\makeatother

\begin{figure}[h!t]
\begin{multicols}{2}
\begin{verbatim}
digraph G {
    label_b965b0 [label="29"];
    label_b96550 [label="13"];
    label_b965b0 -> label_b96550 [label="0"];
    label_b96580 [label="16"];
    label_b965b0 -> label_b96580 [label="1"];
    label_65 [label="e:6"];
    label_b96550 -> label_65 [label="0"];
    label_b964c0 [label="7"];
    label_b96550 -> label_b964c0 [label="1"];
    label_65 [shape=box];
    label_64 [label="d:3"];
    label_b964c0 -> label_64 [label="0"];
    label_b96490 [label="4"];
    label_b964c0 -> label_b96490 [label="1"];
    label_64 [shape=box];
    label_b96340 [label="2"];
    label_b96490 -> label_b96340 [label="0"];
    label_b96370 [label="2"];
    label_b96490 -> label_b96370 [label="1"];
    label_6e [label="n:1"];
    label_b96340 -> label_6e [label="0"];
    label_75 [label="u:1"];
    label_b96340 -> label_75 [label="1"];
    label_6e [shape=box];
    label_75 [shape=box];
    label_70 [label="p:1"];
    label_b96370 -> label_70 [label="0"];
    label_6c [label="l:1"];
    label_b96370 -> label_6c [label="1"];
    label_70 [shape=box];
    label_6c [shape=box];
    label_b964f0 [label="8"];
    label_b96580 -> label_b964f0 [label="0"];
    label_b96520 [label="8"];
    label_b96580 -> label_b96520 [label="1"];
    label_b96400 [label="4"];
    label_b964f0 -> label_b96400 [label="0"];
    label_b96460 [label="4"];
    label_b964f0 -> label_b96460 [label="1"];
    label_b96310 [label="2"];
    label_b96400 -> label_b96310 [label="0"];
    label_b963d0 [label="2"];
    label_b96400 -> label_b963d0 [label="1"];
    label_78 [label="x:1"];
    label_b96310 -> label_78 [label="0"];
    label_48 [label="H:1"];
    label_b96310 -> label_48 [label="1"];
    label_78 [shape=box];
    label_48 [shape=box];
    label_6f [label="o:1"];
    label_b963d0 -> label_6f [label="0"];
    label_0a [label="0x0a:1"];
    label_b963d0 -> label_0a [label="1"];
    label_6f [shape=box];
    label_0a [shape=box];
    label_6d [label="m:2"];
    label_b96460 -> label_6d [label="0"];
    label_b963a0 [label="2"];
    label_b96460 -> label_b963a0 [label="1"];
    label_6d [shape=box];
    label_67 [label="g:1"];
    label_b963a0 -> label_67 [label="0"];
    label_63 [label="c:1"];
    label_b963a0 -> label_63 [label="1"];
    label_67 [shape=box];
    label_63 [shape=box];
    label_b96430 [label="4"];
    label_b96520 -> label_b96430 [label="0"];
    label_20 [label="0x20:4"];
    label_b96520 -> label_20 [label="1"];
    label_61 [label="a:2"];
    label_b96430 -> label_61 [label="0"];
    label_66 [label="f:2"];
    label_b96430 -> label_66 [label="1"];
    label_61 [shape=box];
    label_66 [shape=box];
    label_20 [shape=box];
}
\end{verbatim}
\end{multicols}
\caption{\footnotesize 
Le fichier ``.dot'' qui a servi à produire la Figure~\ref{fig-ex}.
Les identificateurs des feuilles ont été fabriqués à partir des codes
ASCII des caractères. Ceux des n{\oe}uds intermédiaires 
à partir des adresses des structures, écrites en hexadécimal.}
\label{fig-dot}
\end{figure}

\subsection*{Compression à la volée}

A priori, compresser un texte se fait en deux temps~: dans un premier
temps, on compte les occurences des caractères~; dans un deuxième temps,
on construit le codage et on compresse le texte.

Peut-on compresser un texte à la volée (sans le lire deux fois de suite)
avec le codage de Huffman~? Oui. On lit le texte caractère par caractère.
Pour chaque caractère~$c$, deux cas de figure se présentent~: si~$c$ est
lu pour la première fois, on l'imprime «~en clair~» sur la sortie standard,
puis on l'enfile dans~$F$~; si~$c$ a déjà été lu, on construit l'arbre
de Huffman correspondant à l'état courant de la file~$F$, on imprime
la séquence de bits correspondant à~$c$ dans cet arbre, puis on met à jour~$F$
en incrémentant le nombre d'occurences de~$c$ et en abaissant sa priorité.
À chaque caractère lu, un nouvel arbre de Huffman est créé.

Comment l'algorithme de décodage fait-il pour distinguer les séquences
de bits correspondant à un caractère écrit «~en clair~» des séquences
de bits du codage de Huffman~? Il suffit de se donner un caractère
spécial (appelons-le {\tt NYT}) avec un nombre d'occurences~$0$, 
qui n'appartient pas au texte\footnote{C'est facile en UTF-8.}, 
de l'enfiler dans~$F$ en début d'algorithme, et d'imprimer le codage 
de Huffman de {\tt NYT} juste avant les huit bits de~$c$.

Pour pouvoir décoder le texte, l'algorithme de décodage est obligé de
mimer le comportement de l'algorithme de codage et de faire évoluer, lui
aussi, l'arbre de Huffman du texte, à chaque fois qu'un caractère est lu.
Il peut ainsi repérer la séquence de bits codant {\tt NYT} et lire
les huit bits suivants, qui donnent un nouveau caractère
«~en clair~».

Une variante de cette méthode est connue sous le nom d'algorithme de Vitter
\cite{Vitter87}.

\question
Implantez cet algorithme.

\bibliographystyle{plain}
\bibliography{reecriture}
\end{document}


