% xelatex tp1.tex

\documentclass[12pt]{article}
\usepackage{xunicode}
\usepackage{fontspec}
\usepackage{xltxtra}
\usepackage{url}
\usepackage[francais]{babel}
\usepackage{fullpage}

% \usepackage{epsf}
% \usepackage{amsmath}
% \usepackage{amsfonts}
% \usepackage{amssymb}	% pour \varnothing
% \usepackage[latin1]{inputenc}
% \usepackage[T1]{fontenc}
% \usepackage{times}
% \usepackage{epic,eepic}
% \usepackage{mathrsfs}   % pour la commande \mathscr
% \usepackage{moreverb}   % pour verbatiminput
% \usepackage{ifthen}     % pour les versions enseignant/étudiant
% \usepackage{graphics}
% \usepackage{epsfig}
% \usepackage{multicol}


%% Environnements

\newtheorem{notation}{Notation}
\newtheorem{abus}{Abus de langage}
\newtheorem{corollary}{Corollaire}
\newtheorem{theorem}{Théorème}
\newtheorem{lemma}{Lemme}
\newtheorem{proposition}{Proposition}
\newtheorem{convention}{Convention}
\newtheorem{definition}{Définition}
\newcommand{\Algf}{\bfseries \sffamily }
\newcommand{\algf}{\sffamily }
\newenvironment{proof}{{\bf Preuve} }{\ensuremath{\Box}\normalsize}
\newcommand{\marginlabel}[1]{\mbox{}\marginpar{\raggedleft\hspace{0pt}#1}}

%% Algorithmes

\newcommand{\BEGIN}{{\algf begin}}
\newcommand{\END}{{\algf end}}
\newcommand{\IF}{{\algf if}}
\newcommand{\THEN}{{\algf then}}
\newcommand{\ELSE}{{\algf else}}
\newcommand{\ELIF}{{\algf elif}}
\newcommand{\FI}{{\algf fi}}
\newcommand{\WHILE}{{\algf while}}
\newcommand{\FOR}{{\algf for}}
\newcommand{\DO}{{\algf do}}
\newcommand{\OD}{{\algf od}}
\newcommand{\RETURN}{{\algf return}}
\newcommand{\PROCEDURE}{{\algf procedure}}
\newcommand{\FUNCTION}{{\algf function}}
\newcommand{\INDENTER}{{\algf si} \=\+\kill}

%% Opérateurs

\newcommand{\initial}{\mathop{\mathsf{init}}}
\newcommand{\separant}{\mathop{\mathsf{sep}}}
\newcommand{\rem}{\mathop{\mathsf{rem}}}
\newcommand{\quo}{\mathop{\mathsf{quo}}}
\newcommand{\lcoeff}{\mathop{\mathsf{lcoeff}}}
\newcommand{\mvar}{\mathop{\mathsf{mvar}}}
\newcommand{\ld}{\mathop{\mathrm{ld}}}
\newcommand{\rank}{\mathrm{\mathsf{rank}}}
\newcommand{\prem}{\mathop{\mathsf{prem}}}
\newcommand{\remp}{\mathrel{\mathsf{partial\_rem}}}
\newcommand{\remf}{\mathrel{\mathsf{full\_rem}}}
\renewcommand{\gcd}{\mathop{\mathrm{gcd}}}
\renewcommand{\deg}{\mathop{\mathrm{deg}}}
\newcommand{\ord}{\mathop{\mathrm{ord}}}
\newcommand{\pairs}{\mathop{\mathrm{pairs}}}
\newcommand{\dd}{\mathrm{d}}
\newcommand{\ideal}[1]{(#1)}
\newcommand{\cont}{\mathop{\mathrm{contenu}}}
\newcommand{\pp}{\mathop{\mathrm{partie\_primitive}}}
\newcommand{\init}{\mathop{\mathrm{coefficient\_dominant}}}
\newcommand{\pgcd}{\mathop{\mathrm{pgcd}}}
\newcommand{\ppmc}{\mathop{\mathrm{ppcm}}}
\newcommand{\NF}{\mathop{\mathrm{NF}}}
\newcommand{\rang}{\mathop{\mathrm{rang}}}
\newcommand{\card}{\mathop{\mbox{card}}}

%% Ensembles mathématiques

\newcommand{\A}{\mathbb{A}}
\newcommand{\Q}{\mathbb{Q}}
\newcommand{\Z}{\mathbb{Z}}
\newcommand{\C}{\mathbb{C}}
\newcommand{\N}{\mathbb{N}}
\newcommand{\R}{\mathbb{R}}
\renewcommand{\emptyset}{\varnothing}

%% Réécriture

\newcommand{\egaldef}{\mathop{\underset{\mathrm{def}}{=}}}
\newcommand{\fleche}{\hbox to 1cm {\rightarrowfill}}
\newcommand{\chefle}{\hbox to 1cm {\leftarrowfill}}
\newcommand{\mafleche}{\mathrel{\fleche}}
\newcommand{\machefle}{\mathrel{\chefle}}
\def\build#1_#2^#3{\mathrel{\mathop{\kern 0pt#1}\limits_{#2}^{#3}}}

%% Saturations

\newbox\deuxpoints
\setbox\deuxpoints=\hbox{:}
\def\dv{\mathbin{\raisebox{1pt}{\copy\deuxpoints}}}
\newbox\infini
\setbox\infini=\hbox{\footnotesize $\infty$}
\def\sat#1{#1^{\copy\infini}}
\def\H#1{H_{\!{#1}}}
\def\S#1{S_{\!{#1}}}
\def\HH#1{\sat{\H{#1}}}
\def\SS#1{\sat{\S{#1}}}

%% \question -> numérote les question
%% \Question{k} -> indique en plus que la question vaut k points
%% \Section{intitulé} -> rajoute à l'intitulé la somme des points des 
%%    questions qui figurent dans la section (faire deux compilations).

\newcounter{question_counter}
\setcounter{question_counter}{0}
\newcommand{\question}{\addtocounter{question_counter}1 %
    \paragraph{\bf Question \arabic{question_counter}.}}
\newcommand{\questiondiff}{\addtocounter{question_counter}1 %
    \paragraph{\bf Question \arabic{question_counter} (plus difficile).}}

\newcounter{points_counter}
\newcounter{section_points_counter}
\newcommand{\Question}[1]{\addtocounter{question_counter}1 %
    \addtocounter{points_counter}{#1} %
    \addtocounter{section_points_counter}{#1} %
    \typeout{Question \arabic{question_counter}. Total des points \arabic{points_counter}} %
    \typeout{points_counter_\Alph{section} \arabic{section_points_counter}} %
    \paragraph{\bf Question \arabic{question_counter}}{[#1\,\ifnum #1 = 1 pt\else pts\fi]}.}

\newcommand{\Section}[1]{\setcounter{section_points_counter}0 %
    \section{{#1} (\ref{points_counter_\Alph{section}} points)}}

%% \solution{...}
%% \progsol{fichier_source}

% \newboolean{corrige}
% \setboolean{corrige}{false}
% \newcommand{\solution}[1]{%
% \ifthenelse{\boolean{corrige}}%
% {\small\begin{quote}{\sc Solution. }#1\end{quote}\normalsize}{}}
% \newcommand{\progsol}[1]{%
% \ifthenelse{\boolean{corrige}}%
% {\small\begin{quote}{\sc Solution. }\footnotesize\verbatiminput{#1}\end{quote}\normalsize}{}}

%% Fait en sorte que la font verbatim soit \footnotesize

\makeatletter %
\renewcommand{\verbatim@font}{ %
    \ttfamily\footnotesize\catcode`\<=\active\catcode`\>=\active %
}
\makeatother

\begin{document}
\begin{center}
{\bf Polytech'Lille --- GIS 3 --- Structures de Données --- Feuille de TP numéro 1}
 
\end{center}
\medskip
\hrule
 
\bigskip

Ce TP se déroule sur deux séances (quatre heures). Les étudiants qui se
sentent en difficulté devraient réaliser la section~1, uniquement. 
Le contenu de la section~2, réservé aux étudiants plus à l'aise, 
n'est pas au programme de l'examen. 

\section{Réalisation d'un type de chaînes de caractères}

Cette partie est l'occasion d'introduire des outils importants tels 
que {\tt valgrind}, {\tt gdb} et les {\em makefiles} (voir le support
de cours). 

\question
Le programme C suivant lit une chaîne de caractères au clavier
et l'imprime sur la sortie standard.
\begin{verbatim}
#include <stdio.h>
#include <ctype.h>

int main ()
{   int c;

    c = getchar ();
    while (! isspace (c))
    {   putchar (c);
        c = getchar ();
    }
    putchar ('\n');
    return 0;
}
\end{verbatim}
Modifier ce programme de telle sorte que tous les caractères lus soient
stockés dans une variable \texttt{s}, de type \texttt{char*}, avant d'être imprimés.
La variable \texttt{s} doit pointer vers une zone allouée dynamiquement, 
redimensionnée dès qu'elle est pleine (par exemple, en lui rajoutant huit
caractères). À l'exécution, on ne veut ni débordement de tableau, ni 
fuite mémoire (tester avec \texttt{valgrind}).

\question
Reprendre l'exercice précédent, en appliquant une méthode de programmation
modulaire. Réaliser un module \texttt{chaine} au moyen de deux fichiers
(\texttt{chaine.c} et \texttt{chaine.h}) ainsi qu'un fichier \texttt{main.c}
contenant votre programme principal. À l'exécution, on ne veut ni 
débordement de tableau, ni fuite mémoire (tester avec \texttt{valgrind})
Consignes~:
\begin{enumerate}
\item	Le type doit s'appeler \texttt{struct chaine}. Cette structure
	doit comporter un champ de type \texttt{char*} pointant vers
	une zone allouée dynamiquement (plus d'autres champs éventuels).
\item	Faites le minimum~: un constructeur, un destructeur, une
	action qui ajoute un caractère à la fin d'une chaîne existante,
	et une action pour imprimer la chaîne.
\item	Dans le fichier d'entête, spécifier la structure. Attention,
	c'est plus subtil qu'on ne le croit. Faites attention à la
	chaîne vide.
\item	Utiliser les options de
        compilation \texttt{-Wall} et \texttt{-Wmissing-prototypes}.
	Compiler les fichiers séparément. 
\item	Le programme principal ne doit comporter que deux variables~:
	{\tt s}, de type \texttt{struct chaine} et {\tt c}, de type
	\texttt{int}.
\end{enumerate}

\question
Reprendre l'exercice précédent en changeant l'implantation du 
type \texttt{struct chaine} en une liste de caractères. 
Partir de l'implantation des listes de {\tt double} 
accessibles\footnote{Ou avec un navigateur web, à partir de
\url{www.lifl.fr/~boulier}.} par~:
\begin{verbatim}
$ wget --no-cache http://www.lifl.fr/~boulier/polycopies/SD/liste_double.tgz
$ tar xzf liste_double.tgz
\end{verbatim}
Réaliser un module \texttt{liste\_char} au moyen de deux fichiers
(\texttt{liste\_char.c} et \texttt{liste\_char.h}).
Modifier l'implantation du module \texttt{chaine}, pas les entêtes des
fonctions. À l'exécution, on ne veut ni débordement de tableau, ni fuite 
mémoire (tester avec \texttt{valgrind}). Consignes~:
\begin{enumerate}
\item	Le type \texttt{struct chaine} doit comporter un unique champ~:
\begin{verbatim}
struct chaine 
{   struct liste_char L;
};
\end{verbatim}
\item	Faut-il ajouter une directive \texttt{\#include} dans \texttt{chaine.h}~?
\item	Quelle implantation des listes convient le mieux à cette question~?
\item	Utiliser les options de
        compilation \texttt{-Wall} et \texttt{-Wmissing-prototypes}.
	Compiler les fichiers séparément. 
\item	Construire un {\em makefile} en utilisant la méthode décrite
	dans le support de cours.
\end{enumerate}

% \bigskip
% 
% \centerline{\raisebox{-2pt}{$\ast$}\raisebox{2pt}{$\ast$}\raisebox{-2pt}{$\ast$}}
% 
% \bigskip

\section{Le codage UTF-8}

Les caractères non ASCII (les caractères accentués, en particulier), sont
fréquemment une source de problèmes en informatique. En France, on
utilise couramment deux codages différents pour les représenter~: 
le codage ISO 8859-1 (appelé aussi Latin-1) et le codage UTF-8. 
Ce dernier prend de plus en plus d'importance, avec le développement
d'internet, puisqu'il permet de coder des textes écrits dans toutes 
les langues terrestres\footnote{Plus quelques autres, comme le Klingon.}.

\question
Dans cette question ouverte, on vous demande de réfléchir à la
conception d'un module qui aide à résoudre ce problème de cohabitation
de différents formats. 

Le module doit permettre de stocker, dans des variables, des chaînes 
passées en paramètre suivant les deux formats (UTF-8 et Latin-1),
de les afficher suivant les deux formats,
de déterminer le nombre d'octets\footnote{Dans tout ce qui suit, on prend
bien soin de distinguer les {\em caractères} des {\em octets}.} 
nécessaires à la représentation des chaînes sous la forme d'un tableau, 
ainsi que le nombre de caractères de la chaîne.
Bien qu'on ne gère que deux formats, on vous demande de réfléchir à une
solution facilement extensible.

Vous n'aurez probablement pas le temps de réaliser ce module
intégralement. On vous demande de réaliser au moins les fichiers d'entêtes
(implantation, prototype des fonctions globales, spécifications des
structures de données et des fonctions). Si vous tentez de réaliser
le code, supposez, dans un premier temps, que les chaînes de caractères
sont correctement codées (ne vérifiez pas la validité des formats).

\subsubsection*{Une introduction aux codages des caractères}

\paragraph{ASCII.} Un code ASCII est un entier codé sur~$7$ bits, dans
	l'intervalle $[\mbox{\tt 0},\, \mbox{\tt 0x7f} = 127]$. 
Parmi les caractères représentables
en ASCII, on trouve~: les chiffres, les lettres (non accentuées) minuscules
et majuscules, des signes de ponctuations. Habituellement, un code ASCII
est codé sur un octet (le bit de poids fort valant~$0$).
\paragraph{Latin-1.} C'est le jeu de caractères par défaut sous Linux.
C'est une extension de l'ASCII~: tous les codes ASCII sont aussi
des codes Latin-1 mais des codes supplémentaires (comportant toutes
les lettres accentuées de la langue française) sont définis. Ces codes
supplémentaires sont des entiers de l'intervalle\footnote{Ceci
explique que la fonction {\tt getchar} retourne un {\tt int}
et pas un {\tt char}. Si elle retournait un {\tt char}, on ne pourrait
pas distinguer le caractère {\tt 0xff} de l'entier $-1$, qui code {\tt EOF}.
On rencontrerait alors des fins de fichiers prématurées sur certains
fichiers non ASCII.
Comme elle retourne un {\tt int}, le caractère {\tt 0xff} est retourné
sous la forme {\tt 0x00ff}, qui ne peut pas être confondu avec {\tt EOF},
codé par {\tt 0xffff}.}
$[\mbox{\tt 0xa0},\, \mbox{\tt 0xff}]$ (les entiers de l'intervalle 
$[\mbox{\tt 0x80},\, \mbox{\tt 0x9f}]$ ne correspondent à aucun
code Latin-1). Habituellement, un code Latin-1 est codé sur un octet.
\paragraph{UTF-8.} C'est un des codages possibles du jeu de caractères
UNICODE. En UTF-8, un caractère est codé sur~$1$,~$2$,~$3$ ou~$4$
octets. Il a une {\em valeur}\footnote{En ASCII ou en Latin-1, il est
naturel de confondre la valeur (par exemple, $\mbox{\tt 0x41} = 65$
pour le {\tt 'A'}) et le codage (le nombre {\tt 0x41}, codé sur un octet).
En UTF-8, le codage d'un caractère, à partir de sa valeur, est plus compliqué,
ce qui pousse à distinguer les deux notions. Pour être vraiment précis,
UNICODE est l'ensemble des valeurs possibles. UTF-8 est un codage
particulier d'UNICODE (le plus populaire), mais il en existe d'autres, 
comme UTF-16 et UTF-32. Ça vous paraît compliqué~? Si on veut vraiment
creuser les choses, ça le devient encore bien davantage et on finit par
se demander~: «~mais qu'est-ce que c'est, un caractère~?~». Aux personnes
intéressées, on conseille la lecture (passionnante) de \cite{Korpela01}.} 
dans l'intervalle $[0,\, \mbox{\tt 0x10ffff}]$.
Un caractère codé sur~$1$ octet est un code ASCII. Sa valeur
appartient donc à l'intervalle $[\mbox{\tt 0},\, \mbox{\tt 0x7f} = 127]$.

Si un caractère est codé sur $2$, $3$ ou $4$ octets, alors chaque
octet est composé d'une suite de bits de contrôle, puis de
bits destinés à former la valeur du caractère, comme on l'explique
ci-dessous.

Soit un caractère codé par $n$ octets ($2 \leq n \leq 4$).
Le premier octet commence par $n$ bits à $1$, suivi d'un bit à $0$,
suivi de $7-n$ bits de valeur. Les autres octets commencent par les deux
bits $10$, suivi de $6$ bits de valeur.
La valeur du caractère s'obtient en concaténant les bits de valeur.
Exemple~:

\smallskip

\begin{tabular}{lllr}
$2$ octets. & Codage & : & {\tt 110xxxxx 10yyyyyy}\\ 
	    & Valeur & : & {\tt xxx xxyyyyyy} \\
$3$ octets. & Codage & : & {\tt 1110xxxx 10yyyyyy 10zzzzzz} \\
            & Valeur & : & {\tt xxxxyyyy yyzzzzzz} \\
$4$ octets. & Codage & : & {\tt 11110xxx 10yyyyyy 10zzzzzz 10wwwwww} \\
            & Valeur & : & {\tt xxxyy yyyyzzzz zzwwwwww}
\end{tabular}

\smallskip

\noindent
Certains codages sont interdits:
\begin{itemize}
\item
le codage d'une valeur doit se faire sur un nombre minimal d'octets~:
\begin{enumerate}
\item
les premiers octets {\tt 0xc0} et {\tt 0xc1} sont interdits (ils
        donneraient lieu à un code ASCII sur deux octets)~;
\item
si le premier octet est {\tt 0xe0}, la valeur finale doit être
supérieure ou égale à {\tt 0x800}~;
\item
si le premier octet est {\tt 0xf0}, la valeur finale doit être
supérieure ou égale à {\tt 0x10000}.
\end{enumerate}
\item
la valeur ne doit pas dépasser {\tt 0x10ffff}~;
\item
les valeurs dans l'intervalle $[\mbox{\tt 0xd800},\, \mbox{\tt 0xdfff}]$ 
sont réservées pour coder de l'UTF-16 et ne correspondent à aucun 
caractère UTF-8~;
\item
il existe $66$ valeurs pour des «~non-caractères~»~: les valeurs
comprises entre {\tt 0xfdd0} et {\tt 0xfdef} ainsi que les $34$
valeurs terminant par {\tt 0xfffe} et {\tt 0xffff} (de
{\tt 0xfffe}, {\tt 0xffff} à {\tt 0x10fffe}, {\tt 0x10fff}).
\end{itemize}
La conversion entre ASCII et UTF-8 est immédiate, puisque 
tout document ASCII est un cas particulier de document UTF-8.
La conversion entre Latin-1 et UTF-8 n'est pas bien compliquée
non plus~: les codes ASCII sont inchangés~; les codages UTF-8 de
valeur appartenant à $[\mbox{\tt 0xa0},\, \mbox{\tt 0xff}]$ correspondent
aux codages Latin-1 appartenant à ce même intervalle.

Le symbole «~€~» (qui ne fait pas partie du jeu de caractères Latin-1),
est codé sur~$3$~octets~: {\tt 0xe2}, {\tt 0x82} et {\tt 0xac}. Sa
valeur est {\tt 0x20ac}. Certaines lettres, utiles en Français, n'existent
pas en Latin-1\footnote{Mais elles existent en ISO 8859-15, appelé
aussi~: Latin-9.}, comme «~œ,~Œ~» et même, pour certains noms 
propres, «~ÿ~». En UTF-8, elles sont toutes codées sur deux octets.

Le {\em replacement character}, qui apparaît souvent sous la forme
d'un point d'interrogation blanc sur fond noir, est utilisé par
les routines d'impression UTF-8, pour désigner un codage non reconnu.
Il est codé sur~$3$~octets~: {\tt 0xef}, {\tt 0xbf} et {\tt 0xbd}.
Sa valeur est {\tt 0xfffd}.

\subsubsection*{Quelques commandes utiles}

La commande {\tt od} imprime les valeurs des octets d'un fichier.
La commande {\tt file} fournit le codage d'un fichier texte.
La commande {\tt iconv} convertit un fichier d'un codage dans un autre.

\begin{verbatim}
$ cat fichier.txt 
Lætitia, François, à table !
$ od -t x1 -c fichier.txt 
0000000  4c  c3  a6  74  69  74  69  61  2c  20  46  72  61  6e  c3  a7
          L 303 246   t   i   t   i   a   ,       F   r   a   n 303 247
0000020  6f  69  73  2c  20  c3  a0  20  74  61  62  6c  65  20  21  0a
          o   i   s   ,     303 240       t   a   b   l   e       !  \n
$ file -i fichier.txt
fichier.txt: text/plain; charset=utf-8
$ iconv --from utf-8 --to iso8859-1 < fichier.txt > fichier2.txt
$ cat fichier2.txt 
L�titia, Fran�ois, � table !
$ od -t x1 -c fichier2.txt 
0000000  4c  e6  74  69  74  69  61  2c  20  46  72  61  6e  e7  6f  69
          L 346   t   i   t   i   a   ,       F   r   a   n 347   o   i
0000020  73  2c  20  e0  20  74  61  62  6c  65  20  21  0a
          s   ,     340       t   a   b   l   e       !  \n
0000035
$ file -i fichier2.txt
fichier2.txt: text/plain; charset=iso-8859-1
\end{verbatim}

\paragraph{Exercice.} Les commandes ci-dessus ont été exécutées dans
un terminal, qui attend des chaînes de caractères suivant un certain
format. À votre avis, lequel~?

\subsubsection*{Manipulations de bits en C}

Des opérateurs utiles sont~: {\tt \&} (le «~et~» bit à bit),\verb+|+ 
(le «~ou inclusif~» bit à bit),\verb+^+ (le «~ou exclusif~» bit à bit). 
On peut leur ajouter les opérateurs\verb+<<+ (décalage des bits à gauche, d'un
certain nombre de positions) et\verb+>>+ (décalage à droite), dont on 
pourrait se passer, puisqu'ils peuvent se traduire par des multiplications
et des divisions par des puissances de~$2$. 

Attention au fait qu'un {\tt char}, en C, est en réalité, le type «~entier
signé sur~$8$ bits~» (dans l'intervalle $[-128,\, 127]$, donc). 
Lorsqu'on manipule des caractères non ASCII, il est souvent préférable 
d'utiliser le type {\tt unsigned char} du C, dont les valeurs appartiennent
à~$[0,\,255]$. Lors des fréquentes conversions en {\tt int}, on évite
ainsi la propagation du bit de signe (bit numéro~$7$) sur les octets 
de poids fort.

L'action suivante analyse et décompose le premier octet d'un caractère
UTF-8. 
\begin{verbatim}
/*
 * Décompose l'octet c (censé être le premier octet d'une séquence UTF-8).
 * L'entier *counter reçoit le nombre de bits de contrôle à 1
 * Le long *code reçoit les bits de valeur.
 *
 * Par exemple, si c = 110xxxxx alors *counter = 2 et *code = xxxxx.
 * Si c = 0xxxxxxx (code ASCII) alors *counter = 0 et *code = c.
 */

static void analyse_premier_octet (int* counter, long* code, unsigned char c)
{   int cpt, bits, mask;

    cpt = 0;
    bits = c;
    mask = 0x80;                /* 1000 0000 binaire */
/* 
 * Invariants de boucle : 
 * - mask comporte un seul bit à 1
 * - bits est égal à c, sans les bits à 1, situés plus à gauche que mask
 * - cpt est égal au nombre de bits à 1 de c, situés plus à gauche que mask
 */
    while ((bits & mask) != 0)  /* tq bits comporte un bit à 1 face à mask */
    {   bits ^= mask;           /* effacer ce bit */
        cpt += 1;
        mask >>= 1;             /* décaler d'une position vers la droite
                                   l'unique bit à 1 de mask */
    }
    *counter = cpt;
    *code = (long)bits;
}
\end{verbatim}

% Les exercices qui suivent peuvent vous aider à réaliser le module demandé.
% 
% \paragraph{Exercice.} Écrire une fonction {\tt ajoute\_6bits}, 
% paramétrée par une valeur
% (de type {\tt long}) et un octet, qui retourne la valeur, décalée de~$6$
% positions vers la gauche, plus les~$6$ bits de poids faible de l'octet.
% Comment utiliser cette fonction pour reconstituer la valeur
% d'un caractère à partir de son codage UTF-8~?
% 
% \paragraph{Exercice.} Écrire une fonction, paramétrée par une chaîne
% de caractères~{\tt s} (de type {\tt char*}), qui retourne {\tt true} si~{\tt s}
% pourrait être une chaîne au format Latin-1.
% 
% \paragraph{Exercice.} Même question pour le format UTF-8.

\bibliographystyle{plain}
\bibliography{reecriture}
\end{document}


