% xelatex tp1.tex

\documentclass[12pt]{article}
\usepackage{xunicode}
\usepackage{fontspec}
\usepackage{xltxtra}
\usepackage{url}
\usepackage[francais]{babel}
\usepackage{fullpage}

% \usepackage{epsf}
% \usepackage{amsmath}
% \usepackage{amsfonts}
% \usepackage{amssymb}	% pour \varnothing
% \usepackage[latin1]{inputenc}
% \usepackage[T1]{fontenc}
% \usepackage{times}
% \usepackage{epic,eepic}
% \usepackage{mathrsfs}   % pour la commande \mathscr
% \usepackage{moreverb}   % pour verbatiminput
% \usepackage{ifthen}     % pour les versions enseignant/étudiant
% \usepackage{graphics}
% \usepackage{epsfig}
% \usepackage{multicol}


%% Environnements

\newtheorem{notation}{Notation}
\newtheorem{abus}{Abus de langage}
\newtheorem{corollary}{Corollaire}
\newtheorem{theorem}{Théorème}
\newtheorem{lemma}{Lemme}
\newtheorem{proposition}{Proposition}
\newtheorem{convention}{Convention}
\newtheorem{definition}{Définition}
\newcommand{\Algf}{\bfseries \sffamily }
\newcommand{\algf}{\sffamily }
\newenvironment{proof}{{\bf Preuve} }{\ensuremath{\Box}\normalsize}
\newcommand{\marginlabel}[1]{\mbox{}\marginpar{\raggedleft\hspace{0pt}#1}}

%% Algorithmes

\newcommand{\BEGIN}{{\algf begin}}
\newcommand{\END}{{\algf end}}
\newcommand{\IF}{{\algf if}}
\newcommand{\THEN}{{\algf then}}
\newcommand{\ELSE}{{\algf else}}
\newcommand{\ELIF}{{\algf elif}}
\newcommand{\FI}{{\algf fi}}
\newcommand{\WHILE}{{\algf while}}
\newcommand{\FOR}{{\algf for}}
\newcommand{\DO}{{\algf do}}
\newcommand{\OD}{{\algf od}}
\newcommand{\RETURN}{{\algf return}}
\newcommand{\PROCEDURE}{{\algf procedure}}
\newcommand{\FUNCTION}{{\algf function}}
\newcommand{\INDENTER}{{\algf si} \=\+\kill}

%% Opérateurs

\newcommand{\initial}{\mathop{\mathsf{init}}}
\newcommand{\separant}{\mathop{\mathsf{sep}}}
\newcommand{\rem}{\mathop{\mathsf{rem}}}
\newcommand{\quo}{\mathop{\mathsf{quo}}}
\newcommand{\lcoeff}{\mathop{\mathsf{lcoeff}}}
\newcommand{\mvar}{\mathop{\mathsf{mvar}}}
\newcommand{\ld}{\mathop{\mathrm{ld}}}
\newcommand{\rank}{\mathrm{\mathsf{rank}}}
\newcommand{\prem}{\mathop{\mathsf{prem}}}
\newcommand{\remp}{\mathrel{\mathsf{partial\_rem}}}
\newcommand{\remf}{\mathrel{\mathsf{full\_rem}}}
\renewcommand{\gcd}{\mathop{\mathrm{gcd}}}
\renewcommand{\deg}{\mathop{\mathrm{deg}}}
\newcommand{\ord}{\mathop{\mathrm{ord}}}
\newcommand{\pairs}{\mathop{\mathrm{pairs}}}
\newcommand{\dd}{\mathrm{d}}
\newcommand{\ideal}[1]{(#1)}
\newcommand{\cont}{\mathop{\mathrm{contenu}}}
\newcommand{\pp}{\mathop{\mathrm{partie\_primitive}}}
\newcommand{\init}{\mathop{\mathrm{coefficient\_dominant}}}
\newcommand{\pgcd}{\mathop{\mathrm{pgcd}}}
\newcommand{\ppmc}{\mathop{\mathrm{ppcm}}}
\newcommand{\NF}{\mathop{\mathrm{NF}}}
\newcommand{\rang}{\mathop{\mathrm{rang}}}
\newcommand{\card}{\mathop{\mbox{card}}}

%% Ensembles mathématiques

\newcommand{\A}{\mathbb{A}}
\newcommand{\Q}{\mathbb{Q}}
\newcommand{\Z}{\mathbb{Z}}
\newcommand{\C}{\mathbb{C}}
\newcommand{\N}{\mathbb{N}}
\newcommand{\R}{\mathbb{R}}
\renewcommand{\emptyset}{\varnothing}

%% Réécriture

\newcommand{\egaldef}{\mathop{\underset{\mathrm{def}}{=}}}
\newcommand{\fleche}{\hbox to 1cm {\rightarrowfill}}
\newcommand{\chefle}{\hbox to 1cm {\leftarrowfill}}
\newcommand{\mafleche}{\mathrel{\fleche}}
\newcommand{\machefle}{\mathrel{\chefle}}
\def\build#1_#2^#3{\mathrel{\mathop{\kern 0pt#1}\limits_{#2}^{#3}}}

%% Saturations

\newbox\deuxpoints
\setbox\deuxpoints=\hbox{:}
\def\dv{\mathbin{\raisebox{1pt}{\copy\deuxpoints}}}
\newbox\infini
\setbox\infini=\hbox{\footnotesize $\infty$}
\def\sat#1{#1^{\copy\infini}}
\def\H#1{H_{\!{#1}}}
\def\S#1{S_{\!{#1}}}
\def\HH#1{\sat{\H{#1}}}
\def\SS#1{\sat{\S{#1}}}

%% \question -> numérote les question
%% \Question{k} -> indique en plus que la question vaut k points
%% \Section{intitulé} -> rajoute à l'intitulé la somme des points des 
%%    questions qui figurent dans la section (faire deux compilations).

\newcounter{question_counter}
\setcounter{question_counter}{0}
\newcommand{\question}{\addtocounter{question_counter}1 %
    \paragraph{\bf Question \arabic{question_counter}.}}
\newcommand{\questiondiff}{\addtocounter{question_counter}1 %
    \paragraph{\bf Question \arabic{question_counter} (plus difficile).}}

\newcounter{points_counter}
\newcounter{section_points_counter}
\newcommand{\Question}[1]{\addtocounter{question_counter}1 %
    \addtocounter{points_counter}{#1} %
    \addtocounter{section_points_counter}{#1} %
    \typeout{Question \arabic{question_counter}. Total des points \arabic{points_counter}} %
    \typeout{points_counter_\Alph{section} \arabic{section_points_counter}} %
    \paragraph{\bf Question \arabic{question_counter}}{[#1\,\ifnum #1 = 1 pt\else pts\fi]}.}

\newcommand{\Section}[1]{\setcounter{section_points_counter}0 %
    \section{{#1} (\ref{points_counter_\Alph{section}} points)}}

%% \solution{...}
%% \progsol{fichier_source}

% \newboolean{corrige}
% \setboolean{corrige}{false}
% \newcommand{\solution}[1]{%
% \ifthenelse{\boolean{corrige}}%
% {\small\begin{quote}{\sc Solution. }#1\end{quote}\normalsize}{}}
% \newcommand{\progsol}[1]{%
% \ifthenelse{\boolean{corrige}}%
% {\small\begin{quote}{\sc Solution. }\footnotesize\verbatiminput{#1}\end{quote}\normalsize}{}}

%% Fait en sorte que la font verbatim soit \footnotesize

\makeatletter %
\renewcommand{\verbatim@font}{ %
    \ttfamily\footnotesize\catcode`\<=\active\catcode`\>=\active %
}
\makeatother

\begin{document}
\begin{center}
{\bf Polytech'Lille --- GIS 3 --- Structures de Données --- Feuille de TP numéro 4'}
 
\end{center}
\medskip
\hrule

\section{Réalisation d'une pile}

Ce TP se déroule sur une séance. 
On s'intéresse au programme principal {\tt main.c} suivant.
\begin{verbatim}
#include <stdio.h>
#include "pile.h"

int main ()
{   struct pile P;
    int i;

    init_pile (&P);
    for (i = 1; i < 10; i++)
        empiler (&P, i);
    while (! est_vide_pile (P))
    {   i = depiler (&P);
        printf ("%d\n", i);
    }
    clear_pile (&P);
    return 0;
}
\end{verbatim}

\question
Sachant que $P$ est une pile d'entiers, qu'est-ce que ce
programme est censé afficher sur la sortie standard~?

\bigskip

On souhaite réaliser un module de pile d'entiers en utilisant
le module de liste de doubles, étudié au premier TP.
À la fin du TP, vous devrez donc avoir~$5$ fichiers~: {\tt main.c},
{\tt pile.c}, {\tt pile.h}, {\tt liste\_double.c} et {\tt liste\_double.h}.
On impose le fichier {\tt pile.h} suivant.
\begin{verbatim}
#include <stdbool.h>
#include "liste_double.h"

struct pile
{   struct liste_double L;
};

extern void init_pile (struct pile*);
extern void vider_la_pile (struct pile*);
extern void clear_pile (struct pile*);
extern void empiler (struct pile*, int);
extern int depiler (struct pile*);
extern bool est_vide_pile (struct pile);
\end{verbatim}

\question
Écrire {\tt pile.c}. 
Compiler et tester. Utiliser un {\em Makefile} (voir sur le site du cours).
Vérifier avec {\tt valgrind} que le programme est correct.

\section{Réalisation d'une file}

On s'intéresse au programme principal suivant.
\begin{verbatim}
#include <stdio.h>
#include "file.h"

int main ()
{   struct file F;
    int i;

    init_file (&F);
    for (i = 1; i < 10; i++)
        enfiler (&F, i);
    while (! est_vide_file (F))
    {   i = defiler (&F);
        printf ("%d\n", i);
    }
    clear_file (&F);
    return 0;
}
\end{verbatim}

\question
Sachant que $P$ est une file d'entiers, qu'est-ce que ce
programme est censé afficher sur la sortie standard~?

\bigskip
On impose le fichier {\tt file.h} suivant.
\begin{verbatim}
#include <stdbool.h>
#include "liste_double.h"

struct file
{   struct liste_double L;
};

extern void init_file (struct file*);
extern void vider_la_file (struct file*);
extern void clear_file (struct file*);
extern void enfiler (struct file*, int);
extern int defiler (struct file*);
extern bool est_vide_file (struct file);
\end{verbatim}

\question
Écrire {\tt file.c}.
Quelles sont la ou les fonctions de manipulation de listes,
qui seraient souhaitables pour réaliser une file, mais qui
ne sont pas implantées~?

\question
Compléter le module {\tt liste\_double}. Que faut-il ajouter
à {\tt liste\_double.h}~? à {\tt liste\_double.c}~?

\question
Suggérez une variante d'implantation des listes qui optimise le 
fonctionnement du module de file.

\end{document}


