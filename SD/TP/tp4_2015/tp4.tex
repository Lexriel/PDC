% xelatex tp4

\documentclass[11pt]{article}
\usepackage{xunicode}
\DeclareTextCommandDefault{\nobreakspace}{\leavevmode\nobreak\ } 
\usepackage{fontspec}
\usepackage{xltxtra}
\usepackage{url}
\usepackage[francais]{babel}
\usepackage{fullpage}
\usepackage{fancyvrb}
\usepackage{graphics}
\usepackage{tikz}
\usepackage{multicol}
\usepackage{wrapfig}

%% Environnements

%% Algorithmes

\newcommand{\PROCEDURE}{{\sffamily procedure}}
\newcommand{\FUNCTION}{{\sffamily function}}
\newcommand{\LOCAL}{{\sffamily local variables}}
\newcommand{\BEGIN}{{\sffamily begin}}
\newcommand{\END}{{\sffamily end}}
\newcommand{\IF}{{\sffamily if}}
\newcommand{\THEN}{{\sffamily then}}
\newcommand{\ELSE}{{\sffamily else}}
\newcommand{\ELIF}{{\sffamily elif}}
\newcommand{\FI}{{\sffamily end if}}
\newcommand{\ENDIF}{{\sffamily end if}}
\newcommand{\WHILE}{{\sffamily while}}
\newcommand{\FOR}{{\sffamily for}}
\newcommand{\DO}{{\sffamily do}}
\newcommand{\OD}{{\sffamily end do}}
\newcommand{\ENDDO}{{\sffamily end do}}
\newcommand{\ENDWHILE}{{\sffamily end while}}
\newcommand{\TRY}{{\sffamily try}}
\newcommand{\CATCH}{{\sffamily catch}}
\newcommand{\ENDTRY}{{\sffamily end try}}
\newcommand{\RAISE}{{\sffamily raise exception}}
\newcommand{\RETURN}{{\sffamily return}}
\newcommand{\ERROR}{{\sffamily error}}
\newcommand{\INDENTER}{{\sffamily si} \=\+\kill}
\newcommand{\NIL}{{\sffamily NIL}}

%% Opérateurs

%% Ensembles mathématiques

%% Réécriture

%% Saturations

%% \question -> numérote les question
%% \Question{k} -> indique en plus que la question vaut k points
%% \Section{intitulé} -> rajoute à l'intitulé la somme des points des 
%%    questions qui figurent dans la section (faire deux compilations).

\newcounter{question_counter}
\setcounter{question_counter}{0}
\newcommand{\question}{\addtocounter{question_counter}1 %
    \paragraph{\bf Question \arabic{question_counter}.}}
\newcommand{\questiondiff}{\addtocounter{question_counter}1 %
    \paragraph{\bf Question \arabic{question_counter} (plus difficile).}}

\newcounter{points_counter}
\newcounter{section_points_counter}
\newcommand{\Question}[1]{\addtocounter{question_counter}1 %
    \addtocounter{points_counter}{#1} %
    \addtocounter{section_points_counter}{#1} %
    \typeout{Question \arabic{question_counter}. Total des points \arabic{points_counter}} %
    \typeout{points_counter_\Alph{section} \arabic{section_points_counter}} %
    \paragraph{\bf Question \arabic{question_counter}}{[#1\,\ifnum #1 = 1 pt\else pts\fi]}.}

\newcommand{\Section}[1]{\setcounter{section_points_counter}0 %
    \section{{#1} (\ref{points_counter_\Alph{section}} points)}}

%% Fait en sorte que la font Verbatim soit \footnotesize

\makeatletter %
\renewcommand{\verbatim@font}{ %
    \ttfamily\footnotesize\catcode`\<=\active\catcode`\>=\active %
}
\makeatother

\begin{document}
\begin{center}
{\bf Polytech'Lille --- GIS 3 --- Structures de Données --- Feuille de TP numéro 4}
 
\end{center}
\medskip
\hrule
 
\section{Préliminaires}

\subsection{Orientation dans le plan}

Soient $\vec u$ et $\vec v$ deux vecteurs 
dans le plan, de coordonnées
$$\vec u = \left(\begin{array}{c} x_u \\ y_u\end{array}\right)\,,\quad
  \vec v = \left(\begin{array}{c} x_v \\ y_v\end{array}\right)\,.$$
\begin{wrapfigure}{r}{0.3\linewidth}
\centerline{
\begin{tikzpicture}
\draw [->] (0,0) -- (2,1) ;
\draw (1.5,0.5) node {$\vec u$} ;
\draw [->] (0,0) -- (1,2.5) ;
\draw (.5,2) node {$\vec v$} ;
\draw [->] (1,0.5) arc (26.565:68.199:1) ;
\draw (1.2,1.1) node {\small $\theta > 0$} ;
\end{tikzpicture}}
\end{wrapfigure}
Supposons qu'on veuille superposer~$\vec u$ sur~$\vec v$ (voir figure
de droite).
Faut-il tourner dans le sens trigonométrique ou dans le sens inverse~?
Il suffit de tester le signe de l'angle orienté~$\theta$~: il faut
tourner dans le sens trigonométrique si le signe est positif, dans
le sens inverse s'il est négatif. Ce signe s'obtient par un simple
calcul de déterminant~:

$$\mathop{\mbox{\rm signe}}(\theta) =
  \mathop{\mbox{\rm signe}} \left|\begin{array}{cc} x_u & x_v \\ y_u & y_v 
                                \end{array}\right|\,.$$

\question
Pour superposer~$\vec u$ sur~$\vec v$, faut-il tourner dans le
sens trigonométrique ou dans le sens inverse~?
$$\vec u = \left(\begin{array}{c} 2 \\ 1\end{array}\right)\,,\quad
  \vec v = \left(\begin{array}{c} 3 \\ 1\end{array}\right)\,.$$

\begin{wrapfigure}{r}{0.3\linewidth}
\centerline{
\begin{tikzpicture}
\draw (4,1) -- (3,2) -- (0,0) ;
\draw (4.2,.8) node {$A$} ;
\draw (3,2.2) node {$B$} ;
\draw (-.2,-.2) node {$C$} ;
\end{tikzpicture}} 
\centerline{
\begin{tikzpicture}
\draw (.2,-.2) node {$A$} ;
\draw (-1,1.2) node {$B$} ;
\draw (-4.2,-1.2) node {$C$} ; 
\draw [->] (0,0) -- (-1,1) ;
\draw (-.4,.8) node {$\vec u$} ;
\draw [->] (0,0) -- (-4,-1) ;
\draw (-1.9,-.7) node {$\vec v$} ;
\draw [->] (-.7,.7) arc (135:190:1) ;
\draw (-1.6,0.2) node {\small $\theta > 0$} ;
\end{tikzpicture}}
\end{wrapfigure}
Considérons maintenant trois sommets consécutifs 
$$A = \left(\begin{array}{c} x_A \\ y_A\end{array}\right)\,,\quad
  B = \left(\begin{array}{c} x_B \\ y_B\end{array}\right)\,,\quad
  C = \left(\begin{array}{c} x_C \\ y_C\end{array}\right)$$
d'un polygone (figure de droite, en haut) et imaginons qu'on se déplace de~$A$ vers~$C$.
Arrivé en~$B$, tourne-t-on vers la gauche ou vers la droite~?
Réponse~: on tourne à gauche si l'angle~$\theta$ entre~$\vec u$ et
$\vec v$ est positif (figure de droite, en bas), où
$\vec u = B - A$ désigne le vecteur de~$A$ vers~$B$ et 
$\vec v = C - A$ désigne le vecteur de~$A$ vers~$C$ (cela revient 
à appliquer la méthode décrite plus haut en prenant~$A$ comme origine).

$$\mathop{\mbox{\rm signe}}(\theta) =
  \mathop{\mbox{\rm signe}} \left|\begin{array}{cc} 
        x_B - x_A & x_C - x_A \\ 
        y_B - y_A & y_C - y_A 
                            \end{array}\right|\,.$$

\question
Lorsqu'on se déplace de~$A$ vers~$C$ (en passant par~$B$),
tourne-t-on vers la gauche ou vers la droite~? Même question
si on se déplace de~$C$ vers~$A$.
$$A = \left(\begin{array}{c} 4 \\ 1\end{array}\right)\,,\quad
  B = \left(\begin{array}{c} 3 \\ 2\end{array}\right)\,,\quad
  C = \left(\begin{array}{c} 0 \\ 0\end{array}\right)\,.$$

\subsection{Trier un tableau en C}

La fonction \texttt{qsort}, de la bibliothèque standard du langage C,
permet de trier des tableaux d'éléments d'un type quelconque, suivant
une relation d'ordre quelconque. Prenons l'exemple d'un tableau~$T$ de~$n$
doubles, qu'on souhaite trier par ordre croissant. Il suffit d'écrire 
une fonction de comparaison, paramétrée par les adresses~$p$ et~$q$ 
de deux éléments de~$T$, qui retourne~$-1$,~$0$ ou~$1$, suivant que 
le double désigné par~$p$ est inférieur, inférieur ou égal ou supérieur au
double désigné par~$q$. Pour des raisons de généricité, les adresses
passées en paramètre sont de type \texttt{const void*}. Il suffit
d'effectuer une conversion de pointeurs en début de fonction pour
se ramener au type \texttt{double*}.

\begin{Verbatim}[fontsize=\small]
int compare_doubles (const void* p0, const void* q0)
{   double* p = (double*)p0;
    double* q = (double*)q0;
    if (*p < *q)
        return -1;
    else if (*p == *q)
        return 0;
    else
        return 1;
}
\end{Verbatim}

\begin{wrapfigure}{r}{.3\linewidth}
\begin{tikzpicture}
\draw (0,0) -- (6,0) ;
\draw (3,2) node {$\circ$} ;
\draw [style=dotted] (0,0) -- (4.5,3) ;
\draw (2.8,2.3) node {$A$} ;
\draw (5,1) node {$\circ$} ;
\draw [style=dotted] (0,0) -- (6,1.2) ;
\draw [->] (2.157,0.431) arc (11.31:33.69:2.2);
\draw (5,1.3) node {$B$} ;
\draw [->] (2,0) arc (0:11.30:2) ;
\draw (1,3) node {$\circ$} ;
\draw [style=dotted] (0,0) -- (1.5,4.5) ;
\draw (.7,3.1) node {$C$} ;
\draw [->] (1.997,1.331) arc (33.69:45:2.4) ;
\draw (4,4) node {$\circ$} ;
\draw [style=dotted] (0,0) -- (5,5) ;
\draw (3.7,4.2) node {$D$} ;
\draw [->] (1.838,1.838) arc (45:71.565:2.6) ;
\end{tikzpicture}
\end{wrapfigure}

Dans la fonction qui doit trier~$T$, l'appel à \texttt{qsort} peut alors
s'écrire~:
\begin{Verbatim}[fontsize=\small]
    double T [n];
    ...
    qsort (T, n, sizeof (double), &compare_doubles);
\end{Verbatim}

\question
Supposons que~$T$ soit un tableau de~$n$ éléments de 
type \texttt{struct point} (voir déclaration ci-dessous).
Écrire une fonction de comparaison \texttt{compare\_points}
et un appel à \texttt{qsort} qui
permette de trier~$T$ par angle croissant (sur la figure
de droite, par angle croissant, les quatre points sont
$B$, $A$, $D$ et~$C$).
\begin{Verbatim}[fontsize=\small]
    struct point {
        double x;     /* abscisse */
        double y;     /* ordonnée */
        char ident;   /* identificateur */
    };
\end{Verbatim}

\question
Écrire une fonction \texttt{tourne\_a\_gauche}, paramétrée
par trois points~$A$, $B$ et~$C$, qui retourne \texttt{true}
si le chemin de~$A$ vers~$C$ tourne à gauche en~$B$, et
\texttt{false} sinon.

\section{L'algorithme de Graham}

L'algorithme de Graham calcule l'enveloppe convexe d'un ensemble
de~$n$ points du plan (voir Figure~\ref{fig-enveloppe}) avec une
complexité en $\Theta (n\,\log(n))$. 
Un pseudo-code est donné Figure~\ref{algo-Graham}.
Voir aussi \cite[section 33.3]{CLRS02}.

\begin{figure}[ht]
\centerline{\begin{tikzpicture}[scale=.09]
\draw (-2,-2)  node {$A$} ;
\draw (0.000000,0.000000) node {$\circ$} ;
\draw (38.300775,81.387560) node {$B$} ;
\draw (40.300775,79.387560) node {$\circ$} ;
\draw (85.307519,4.746114) node {$C$} ;
\draw (82.307519,4.746114) node {$\circ$} ;
\draw (38.844346,64.690464) node {$D$} ;
\draw (35.844346,64.690464) node {$\circ$} ;
\draw (16.232867,24.879141) node {$E$} ;
\draw (13.232867,24.879141) node {$\circ$} ;
\draw (8.797636,10.095421) node {$F$} ;
\draw (5.797636,10.095421) node {$\circ$} ;
\draw (16.519759,56.930767) node {$G$} ;
\draw (18.519759,54.930767) node {$\circ$} ;
\draw (68.672118,24.024093) node {$H$} ;
\draw (65.672118,24.024093) node {$\circ$} ;
\draw (83.232410,92.516572) node {$I$} ;
\draw (81.232410,90.516572) node {$\circ$} ;
\draw (59.019427,90.984004) node {$J$} ;
\draw (59.019427,87.984004) node {$\circ$} ;
\draw [style=dotted] (0.000000,0.000000) -- (18.519759,54.930767) -- (40.300775,79.387560) -- (59.019427,87.984004) -- (81.232410,90.516572) -- (82.307519,4.746114) -- (0.000000,0.000000) ;
\end{tikzpicture}}
\caption{Enveloppe convexe d'un ensemble de points. Supposons que
l'origine soit en~$A$ et énumérons les autres les sommets de l'enveloppe
convexe par angle croissant~: $C$, $I$, $J$, $B$, $G$. On remarque
une propriété, exploitée par l'algorithme de Graham~: à chaque point
rencontré, on tourne à gauche.}
\label{fig-enveloppe}
\end{figure}

\begin{figure}[ht]
\begin{tabbing}
\PROCEDURE\ {\sffamily Graham} ($T$)\\
\INDENTER
{\em $T[0,1,\ldots,n-1]$ est un tableau de~$n$ points ($n \geq 3$). 
L'origine est en $T_0$.} \\
\quad
{\em Les autres points ont des coordonnées strictement positives.} \\
\BEGIN \\
\INDENTER
    Trier~$T [1, \ldots, n-1]$ par angle croissant \\
    Initialiser une pile de points à vide \\
    Empiler $T_0$, puis~$T_1$ \\
\< {\em La pile contient toujours au moins deux points } \\
    $i$ := $2$ \\
    \WHILE\ $i < n$ \DO \\
    \INDENTER
        $\mbox{\em cour}$ := le point en sommet de pile \\
        $\mbox{\em prec}$ := le point encore en-dessous \\
        On considère le chemin défini par les trois points
                $\mbox{\em prec}$, $\mbox{\em cour}$ et~$T_i$ \\
        \IF\ ce chemin tourne à gauche en $\mbox{\em cour}$ \THEN \\
        \INDENTER
            Empiler $T_i$ \\
            $i$ := $i+1$ \\
        \< \ELSE \\
            Dépiler un point \-\\
        \ENDIF \-\\
    \ENDDO \\
    Le contenu de la pile constitue l'enveloppe convexe de~$T$.
    Afficher ce contenu \-\\
\END
\end{tabbing}
\caption{L'algorithme de Graham. Cet algorithme utilise une pile
de points. On peut montrer qu'à chaque itération, cette pile
contient au moins deux éléments.}
\label{algo-Graham}
\end{figure}

Appliquons le pseudo-code de la Figure~\ref{algo-Graham}
sur l'exemple de la Figure~~\ref{fig-enveloppe}.
Après le tri initial, le tableau~$T$ contient
$$A,\, C,\, H,\, I,\, J,\, F,\, D,\, E,\, B,\, G\,.$$
Les valeurs successives de la pile sont~:
$$\begin{array}{l}
A,\,C \\
A,\, C,\, H \\
A,\, C,\, I \\
A,\, C,\, I,\,J \\
A,\, C,\, I,\,J,\,F \\
A,\, C,\, I,\,J,\,D \\
A,\, C,\, I,\,J,\,D,\,E \\
A,\, C,\, I,\,J,\,B \\
A,\, C,\, I,\,J,\,B,\,G
\end{array}$$

\section{Réalisation}

\question
Coder le fichier \texttt{point.c} dont le fichier d'entête est 
ci-dessous (voir questions 3 et~4).
\begin{Verbatim}[fontsize=\small,samepage=true]
/* point.h */

#include <stdbool.h>

struct point {
    double x;   /* abscisse */
    double y;   /* ordonnée */
    char c;     /* identificateur */
};

extern void init_point (struct point*, double, double, char);
extern int compare_points (const void*, const void*);
extern bool tourne_a_gauche (struct point*, struct point*, struct point*);
\end{Verbatim}

\question
On souhaite implanter la pile de points de l'algorithme de Graham
au moyen d'une liste chaînée. 
Adapter le module \texttt{liste\_double} en un module \texttt{liste\_point}
qui permette de coder naturellement l'algorithme de Graham.
Faut-il ajouter des fonctions à ce module~?
\begin{Verbatim}[fontsize=\small,samepage=true]
/* liste_double.h */

struct maillon_double
{   double value;
    struct maillon_double* next;
};

struct liste_double
{   struct maillon_double* tete;
    int nbelem;
};

extern void init_liste_double (struct liste_double*);
extern void clear_liste_double (struct liste_double*);
extern void ajouter_en_tete_liste_double (struct liste_double*, double);
extern void extraire_tete_liste_double (double* d, struct liste_double* L);
\end{Verbatim}

\question
Compléter le code C ci-dessous afin d'implanter l'algorithme
de la Figure~\ref{algo-Graham}.
\begin{figure}
\begin{Verbatim}[fontsize=\small,samepage=true]
/* Graham.c */

#include <stdio.h>
#include <stdlib.h>
#include <assert.h>
#include "liste_point.h"

#define N 10
#define MAX_COORDINATES 100
#define SCENARIO 78738

int main ()
{   struct point T[N];
    struct liste_point L;
    FILE* f;
    int i;

    srand48 (SCENARIO);
/* 
 * On crée N points. Le point A est en (0,0).
 * On les enregistre dans "points.dat"
 */
    init_point (&T[0], 0, 0, 'A');
    for (i = 1; i < N; i++)
        init_point (&T[i], drand48 () * MAX_COORDINATES,
                           drand48 () * MAX_COORDINATES, 'A' + i);
    f = fopen ("points.dat", "w");
    assert (f != NULL);
    for (i = 0; i < N; i++)
        fprintf (f, "%f %f %c\n", T[i].x, T[i].y, T[i].c);
    fclose (f);
/* 
 * On trie T [1 .. N-1] par angle croissant.
 * On s'assure qu'il n'y a pas deux points alignés.
 */
    qsort (T+1, N-1, sizeof (struct point), &compare_points);
    for (i = 1; i < N-1; i++)
        assert (compare_points (&T[i], &T[i+1]) != 0);
/*
 * À FAIRE : BOUCLE PRINCIPALE DE L'ALGORITHME DE GRAHAM.
 *           UTILISER L POUR LA PILE DE POINTS.
 */
    f = fopen ("enveloppe.dat", "w");
    assert (f != NULL);
    fprintf (f, "%f %f\n", T[0].x, T[0].y);
/*
 * À FAIRE : IMPRIMER TOUS LES POINTS DE L DANS "enveloppe.dat"
 */
    fclose (f);
/* 
 * plot [-10:100][-10:100] "points.dat" with labels, "enveloppe.dat" with lines
 */
    clear_liste_point (&L);
    return 0;
}
\end{Verbatim}
\end{figure}

\bibliographystyle{plain}
\bibliography{tp4}
\end{document}



